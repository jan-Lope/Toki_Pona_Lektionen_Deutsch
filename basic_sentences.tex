%%%%%%%%%%%%%%%%%%%%%%%%%%%%%%%%%%%%%%%%%%%%%%%%%%%%%%%%%%%%%%%%%%%%%%%%%%
\section{Einfache Sätze}
%%%%%%%%%%%%%%%%%%%%%%%%%%%%%%%%%%%%%%%%%%%%%%%%%%%%%%%%%%%%%%%%%%%%%%%%%%
\subsection*{Vokabeln}
%%%%%%%%%%%%%%%%%%%%%%%%%%%%%%%%%%%%%%%%%%%%%%%%%%%%%%%%%%%%%%%%%%%%%%%%%%

\begin{supertabular}{p{2,5cm}|ll}
%
\index{jan}
\textbf{\dots jan} && \textit{Adjektiv}: persönlich, human, jemandens \\ % no-dictionary
\textbf{\dots jan} && \textit{Adverb}: persönlich, human, jemandens \\ % no-dictionary
\textbf{jan} && \textit{Substantiv}: Mensch, Person, Leute, Jemand \\ % no-dictionary
\textbf{jan (e \dots)} && \textit{Verb, transitiv}: personifizieren, verkörpern, vermenschlichen, personalisieren \\ % no-dictionary
 && \\ % no-dictionary
%
\index{li}
\textbf{\dots li \dots} && \textit{Separator}: Er trennt die Subjekt-Phrase, außer \textit{mi} und \textit{sina}, von der Prädikat-Phrase. \\ % && Verwende ein \glqq li\grqq nicht zusammmen mit einem anderen Separator. \\  % no-dictionary
 && \\ % no-dictionary
%
\index{mi}
\textbf{mi} && \textit{Personalpronomen}: ich, wir  \\ 
\textbf{\dots mi} && \textit{Possessivpronomen}: mein, unser \\  
\textbf{\dots e mi} && \textit{Reflexivpronomen}: mich, uns  \\ 
 && \\ % no-dictionary
%
\index{moku}
\textbf{\dots moku} && \textit{Adjektiv}: essend \\ % no-dictionary
\textbf{\dots moku} && \textit{Adverb}: essend \\ % no-dictionary
\textbf{moku} && \textit{Substantiv}: Essen, Lebensmittel, Nahrung, Nahrungsmittel, Mahlzeit, Fressen \\ % no-dictionary
\textbf{moku (e \dots)} && \textit{Verb, transitiv}: essen, trinken, schlucken, verzehren, verspeisen \\ % no-dictionary
 && \\ % no-dictionary
%
\index{ona}
\textbf{ona} && \textit{Personalpronomen}: er, sie, es, sie (Plural)  \\ % no-dictionary
\textbf{\dots ona} && \textit{Possessivpronomen}: seins, ihres, seine, deren \\  % no-dictionary
\textbf{\dots e ona} && \textit{Reflexivpronomen}: seins, ihres, seine, deren \\  
 && \\ % no-dictionary 
%
\index{pona}
\textbf{\dots pona} && \textit{Adjektiv}: gut, einfach, positiv, schön, richtig, korrekt, toll \\ % no-dictionary
\textbf{\dots pona} && \textit{Adverb}: gut, einfach, schön, richtig, korrekt, toll \\ % no-dictionary
\textbf{pona} && \textit{Substantiv}: das Gute, Einfachheit, Positivismus \\ % no-dictionary
\textbf{pona (e \dots)} && \textit{Verb, transitiv}: verbessern, fixen, reparieren, festmachen \\ % no-dictionary
 && \\ % no-dictionary
%
\index{sina}
\textbf{sina} && \textit{Personalpronomen}: du, ihr (Plural), euch  \\ % no-dictionary
\textbf{\dots sina} && \textit{Possessivpronomen}: deins, euer  \\  % no-dictionary
\textbf{\dots e sina} && \textit{Reflexivpronomen}: dich, euch  \\  
 && \\ % no-dictionary
%
\index{suno}
\textbf{\dots suno} && \textit{Adjektiv}: sonnig \\ % no-dictionary
\textbf{\dots suno} && \textit{Adverb}: sonnig \\ % no-dictionary
\textbf{suno} && \textit{Substantiv}: Sonne, Licht, Helligkeit \\ % no-dictionary
\textbf{suno (e \dots)} && \textit{Verb, transitiv}: scheinen, beleuchten \\ % no-dictionary
 && \\ % no-dictionary
%
\index{suli}
\textbf{\dots suli} && \textit{Adjektiv}: groß, schwer, wichtig, lang, erwachsen \\ % no-dictionary
\textbf{\dots suli} && \textit{Adverb}: schwer, wichtig, erwachsen \\ % no-dictionary
\textbf{suli} && \textit{Substantiv}: Größe, Format \\ % no-dictionary
\textbf{suli (e \dots)} && \textit{Verb, transitiv}: vergrößern, ausbauen, ausdehnen, verlängern \\ % no-dictionary
 && \\ % no-dictionary
%
\index{telo}
\textbf{\dots telo} && \textit{Adjektiv}: feucht, klebrig, verschwitzt \\ % no-dictionary
\textbf{\dots telo} && \textit{Adverb}: feucht, klebrig, verschwitzt \\ % no-dictionary
\textbf{telo} && \textit{Substantiv}: Wasser, Flüssigkeit, Feuchtigkeit, Saft, Soße \\ % no-dictionary
\textbf{telo (e \dots)} && \textit{Verb, transitiv}: begießen, bewässern, befeuchten, waschen, schmelzen \\ % no-dictionary
 && \\ % no-dictionary
%
\index{Apostroph}
\textbf{'} && \textit{inoffiziell}: Ein Apostroph kann ein Prädikat, dass kein Verb enthält, kennzeichnen. \\ % no-dictionary
\end{supertabular} \\
%
%%%%%%%%%%%%%%%%%%%%%%%%%%%%%%%%%%%%%%%%%%%%%%%%%%%%%%%%%%%%%%%%%%%%%%%%%%
\newpage
%
\subsection*{Die Mehrdeutigkeit von \textit{toki pona}}
%
\index{Mehrdeutigkeit}
%%%%%%%%%%%%%%%%%%%%%%%%%%%%%%%%%%%%%%%%%%%%%%%%%%%%%%%%%%%%%%%%%%%%%%%%%%
%
\index{\textit{suli}}
Viele Wörter in \textit{toki pona} haben mehrere Bedeutungen. 
Zum Beispiel kann \textit{suli} sowohl 'lang', 'hoch', 'groß', 'wichtig' oder 'die Größe' bedeuten. 
Wie kann ein Wort so viele verschiedene Bedeutungen haben?

Willkommen in der Welt von \textit{toki pona}! 
Die meisten Wörter in \textit{toki pona} sind mehrdeutig, weil diese Sprache so wenig Vokabeln hat.
Wie auch immer, diese Vieldeutigkeit ist nicht unbedingt schlecht: 
Gerade wegen der verschwommenen Bedeutungen ist ein \textit{toki pona}-Sprecher dazu gezwungen, sich auf die konkreten Aspekte zu konzentrieren und sich nicht in unwichtigen Details zu verlieren.

\index{Singular}
\index{Plural}
Eine weitere Form von Mehrdeutigkeit ist das Fehlen von Singular oder Plural. 
So bedeutet \textit{jan} sowohl eine Person oder Leute. 
\textit{toki pona} ist nicht die einzige Sprache, die bei den Formen der Substantive
nicht zwischen Singular oder Plural unterscheidet. 
Die japanische Sprache verfährt ebenso. 

\index{Zeitform}
Weiterhin gibt es keine Zeitformen, d.~h. alle Verben bleiben gleich.
Es gibt auch andere Sprachen, die keine Zeitformen haben.
Wir werden aber Möglichkeiten kennenlernen, um mit Zeiten zu arbeiten.

\index{Wortart}
Wie in der Vokabel-Liste zu sehen ist, können die meisten Wörter als unterschiedliche Wortarten verwendet werden. 
Dabei bleiben sie unverändert. 
Die Wortart ergibt sich aus der Position im Satz. 
In dieser Lektion werden wir uns mit Substantiven, Pronomen, Verben, Adjektiven und einem speziellen Trennwort (Separator) beschäftigen. 

\index{Substantiv}
\index{Adjektiv}
\index{Verb}
Ein Substantiv ist ein Wort für eine Person, Ort oder Ding. 
Ein Adjektiv ist ein Wort, das ein Substantiv beschreibt. 
Ein Verb beschreibt eine Aktion. 

\index{Pronomen}
\index{Fürwort}
\index{Pronomen!Personal-}
\index{Pronomen!Possesiv-}
\index{Personalpronomen}
\index{Possessivpronomen}
Pronomen (Fürwörter) sind Stellvertreter für verschiedene Arten von Wörtern. 
Sie werden an der gleichen Stelle wie das zu vertretene Wort verwendet und haben die gleichen grammatischen Merkmale wie dieses.
Pronomen sind keine Inhaltswörter, sondern bezeichnen Personen oder Dinge durch Verweis auf den Kontext. 
Personalpronomen (ich, du, \dots) vertreten Substantive. 
Possessivpronomen (mein, dein, \dots) vertreten Adjektive. 
In den nächsten Lektionen werden wir die weiteren Arten von Pronomen kennenlernen. 

%
%%%%%%%%%%%%%%%%%%%%%%%%%%%%%%%%%%%%%%%%%%%%%%%%%%%%%%%%%%%%%%%%%%%%%%%%%%
\label{'predicate'}
\subsection*{Die Personalpronomen \textit{mi} oder \textit{sina} als Subjekt}
%
\index{Nebensätze!verschachtelt}
\index{Komma}
\index{Aussagesatz}
\index{Punkt}
\index{\textit{mi!Personalpronom}}
\index{\textit{sina!Personalpronom}}
\index{\textit{pona}}
\index{\textit{moku}}
%%%%%%%%%%%%%%%%%%%%%%%%%%%%%%%%%%%%%%%%%%%%%%%%%%%%%%%%%%%%%%%%%%%%%%%%%%
%

Mit dem Personalpronomen \textit{mi} oder dem Personalpronomen \textit{sina} am Anfang und einem nachfolgenden Verb ist bereits ein einfacher Satz in \textit{toki pona} komplett. 
Beendet wird ein Aussagesatz mit einem Punkt.
\textit{toki pona} kennt keine verschachtelten Nebensätze und fast keine Kommas.

\begin{supertabular}{p{5,5cm}|ll}
mi moku. && Ich esse. \\
sina pona. && Du reparierst. \\
\end{supertabular} 

\index{Subjekt-Phrase}
In diesen Sätzen sind die Personalpronomen \textit{mi} und \textit{sina} jeweils die Subjekt-Phrase. 
In \textit{toki pona} steht eine Subjekt-Phrase immer am Anfang des Satzes. 
In diesen Beispielen bestehen die Subjekt-Phrasen jeweils nur aus einem Subjekt (\textit{mi} bzw. \textit{sina}).

Das Subjekt ist der Träger der Handlung, des Vorgangs oder des Zustands. 
Es ist die wichtigste Ergänzung des Verbs im Satz, ein vollständiger Satz enthält immer ein Subjekt. 
Nach dem Subjekt fragt man mit wer oder was.

%
%%%%%%%%%%%%%%%%%%%%%%%%%%%%%%%%%%%%%%%%%%%%%%%%%%%%%%%%%%%%%%%%%%%%%%%%%%
%% \newpage{}
\subsection*{Verben als Prädikat}
%
\index{Prädikat}
\index{Satzaussage}
\index{Prädikat!kein Verb}
\index{Verb vs. Prädikat}
%%%%%%%%%%%%%%%%%%%%%%%%%%%%%%%%%%%%%%%%%%%%%%%%%%%%%%%%%%%%%%%%%%%%%%%%%%
%
Die Verben \textit{moku} und \textit{pona} bilden in diesen Beispielen das Prädikat.  
Das Prädikat ist ein Kernbestandteil in einem Satz und die Aussage des Satzes. 

In den meisten Sprachen wird ein Prädikat durch ein Verb gebildet, dies ist jedoch nicht in allen Sprachen zwingend. 
Wie wir gleich sehen werden, wird auch in \textit{toki pona} das Prädikat nicht zwingend durch ein Verb gebildet. 
Der Unterschied zwischen den Begriffen Verb und Prädikat liegt also darin, dass Verb eine Wortart bezeichnet, und Prädikat eine grammatische Funktion.
Ein Prädikat und mögliche Objekte bilden eine Prädikat-Phrase. 
%
%%%%%%%%%%%%%%%%%%%%%%%%%%%%%%%%%%%%%%%%%%%%%%%%%%%%%%%%%%%%%%%%%%%%%%%%%%
%% \newpage{}
\subsection*{Substantive oder Adjektive als Prädikat}
%
\index{Apostroph}
\index{sein}
\index{Slot}
\index{\textit{mi!Personalpronom}}
\index{\textit{sina!Personalpronom}}
\index{\textit{pona}}
\index{\textit{moku}}
\index{No-Copula-Sprache}
%%%%%%%%%%%%%%%%%%%%%%%%%%%%%%%%%%%%%%%%%%%%%%%%%%%%%%%%%%%%%%%%%%%%%%%%%%
% https://www.cafe-lingua.de/deutsche-grammatik/subjektspraedikativ.php
% https://de.wikipedia.org/wiki/Nominalsatz
% http://nualeargais.ie/gnag/kopul4.htm
%
Eine der ersten Prinzipien ist, dass es das statitsche Verb 'sein' nicht gibt. 
Dadurch kann der Verb-Slot unbelegt sein und nach \textit{mi} oder \textit{sina} kann auch ein Substantiv oder Adjektiv folgen. 
In diesen Lektionen wird der Begriff 'Slot' verwendet, um eine gültige Position eines Worttyps im Satz anzugeben.

Auch in anderen Sprachen können reguläre Sätze gebildet werden, ohne dass ein Verb darin erscheint. 
Beispiele hierfür sind Russisch und Arabisch. 
Man nennt diese Sprachen No-Kopula-Sprachen. 
Eine Kopula ist ein Wort, das Subjekt und Prädikat verbindet ('kopuliert').
Wenn ein 'normales' Verb das Prädikat bildet, braucht man keine zusätzliche Kopula.
Sie tritt nur auf, wenn ein Substantiv, Pronomen oder Adjektiv das Prädikat bildet.
Im Deutschen dient das Verb "sein" als Kopula. 
No-Kopula-Sprache, wie Toki Pona, benötigen keine Kopula. 

Ein Substantiv funktioniert dann als Prädikatsnomen oder ein Adjektiv als Prädikatsadjektiv.
Das Substantiv bzw. Adjektiv wird damit aber nicht zum Verb. 
Der leere Verb-Slot kann aber nicht alleine eine Prädikat-Phrase bilden. 
Es muss ein Substantiv oder Adjektiv folgen. 
Das heißt, direkt nach \textit{mi} oder \textit{sina} kann der Satz noch nicht beendet werden. 

Bei den No-Copula-Sprachen erkennt man meist an der Wortform, ob es sich beim Prädikat um ein Verb, Substantiv oder Adjektiv handelt. 
In Toki Pona ist das nicht möglich. 
In diesen Lektionen wird mit einem Apostrophe ein nachfolgendes Substantiv beziehungsweise Adjektiv gekennzeichnet. 
Das ist aber kein offizielle Regel. 

\begin{supertabular}{p{5cm}|ll}
mi moku. && Ich esse.  \\
mi ' moku. && Ich bin eine Mahlzeit. \\ 
sina pona. && Du reparierst. \\   
sina ' pona. && Du bist gut. \\  
\end{supertabular} 

Ohne das Wort 'sein' geht die genaue Bedeutung verloren. 
\textit{moku} kann in diesem Satz ein Verb oder ein Substantiv bzw. \textit{pona} kann 
ein Adjektiv oder Verb sein. 
In solchen Situationen muss der Zuhörer auf den Kontext achten. 
Wie oft hast du schon etwas wie 'Ich bin eine Mahlzeit.' gehört? 
Ich hoffe nicht sehr oft. 
Also kannst du dir ziemlich sicher sein, dass \textit{mi moku} 'ich esse' bedeutet.
%
%%%%%%%%%%%%%%%%%%%%%%%%%%%%%%%%%%%%%%%%%%%%%%%%%%%%%%%%%%%%%%%%%%%%%%%%%%
\subsection*{Der Separator \textit{li}}
%
\index{Separator}
\index{Prädikatmarker}
\index{\textit{li}}
\index{\textit{ona!Personalpronom}}
%%%%%%%%%%%%%%%%%%%%%%%%%%%%%%%%%%%%%%%%%%%%%%%%%%%%%%%%%%%%%%%%%%%%%%%%%%
%
Für Sätze, die nicht die Pronomen \textit{mi} oder \textit{sina} als Subjekt-Phrase verwenden, gibt es das Wort \textit{li}. 
\textit{li} ist ein grammatikalisches Wort, dass die Subjekt-Phrase von der Prädikat-Phrase trennt. 
Der Prädikatmarker \textit{li} wird nur benutzt, wenn die Subjekt-Phrase nicht \textit{mi} oder \textit{sina} ist. 
Auch wenn dir der Separator \textit{li} jetzt sinnlos erscheint, wirst du später feststellen, dass
die Sätze ziemlich verwirrend werden können, wenn es \textit{li} nicht gäbe. 

\begin{supertabular}{p{5cm}|ll}
telo li pona. && Wasser reinigt. \\
suno li suno. && Die Sonne scheint. \\
moku li ' pona. && Das Essen ist gut. \\
ona li ' moku. && Es ist Essen. \\
\end{supertabular} 

Ist der Verb-Slot leer, kann nach \textit{li} ein Substantiv oder Adjektiv folgen. 
Wie schon geschrieben, kann  ein leerer Verb-Slot nicht alleine eine Prädikat-Phrase bilden. 
Es muss ein Substantiv oder Adjektiv folgen. 
Das heißt, direkt nach \textit{li} kann der Satz noch nicht beendet werden oder ein Objekt folgen.
%
\newpage
%%%%%%%%%%%%%%%%%%%%%%%%%%%%%%%%%%%%%%%%%%%%%%%%%%%%%%%%%%%%%%%%%%%%%%%%%%
\subsection*{Übungen (Antworten siehe Seite~\pageref{'basic_sentences'})}
%%%%%%%%%%%%%%%%%%%%%%%%%%%%%%%%%%%%%%%%%%%%%%%%%%%%%%%%%%%%%%%%%%%%%%%%%%
%

Schreibe bitte die Antworten auf einen Zettel und überprüfe sie anschließend. 

\begin{supertabular}{p{5,5cm}|ll}
Was ist ein Verb? &&  \\  % no-dictionary
Was ist ein Substantiv? &&  \\  % no-dictionary
Wozu dient \textit{li}? &&   \\ % no-dictionary
Was vertritt ein Personalpronomen? &&  \\ % no-dictionary
Wie erkennt man Substantive, Pronomen, Verben und Adjektive in \textit{toki pona}? &&  \\  % no-dictionary
Was ist ein Subjekt? &&  \\ % no-dictionary
Nach welchen Subjekt-Phrasen wird \textit{li} nicht benutzt? && \\ % no-dictionary
Wo steht das Subjekt im Satz? &&  \\  % no-dictionary
Kann ein leerer Verb-Slot alleine ein Prädikat bilden? &&  \\  % no-dictionary
Wann kann ein Verb-Slot leer sein? &&   \\  % no-dictionary
Was ist ein Prädikat? &&   \\  % no-dictionary
Ein vollständiger Satz in \textit{toki pona} enthält immer\dots &&  \\  % no-dictionary
Aus welchen Wortarten kann in \textit{toki pona} ein Prädikat gebildet werden? &&  \\  % no-dictionary
Was ist ein Adjektiv? &&   \\ % no-dictionary
Wo befinden sich mögliche Adjektiv-Slots? &&  \\  % no-dictionary
Warum kann nach \textit{li} ein Satz nicht beendet werden? &&  \\ % no-dictionary
\end{supertabular} 

Welche Wortarten kann das jeweilige Wort in dem Satz nach dem Bindestrich darstellen?
Beispiel: 

\begin{supertabular}{p{5,5cm}|ll}
mi - mi moku. && Personalpronomen \\ % no-dictionary
\end{supertabular}

\begin{supertabular}{p{5,5cm}|ll}
sina - sina pona. &&  \\ % no-dictionary
moku - moku li ' pona. &&  \\ % no-dictionary
ona - ona li ' moku. &&  \\ % no-dictionary
li - moku li ' pona. &&  \\ % no-dictionary
\end{supertabular}

Versuche diese Sätze zu übersetzen. 
Mit dem Tool \textit{Toki Pona Parser} (\cite{www:rowa:02}) kann man Rechtschreibung und Grammatik überprüfen. 

\begin{supertabular}{p{5,5cm}|ll}
Menschen sind gut. && \\ % no-dictionary
Ich esse. &&  \\ % no-dictionary
Du bist groß. &&  \\ % no-dictionary
Wasser ist einfach. &&  \\ % no-dictionary
Der See ist groß. && \\ % no-dictionary
\end{supertabular}

\begin{supertabular}{p{5,5cm}|ll}
suno li ' suli. &&  \\% no-dictionary
mi ' suli. &&  \\% no-dictionary
jan li moku. &&  \\% no-dictionary
\end{supertabular} \\% no-dictionary
%
%%%%%%%%%%%%%%%%%%%%%%%%%%%%%%%%%%%%%%%%%%%%%%%%%%%%%%%%%%%%%%%%%%%%%%%%%%
% eof
