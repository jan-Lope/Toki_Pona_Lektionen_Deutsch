%%%%%%%%%%%%%%%%%%%%%%%%%%%%%%%%%%%%%%%%%%%%%%%%%%%%%%%%%%%%%%%%%%%%%%%%%%
\section{Einfache S�tze}
%%%%%%%%%%%%%%%%%%%%%%%%%%%%%%%%%%%%%%%%%%%%%%%%%%%%%%%%%%%%%%%%%%%%%%%%%%
\subsection*{Vokabeln}
%%%%%%%%%%%%%%%%%%%%%%%%%%%%%%%%%%%%%%%%%%%%%%%%%%%%%%%%%%%%%%%%%%%%%%%%%%
\index{\textit{jan}}
\index{\textit{mi}}
\index{\textit{moku}}
\index{\textit{sina}}
\index{\textit{suno}}
\index{\textit{telo}}
\index{\textit{pona}}
\index{\textit{suli}}
\index{\textit{li}}
\index{ich}
\index{wir}
\index{mir}
\index{mich}
\index{mein}
\index{meins}
\index{unser}
\index{du}
\index{dein}
\index{deins}
\index{ihr}
\index{euch}
\index{eurer}
\index{jemand}
\index{irgendwer}
\index{Person}
\index{Mensch}
\index{Leute}
\index{pers�nlich}
\index{Trennwort!\textit{li}}
\index{gut}
\index{einfach}
\index{reparieren}
\index{festmachen}
\index{Positivismus}
\index{essen}
\index{Mahlzeit}
\index{trinken}
\index{Getr�nk}
\index{einnehmen}
\index{Sonne}
\index{Licht}
\index{Wasser}
\index{Fl�ssigkeit}
\index{waschen}
\index{bew�ssern}
\index{gro�}
\index{hoch}
\index{lang}
\index{wichtig}
\index{vergr��ern}
\index{Gr��e}
\begin{supertabular}{p{5,5cm}|ll}
mi &&  ich, wir, mir, mich, mein, meins, unser  \\
sina &&  du, dein, deins, ihr, euch, eurer  \\
jan && jemand, irgendwer, Person(en), Mensch(en), Leute, pers�nlich  \\
li && Trennwort: Es trennt das Dritte-Person-Subjekt vom Verb im Satz. \\
pona &&  gut, einfach, reparieren, festmachen, Positivismus \\
moku &&  essen, Essen, Mahlzeit, trinken, Getr�nk, einnehmen \\
suno &&  Sonne, Licht, leuchten, scheinen \\
telo &&  Wasser, Fl�ssigkeit, waschen, bew�ssern  \\
suli &&  gro�, hoch, lang, wichtig, vergr��ern, Gr��e \\
\end{supertabular} 
%
%%%%%%%%%%%%%%%%%%%%%%%%%%%%%%%%%%%%%%%%%%%%%%%%%%%%%%%%%%%%%%%%%%%%%%%%%%
\index{Singular}
\index{Plural}
\subsection*{Die Mehrdeutigkeit von \textit{toki pona}}
%%%%%%%%%%%%%%%%%%%%%%%%%%%%%%%%%%%%%%%%%%%%%%%%%%%%%%%%%%%%%%%%%%%%%%%%%%
%
Viele W�rter in toki pona haben mehrere Bedeutungen. 
Zum Beispiel kann \textit{suli} sowohl 'lang', 'hoch', 'gro�' oder 'wichtig' bedeuten. 
Wie kann ein Wort so viele verschiedene Bedeutungen haben?

Willkommen in der Welt von \textit{toki pona}. 
Die meisten W�rter in \textit{toki pona} sind mehrdeutig weil diese Sprache so wenig
Vokabeln hat.
Wie auch immer, diese Vieldeutigkeit ist nicht unbedingt schlecht: 
Gerade wegen der verschwommenen Bedeutungen ist ein \textit{toki pona} Sprecher dazu gezwungen, sich auf die konkreten Aspekte zu konzentrieren und sich nicht in unwichtigen Details zu verlieren.

Eine weitere Form von Mehrdeutigkeit ist das Fehlen von Singular oder Plural. 
So bedeutet \textit{jan} sowohl eine Person oder Leute. 
\textit{toki pona} ist nicht die einzige Sprache, die bei den Formen der Substantive
nicht zwischen Singular oder Plural unterscheidet. 
Die japanische Sprache verf�hrt ebenso. 

Weiterhin gibt es keine Zeitformen, d.h. alle Verben bleiben gleich.
Es gibt auch andere Sprachen, die keine Zeitformen haben.
Wir werden aber M�glichkeiten kennenlernen, um mit Zeiten zu arbeiten.
%
%%%%%%%%%%%%%%%%%%%%%%%%%%%%%%%%%%%%%%%%%%%%%%%%%%%%%%%%%%%%%%%%%%%%%%%%%%
\index{\textit{mi}}
\index{\textit{sina}}
\index{\textit{mi}!Subjekt}
\index{\textit{sina}!Subjekt}
\index{Subjekt!\textit{mi}}
\index{Subjekt!\textit{sina}}
\index{Verb!sein}
\index{sein}
\subsection*{S�tze mit \textit{mi} oder \textit{sina} als Subjekt}
%%%%%%%%%%%%%%%%%%%%%%%%%%%%%%%%%%%%%%%%%%%%%%%%%%%%%%%%%%%%%%%%%%%%%%%%%%
%
Mit \textit{mi} oder \textit{sina} am Anfang und einem Verb ist bereits ein einfacher Satz in toki pona komplett. 
Beendet wird ein Aussagesatz mit einem Punkt.
Toki Pona kennt keine verschachtelte Nebens�tze und fast keine Kommas.

\begin{supertabular}{p{5,5cm}|ll}
mi moku. && Ich esse. \\
sina pona. suli. && Du reparierst. \\
\end{supertabular} 
%
%%%%%%%%%%%%%%%%%%%%%%%%%%%%%%%%%%%%%%%%%%%%%%%%%%%%%%%%%%%%%%%%%%%%%%%%%%
\newpage
\subsection*{Das fehlende Verb 'sein'}
%%%%%%%%%%%%%%%%%%%%%%%%%%%%%%%%%%%%%%%%%%%%%%%%%%%%%%%%%%%%%%%%%%%%%%%%%%
%
Eine der ersten Prinzipien ist, da� es das Verb 'sein' nicht gibt.
Dadurch kann nach \textit{mi} oder \textit{sina} auch ein Substantiv oder Adjektiv folgen. 
Das fehlende Verb 'sein' ist hier zur Verdeutlichung mit einem Apostroph gekennzeichnet. 

\begin{supertabular}{p{5cm}|ll}
mi moku. && Ich esse.  \\
mi ' moku. && Ich bin eine Mahlzeit. \\  % no-dictionary
sina pona. && Du reparierst. \\   
sina ' pona. && Du bist gut. \\  % no-dictionary
\end{supertabular} 

Ohne das Wort 'sein' geht die genaue Bedeutung verloren. 
\textit{moku} kann in diesem Satz ein Verb oder ein Substantiv bzw. \textit{pona} kann 
ein Adjektiv oder Verb sein. 
In solchen Situationen mu� der Zuh�rer auf den Kontext achten. 
Wie oft hast du schon etwas wie 'Ich bin eine Mahlzeit.' geh�rt? 
Ich hoffe nicht sehr oft. 
Also kannst du dir ziemlich sicher sein, da� \textit{mi moku}, 'Ich esse.' bedeutet.
%
%%%%%%%%%%%%%%%%%%%%%%%%%%%%%%%%%%%%%%%%%%%%%%%%%%%%%%%%%%%%%%%%%%%%%%%%%%
\index{\textit{li}}
\index{Subjekt!\textit{mi}}
\index{Subjekt!\textit{sina}}
\index{\textit{mi}!Subjekt}
\index{\textit{sina}!Subjekt}
\subsection*{S�tze ohne \textit{mi} oder \textit{sina} als Subjekt}
%%%%%%%%%%%%%%%%%%%%%%%%%%%%%%%%%%%%%%%%%%%%%%%%%%%%%%%%%%%%%%%%%%%%%%%%%%
%
F�r S�tze, die nicht \textit{mi} oder \textit{sina} als Subjekt verwenden, gibt es das Wort \textit{li}. 
\textit{li} ist ein grammatikalisches Wort, da� ein Subjekt von seinem Verb trennt. 
Es wird nur benutzt, wenn das Subjekt nicht \textit{mi} oder \textit{sina} ist. 
Auch wenn dir \textit{li} jetzt sinnlos erscheinen wirst du sp�ter feststellen, da�
die S�tze ziemlich verwirrend werden k�nnen, wenn es \textit{li} nicht g�be. 

\begin{supertabular}{p{5cm}|ll}
telo \textbf{li} pona. && Wasser reningt. \\
suno \textbf{li} suno. && Die Sonne scheint. \\
moku \textbf{li} ' pona. && Das Essen ist gut. \\
\end{supertabular} 

Da es das Verb 'sein' nicht gibt, kann auch nach \textit{li} nicht nur ein Verb , sondern auch ein Substantiv oder Adjektiv folgen. 
Das fehlende Verb 'sein' ist hier zur Verdeutlichung wieder mit einem Apostroph gekennzeichnet. 
%
%%%%%%%%%%%%%%%%%%%%%%%%%%%%%%%%%%%%%%%%%%%%%%%%%%%%%%%%%%%%%%%%%%%%%%%%%%
\subsection*{�bungen 3 (Antworten siehe Seite~\pageref{'basic_sentences'})}
%%%%%%%%%%%%%%%%%%%%%%%%%%%%%%%%%%%%%%%%%%%%%%%%%%%%%%%%%%%%%%%%%%%%%%%%%%
%
�bersetze!

\begin{supertabular}{p{5cm}|ll}
Menschen sind gut. && \\ % no-dictionary
Ich esse. &&  \\ % no-dictionary
Du bist gro�. &&  \\ % no-dictionary
Wasser ist einfach. &&  \\ % no-dictionary
Der See ist gro�. &&\\ % no-dictionary
 && \\
suno li suli. &&  \\% no-dictionary
mi suli. &&  \\% no-dictionary
jan li moku. &&  \\% no-dictionary
\end{supertabular} \\% no-dictionary
%
%%%%%%%%%%%%%%%%%%%%%%%%%%%%%%%%%%%%%%%%%%%%%%%%%%%%%%%%%%%%%%%%%%%%%%%%%%
% eof
