%%%%%%%%%%%%%%%%%%%%%%%%%%%%%%%%%%%%%%%%%%%%%%%%%%%%%%%%%%%%%%%%%%%%%%%%%%
\section{Präpositionalobjekte}
%%%%%%%%%%%%%%%%%%%%%%%%%%%%%%%%%%%%%%%%%%%%%%%%%%%%%%%%%%%%%%%%%%%%%%%%%%
%
\subsection*{Vokabeln}
%%%%%%%%%%%%%%%%%%%%%%%%%%%%%%%%%%%%%%%%%%%%%%%%%%%%%%%%%%%%%%%%%%%%%%%%%%
%
\begin{supertabular}{p{2,5cm}|ll}
%
\index{ali}
\textbf{\dots ali} && \textit{Adjektiv}: alle(r/s), je, jeweils, komplett, ganz \\ % no-dictionary
\textbf{\dots ali} && \textit{Adverb}: alle(r/s), je, jeweils, komplett, ganz \\ % no-dictionary
\textbf{ali} && \textit{Substantiv}: Alles, Etwas, Leben, Universum, Welt \\ % no-dictionary
 && \\ % no-dictionary
%
\index{pipi}
\textbf{pipi} && \textit{Substantiv}: Insekt, Käfer, Spinne, Skorpion, Krebs, Krabbe, Schmetterling, Ameise \\  % no-dictionary
 && \\ % no-dictionary
%
\index{sama}
\textbf{\dots sama} && \textit{Adjektiv}: gleichaltrig, ähnlich, paritätisch \\ % no-dictionary
\textbf{\dots sama} && \textit{Adverb}: ähnlich \\ % no-dictionary
\textbf{sama} && \textit{Substantiv}: Gleichheit, Parität, Identität \\ % no-dictionary
\textbf{\dots , sama \dots} && \textit{Präposition}: wie, dergleichen \\ % no-dictionary
\textbf{sama (e \dots)} && \textit{Verb, transitiv}: gleichsetzen, gleich machen, nachäffen \\ % no-dictionary
 && \\ % no-dictionary
%
\index{sona}
\textbf{\dots sona} && \textit{Adjektiv}: wissend, verständnisvoll, klug, schlau, clever \\ % no-dictionary
\textbf{sona} && \textit{Substantiv}: Kenntnis, Wissen, Erkenntnis, Weisheit, Intelligenz, Verständnis \\ % no-dictionary
\textbf{sona} && \textit{Verb, intransitiv}: wissen, können, verstehen \\ % no-dictionary
\textbf{sona (e \dots)} && \textit{Verb, transitiv}: wissen, können, verstehen, zu etwas fähig sein \\ % no-dictionary
kama \textbf{sona (e \dots)} && \textit{Verb, transitiv}: lernen, studieren \\ % no-dictionary
\textbf{sona \dots} && \textit{Hilfsverb}: wissen \\ % no-dictionary
 && \\ % no-dictionary
%
\index{tan}
\textbf{\dots tan} && \textit{Adjektiv}: kausal, ursächlich \\ % no-dictionary
\textbf{tan} && \textit{Substantiv}: Ursprung, Abstammung, Herkunft, Entstehung, Grund, Ursache \\ % no-dictionary
\textbf{\dots , tan \dots} && \textit{Präposition}: von, aus, durch, wegen, weil, seit \\ % no-dictionary
\textbf{tan} && \textit{Verb, intransitiv}: stammen aus, kommen aus \\ % no-dictionary
\end{supertabular}
%%%%%%%%%%%%%%%%%%%%%%%%%%%%%%%%%%%%%%%%%%%%%%%%%%%%%%%%%%%%%%%%%%%%%%%%%%
%
%%%%%%%%%%%%%%%%%%%%%%%%%%%%%%%%%%%%%%%%%%%%%%%%%%%%%%%%%%%%%%%%%%%%%%%%%%
\newpage
%
\subsection*{Präpositionalobjekte und Präpositionen}
%
\index{Objekt!präpositional}
\index{Präpositionalobjekt}
\index{Präposition}
%%%%%%%%%%%%%%%%%%%%%%%%%%%%%%%%%%%%%%%%%%%%%%%%%%%%%%%%%%%%%%%%%%%%%%%%%%

Die dritte Objektklasse in \textit{toki pona} ist das Präpositionalobjekt. 
Ein Präpositionalobjekt beginnt mit einer Präposition. 
Eine Präposition beschreibt eine Beziehung zwischen Wörtern in einem Satz und steht vor Substantiven oder Pronomen. 
Es ist eng mit dem Verb verbunden. 
Die Präposition bestimmt den Kasus (Fall). 
Die Frage nach dem Präpositionalobjekt ist abhändig von der verwendeten Präposition. 
In \textit{toki pona} befindet sich ein Slot für Präpositionen nur am Anfang eines Präpositionalobjekts. 
Es wird empfohlen, ein Komma vor einer Präposition zu setzen. 
Das ist aber keine offizielle Regel. 

Im Präpositionalobjekt ist der erste Slot nach der Präposition immer ein Substantiv- oder Pronomen-Slot.
Danach sind optionale Slots für Adjektive, Possessivpronomen und Demonstrativpronomen möglich. 
In \textit{toki pona} steht ein optionales Präpositionalobjekt am Ende eines Satzes. 
Mögliche direkte Objekte oder indirekte Objekte stehen immer vor einem Präpositionalobjekt. 
Wie die anderen Objektarten ist ein Präpositionalobjekt ein optionaler Bestandteil einer Prädikat-Phrase. 

%%%%%%%%%%%%%%%%%%%%%%%%%%%%%%%%%%%%%%%%%%%%%%%%%%%%%%%%%%%%%%%%%%%%%%%%%%
\index{\textit{kepeken}!Präposition}
%
Die Präposition \textit{kepeken} bedeutet 'mit', 'mittels' oder 'per'.

\begin{supertabular}{p{5,5cm}|ll}
mi moku, kepeken ilo moku. && Ich esse mit Besteck. \\
mi lukin, kepeken ilo suno. && Ich sehe mittels Taschenlampe.  \\
\end{supertabular} 

%
%%%%%%%%%%%%%%%%%%%%%%%%%%%%%%%%%%%%%%%%%%%%%%%%%%%%%%%%%%%%%%%%%%%%%%%%%%
\index{Komma}
\index{\textit{lon}!Präposition}
\index{\textit{wile}!Verb}
\index{\textit{wile}!Hilfsverb}
\index{\textit{ni}}
%%%%%%%%%%%%%%%%%%%%%%%%%%%%%%%%%%%%%%%%%%%%%%%%%%%%%%%%%%%%%%%%%%%%%%%%%%
%
Die Präposition \textit{lon} bedeutet 'in', 'im', 'an', 'am', 'bei' oder 'auf'.

\begin{supertabular}{p{5,5cm}|ll}
mi moku, lon tomo. && Ich esse im Haus. \\
mi telo e mi, lon tomo telo. && Ich wasche mich im Badezimmer. \\
\end{supertabular} 

Da es sowohl die Präposition \textit{lon} auch das intransitive Verb \textit{lon} gibt, ist möglicherweise die Aussage folgender Sätze verwirrend.

\begin{supertabular}{p{5,5cm}|ll}
mi wile lon tomo. && Ich will zu Hause sein. / Ich will im Haus. \\
\end{supertabular} 

Dieser Satz hat mindestens zwei mögliche Übersetzungen. 
Die erste Übersetzung drückt aus, dass der Sprecher zu Hause sein will. 
Die zweite Übersetzung sagt aus, dass er in einem Haus ist und etwas will (ohne zu sagen, was genau). 
Nach einem Komma ist hier nur die Präposition \textit{lon} möglich.

\begin{supertabular}{p{5,5cm}|ll}
mi wile, lon tomo. && Ich will im Haus.  \\
\end{supertabular}

Wenn man sagen möchte 'ich will zu Hause sein', muss man den Satz mit einem Doppelpunkt in zwei Sätze aufteilen.

\begin{supertabular}{p{5,5cm}|ll}
mi wile e ni: mi lon tomo. && Ich will dies: Ich bin zu Hause. \\
\end{supertabular} 

In \textit{toki pona} wird oft dieser Trick verwendet. 
Beachte dabei, dass jeweils vor und nach dem Doppelpunkt ein vollständiger Satz ist. 
\textit{toki pona} kennt keine verschachtelten Nebensätze.

\begin{supertabular}{p{5,5cm}|ll}
sina toki e ni, tawa mi: sina moku. && Du sagtest zu mir, dass Du isst. \\
\end{supertabular} 

%%%%%%%%%%%%%%%%%%%%%%%%%%%%%%%%%%%%%%%%%%%%%%%%%%%%%%%%%%%%%%%%%%%%%%%%%%
\index{\textit{tawa}!Präposition}
\index{\textit{tawa}!Verb}
\index{\textit{tawa}!Adjektiv}
%
In dem letzten Satz steht nach dem Komma die Präposition \textit{tawa}. 

\begin{supertabular}{p{5,5cm}|ll}
mi toki, tawa sina. && Ich spreche zu dir. \\
ona li lawa e jan, tawa ma pona. && Er führte die Leute in das gelobte Land. \\
ona li kama, tawa ma mi. && Er kommt in mein Land. \\
\end{supertabular} 

In den folgenden Sätzen ist jeweils das erste \textit{tawa} ein intransitives Verb.
Das jeweils zweite \textit{tawa} ist eine Präposition und leitet das Präpositional-Objekt ein. 

\begin{supertabular}{p{5,5cm}|ll}
mi tawa, tawa tomo mi. && Ich gehe zu meinem Haus. \\
ona mute li tawa, tawa utala. && Sie ziehen in den Krieg. \\
sina wile tawa, tawa telo suli. && Du willst zum Meer gehen. \\
ona li tawa, tawa sewi kiwen.  && Sie besteigt die Felsspitze. \\
\end{supertabular} 

In den folgenden Sätzen ist jeweils das erste \textit{tawa} ein transitives Verb.
Das jeweils zweite \textit{tawa} ist eine Präposition

\begin{supertabular}{p{5,5cm}|ll}
mi tawa e mi, tawa tomo mi. && Ich bewege mich zu meinem Haus. \\
mi tawa e kiwen, tawa sewi. && Ich bewege den Felsen zum Gipfel. \\
\end{supertabular} 

\index{\textit{pona}}
\index{\textit{ike}}
%
Wenn man ausdrücken möchte, dass einem etwas gefällt, verwendet man auch die Präposition \textit{tawa}.
Dies erfolgt nach dem Muster 'es ist gut zu mir' oder 'es ist schlecht zu mir'.

\begin{supertabular}{p{5,5cm}|ll}
ni li ' pona, tawa mi. && Das ist gut für mich. / Ich mag das. \\
ni li ' ike, tawa mi && Das ist schlecht für mich. / Ich mag es nicht. \\
kili li ' pona, tawa mi. && Ich mag Obst. \\
toki li ' pona, tawa mi. && Ich rede gern. / Ich mag Sprachen. \\
utala li ' ike, tawa mi. && Ich mag keinen Krieg. \\
telo suli li ' ike, tawa mi. && Ich mag das Meer nicht. \\
pipi li ' ike, tawa mi. && Ich hasse Spinnen. \\
ali li ' pona, tawa mi. && Für mich ist alles in Ordnung. \\
ma ali li ' pona, tawa mi. && Für mich sind alle Länder gut. \\
\end{supertabular} 

\index{Nebensatz}
\index{Satz!-Neben}
%
In \textit{toki pona} gibt es keine Nebensätze. 
Wenn man z.~B. sagen will 'mir gefällt es, die Landschaft anzuschauen', ist es besser den Satz aufzuteilen.

\begin{supertabular}{p{5,5cm}|ll}
mi lukin e ma. ni li ' pona, tawa mi. && Ich schaue mir die Landschaft an. Das ist gut für mich. \\
\end{supertabular} 

Natürlich kann man den Satz auch anders sagen.

\begin{supertabular}{p{5,5cm}|ll}
ma li pona lukin. && Die Landschaft ist schön anzuschauen. \\
\end{supertabular} 

%
Die Präposition \textit{tawa} kann auch 'für' bedeuten. 

\begin{supertabular}{p{5,5cm}|ll}
mi pona e tomo, tawa jan pakala. && Ich reparierte das Haus für den behinderten Mann. \\
\end{supertabular} 

\index{\textit{tawa}!Adjektiv}
%
Es ergeben sich Mehrdeutigkeiten, da \textit{tawa} auch als Adjektiv verwendet werden kann. 
\textit{tawa} wird als Adjektiv benutzt, um eine sich bewegende Konstruktion zu bezeichnen. 

\begin{supertabular}{p{5,5cm}|ll}
tomo tawa && Auto, PKW, LKW, Bus \\
tomo tawa telo && Boot, Schiff, Floß \\
tomo tawa kon && Flugzeug, Hubschrauber, Ballon, Zeppelin \\
\end{supertabular} 

Betrachten wir uns den folgenden Satz.

\begin{supertabular}{p{5,5cm}|ll}
mi pana e tomo tawa sina. && ? \\   % no-dictionary
\end{supertabular} 

Nach \textit{mi pana e tomo} ist sowohl ein Slot für ein Adjektiv als auch ein Slot für eine Präposition möglich. 
Mit dem Adjektiv \textit{tawa} bedeutet der Satz 'ich gebe dein Auto'. 
Mit der Präposition \textit{tawa} bedeutet dagegen der Satz 'ich gebe dir das Haus'. 
Man kann ein Komma vor \textit{tawa} einfügen, um nur einen Slot für eine Präposition zu erzwingen. 
Besser ist es, den Satz aufzuteilen. 

\begin{supertabular}{p{5,5cm}|ll}
mi jo e tomo tawa sina. mi pana e ni tawa sina. && Ich habe dein Auto. Ich gebe es dir. \\
ni li tomo. mi pana e ni tawa sina. && Dies ist ein Haus. Ich gebe es dir. \\
\end{supertabular} 

%
%%%%%%%%%%%%%%%%%%%%%%%%%%%%%%%%%%%%%%%%%%%%%%%%%%%%%%%%%%%%%%%%%%%%%%%%%%
\index{\textit{kama}!intransitives Verb}
%
In diesm Satz wird das intransitive Verb \textit{kama} und die Präposition \textit{tawa} verwendet.

\begin{supertabular}{p{5,5cm}|ll}
ona li kama, tawa tomo mi. && Er kam zu meinem Haus. \\
\end{supertabular} 

%
%%%%%%%%%%%%%%%%%%%%%%%%%%%%%%%%%%%%%%%%%%%%%%%%%%%%%%%%%%%%%%%%%%%%%%%%%%
\index{\textit{sama}!Präposition}
\index{\textit{sama}!Adjektiv}
%
Die Präposition \textit{sama} bedeutet 'wie' oder 'dergleichen'.

\begin{supertabular}{p{5,5cm}|ll}
ona li lukin, sama pipi. && Er guckt wie ein Käfer. \\
\end{supertabular} 

Dagegen kann ja direkt nach dem Separator \textit{li} keine Präposition folgen. 
Es gäbe dann kein Prädikat. 
Das Adjektiv \textit{sama} wird hier als Prädikatsadjektiv verwendet.

\begin{supertabular}{p{5,5cm}|ll}
jan ni li ' sama mi. && Diese Person ist mir ähnlich. \\
\end{supertabular} 

%
%%%%%%%%%%%%%%%%%%%%%%%%%%%%%%%%%%%%%%%%%%%%%%%%%%%%%%%%%%%%%%%%%%%%%%%%%%
\index{\textit{tan}!Präposition}
% 
Die Präposition \textit{tan} bedeutet 'von', 'aus', 'durch', 'wegen', 'weil' oder 'seit'.

\begin{supertabular}{p{5,5cm}|ll}
mi moku, tan ni: mi wile moku. &&  Ich esse weil ich hungrig bin. \\
\end{supertabular} 

%
%%%%%%%%%%%%%%%%%%%%%%%%%%%%%%%%%%%%%%%%%%%%%%%%%%%%%%%%%%%%%%%%%%%%%%%%%%
\subsection*{Vergleich indirekter Objekte und Präpositionalobjekte}
%
\index{Objekt!indirekt vs. Präpositional-}
\index{indirektes Objekt!vs. Präpositional-}
\index{Präpositionalobjekt!vs. indirektes Objekt}
\index{inttransitives Verb!und Hilfsverb}
\index{Hilfsverb! und inttransitives Verb}
\index{\textit{kepeken}!preposition}
\index{\textit{kepeken}!intransitive verb}
%%%%%%%%%%%%%%%%%%%%%%%%%%%%%%%%%%%%%%%%%%%%%%%%%%%%%%%%%%%%%%%%%%%%%%%%%%
%

Sowohl indirekte Objekte als auch Präpositionalobjekte werden nicht direkt vom Prädikat beeinflusst. 
Präpositionalobjekte sind daher eine Sonderform indirekter Objekte. 
Im folgenden Beispiel wird mit dem intransitiven Verb \textit{kepeken} das indirekte Objekt \textit{ilo ni} verwendet.

\begin{supertabular}{p{5,5cm}|ll}
mi pona e tomo tawa. mi kepeken ilo ni. && Ich reparier das Auto. Ich verwende dieses Werkzeug. \\
\end{supertabular} 

Man kann die Aussage kürzer und eindeutiger formulieren, wenn die Präposition \textit{kepeken} das Präpositionalobjekt \textit{ilo ni} einleitet. 

\begin{supertabular}{p{5,5cm}|ll}
mi pona e tomo tawa, kepeken ilo ni. &&  Ich repariere das Auto mit diesem Werkzeug. \\
\end{supertabular} 

Wenn man allerdings unbedingt dieses Werkzeug verwenden möchte, muss man das intransitive Verb \textit{kepeken} verwenden. 
Hilfsverben lassen sich nur mit Verben verwenden und nicht mit Präpositionen. 
Vor dem intransitiven Verb \textit{kepeken} wird hier das Hilfsverb \textit{wile} verwendet. 

\begin{supertabular}{p{5,5cm}|ll}
mi pona e tomo tawa. mi wile kepeken ilo ni. && Ich reparier das Auto. Ich möchte dieses Werkzeug verwenden. \\
\end{supertabular} 

%
\index{\textit{tawa}!preposition}
\index{\textit{tawa}!intransitive verb}
\index{\textit{tawa}!transitive verb}
%
In anderen Lektionen wird das intransitive Verb \textit{tawa} verwendet.

\begin{supertabular}{p{5,5cm}|ll}
mi tawa sina. && Ich gehe zu dir. \\ % no-dictionary
\end{supertabular} 

Dieser Satz ist mehrdeutig. 
Nach \textit{mi} ist hier sowohl ein Subtantiv- (Prädikatsnomen) als auch ein Adjektiv-Slot (Prädikatsadjektiv) möglich. 

\begin{supertabular}{p{5,5cm}|ll}
mi tawa sina. &&  Ich bin deine Bewegung. \\ % no-dictionary
\end{supertabular} 

Besser ist die Verwendung eines Präpositionalobjektes.
Wird, wie in diesen Lektionen empfohlen, ein Komma vor der Präposition gesetzt, wird der Satz eindeutiger.

\begin{supertabular}{p{5,5cm}|ll}
mi tawa, tawa sina. && Ich gehe zu dir. \\ 
\end{supertabular} 

Bei genauer Betrachtung fällt auf, dass \textit{tawa} hier gar kein intransitives Verb ist.
Man kann den Satz auch mit dem Reflexivpronomen \textit{mi} als direktes Objekt formulieren. 
Das erste \textit{tawa} ist hier ein transitives Verb. 
Das zweite \textit{tawa} ist eine Präposition. 

\begin{supertabular}{p{5,5cm}|ll}
mi tawa e mi, tawa sina. && Ich bewege mich zu dir. \\ % no-dictionary
\end{supertabular} 

%
%%%%%%%%%%%%%%%%%%%%%%%%%%%%%%%%%%%%%%%%%%%%%%%%%%%%%%%%%%%%%%%%%%%%%%%%%%
\newpage
%
\subsection*{Übungen (Antworten siehe Seite~\pageref{'prepositional_objects'})}
%%%%%%%%%%%%%%%%%%%%%%%%%%%%%%%%%%%%%%%%%%%%%%%%%%%%%%%%%%%%%%%%%%%%%%%%%%
%
Schreibe bitte die Antworten auf einen Zettel und überprüfe sie anschließend. 

\begin{supertabular}{p{5,5cm}|ll}
Mit was ist eine Präposition eng verbunden? &&   \\ % no-dictionary
Zu welcher Phrase im Satz gehört das Präpositionalobjekt? &&    \\ % no-dictionary
Wo befinden sich Slots für Präpositionen? &&   \\ % no-dictionary
An welcher Position im Satz kann ein Präpositionalobjekt stehen? &&  \\ % no-dictionary
Mit welchen Separatoren kann man zusammengesetzte Sätze bilden? &&   \\ % no-dictionary
Welche Slots sind an zweiter Stelle im Präpositionalobjekt möglich? &&   \\ % no-dictionary
\end{supertabular}

Versuche diese Sätze zu übersetzen. 
Mit dem Tool \textit{Toki Pona Parser} (\cite{www:rowa:02}) kann man Rechtschreibung und Grammatik überprüfen. 

\begin{supertabular}{p{5,5cm}|ll}
Ich reparierte die Taschenlampe mit einem kleinen Werkzeug.  &&  \\ % no-dictionary
Ich mag \textit{toki pona}.  &&  \\ % no-dictionary
Wir gaben ihnen Essen.  &&  \\ % no-dictionary
Ich möchte mit meinem Auto zu seinem Haus fahren.  &&  \\ % no-dictionary
Leute sehen wie Ameisen aus. && \\ % no-dictionary
\end{supertabular}

\begin{supertabular}{p{5,5cm}|ll}
sina wile kama, tawa tomo toki.  &&  \\ % no-dictionary
jan li toki, kepeken toki pona, lon tomo toki.  &&  \\ % no-dictionary
mi tawa, tawa tomo toki. ona li pona, tawa mi.  &&  \\ % no-dictionary
sina kama jo e jan pona, lon ni.  &&  \\ % no-dictionary
sama li ' ike. &&  \\ % no-dictionary
mi sona e tan. &&  \\ % no-dictionary
\end{supertabular} 

%
%%%%%%%%%%%%%%%%%%%%%%%%%%%%%%%%%%%%%%%%%%%%%%%%%%%%%%%%%%%%%%%%%%%%%%%%%%
% eof
