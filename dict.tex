%%%%%%%%%%%%%%%%%%%%%%%%%%%%%%%%%%%%%%%%%%%%%%%%%%%%%%%%%%%%%%%%%%%%%%%%%%
\section{toki-pona -- Deutsch Wörterbuch}
\label{'dict'}
%%%%%%%%%%%%%%%%%%%%%%%%%%%%%%%%%%%%%%%%%%%%%%%%%%%%%%%%%%%%%%%%%%%%%%%%%%
%
\begin{supertabular}{p{2,5cm}|ll}
\textbf{.} && \textit{Separator}: Beendet wird ein Aussagesatz ((Deklarativsatz)) mit einem Punkt. \\ && Verwende einen Punkt nicht zusammmen mit einem der anderen Separatoren. \\ 
 && \\ % no-dictionary
\textbf{!} && \textit{Separator}: Aufforderungssätze (Imperativ) und \\ &&  Ausrufesätze (Interjektionen, Exclamatory) enden mit einem Ausrufungszeichen. \\ && Verwende ein Ausrufungszeichen nicht zusammmen mit einem der anderen Separatoren. \\ 
 && \\ % no-dictionary
\textbf{?} && \textit{Separator}: Beendet wird eine Frage (Interrogativ) mit einem Fragezeichen. \\ && Verwende ein Fragezeichen nicht zusammmen mit einem der anderen Separatoren. \\ 
 && \\ % no-dictionary
\textbf{:} && \textit{Separator}: Ein Doppelpunkt trennt ein Hinweis-Satz von einem Satz. \\  && Jeweils vor und nach dem Doppelpunkt muss ein vollständiger Satz sein. \\  && Verwende ein Doppelpunkt nicht zusammmen mit einem der anderen Separatoren. \\ 
 && \\ % no-dictionary
\textbf{,} && \textit{Separator}: Ein Komma wird nach \glqq o\grqq verwendet wenn man Leute anspricht. \\ && Optional kann es vor einer Präposition eingefügt werden. \\ && Verwende ein Komma nicht zusammmen mit einem der anderen Separatoren ausser \glqq li\grqq. \\ 
 && \\ % no-dictionary
\textbf{"} && \textit{Separator}: Anführungszeichen werden für Zitate oder Originalschreibweise verwendet. \\ 
 && \\ % no-dictionary
\textbf{'} && \textit{inoffiziell}: Ein leerer Verb-Slot wird hier mit einem Apostroph gekennzeichnet. \\ 
 && \\ % no-dictionary
\textbf{\#} && \textit{inoffiziell}: Zur Kennzeichnung von Zahlen \\ 
 && \\ % no-dictionary
\textbf{a} && \textit{Interjektion}: ah, ha, uh, oh, ooh, aw, well  (emotionales Wort) \\ 
\textbf{a a a!} && \textit{Interjektion}: Lachen \\ 
 && \\ % no-dictionary
\textbf{\dots akesi} && \textit{Adjektiv}: Amphibien-, Reptilien-, schleimig \\ 
\textbf{akesi} && \textit{Substantiv}: Reptil, Lurch, Amphibie, Saurier, Schlange, Drache, Frosch, Schildkröte \\ 
 && \\ % no-dictionary
\textbf{\dots ala} && \textit{Adjektiv}: nein, nichts, un- \\ 
\textbf{\dots ala} && \textit{Adjektiv, Ziffer}: Null, 0 \\ 
\textbf{\dots ala} && \textit{Adverb}: nicht, un- \\ 
\textbf{ala!} && \textit{Interjektion}: Nein! \\ 
\textbf{ala} && \textit{Substantiv}: Nichts, Negation, Null \\ 
 && \\ % no-dictionary
\textbf{\dots alasa} && \textit{Adjektiv}: jagend \\ 
\textbf{alasa} && \textit{Substantiv}: Jagd \\ 
\textbf{alasa (e \dots)} && \textit{Verb, transitiv}: jagen, verfolgen und töten, nach Futter suchen \\ 
 && \\ % no-dictionary
\textbf{\dots ale} && \textit{Adjektiv}: alle(r/s), je, jeweils, komplett, ganz, (ale = ali), (veraltet) \\ 
\textbf{\dots ale} && \textit{Adjektiv, Ziffer}: 100 (offizielles Toki Pona Buch) \\ 
\textbf{\dots ale} && \textit{Adverb}: alle(r/s), je, jeweils, komplett, ganz, (ale = ali), (veraltet) \\ 
\textbf{ale} && \textit{Substantiv}:  Alles, Etwas, Leben, Universium, Welt (veraltet) \\ 
 && \\ % no-dictionary
\textbf{\dots ali} && \textit{Adjektiv}: alle(r/s), je, jeweils, komplett, ganz \\ 
\textbf{\dots ali} && \textit{Adverb}: alle(r/s), je, jeweils, komplett, ganz \\ 
\textbf{ali} && \textit{Substantiv}: Alles, Etwas, Leben, Universium, Welt \\ 
 && \\ % no-dictionary
\textbf{\dots anpa} && \textit{Adjektiv}: unten, tief, niedrig \\ 
\textbf{\dots anpa} && \textit{Adverb}: tief, niedrig \\ 
\textbf{anpa} && \textit{Substantiv}: Boden, Erdboden, Grund, Talsohle \\ 
\textbf{anpa} && \textit{Verb, intransitiv}:  sich unterwerfen \\
\textbf{anpa (e \dots)} && \textit{Verb, transitiv}: niederlassen, absenken, abseilen, herunterholen, besiegen, bezwingen \\ 
 && \\ % no-dictionary
\textbf{\dots ante} && \textit{Adjektiv}: unterschiedlich, verschieden (von) \\ 
\textbf{ante} && \textit{Substantiv}: Unterschied, Abweichung, Differenz \\ 
\textbf{ante la \dots} && \textit{Substantiv}: Wenn Unterschied, bei Abweichung, bei Differenz  \\ 
\textbf{ante (e \dots)} && \textit{Verb, transitiv}: wechseln; umschalten, ändern, abändern, verändern, modifizieren \\ 
 && \\ % no-dictionary
\textbf{\dots anu \dots} && \textit{Konjunktion}: oder  \\ 
 && \\ % no-dictionary
\textbf{\dots awen} && \textit{Adjektiv}: bleibend; verbleibend, feststehend, ortsfest, parkend, permanent, sesshaft \\ 
\textbf{\dots awen} && \textit{Adverb}: verbleibend, noch, gerade beim ...  \\ 
\textbf{awen} && \textit{Substantiv}: Trägheit, Kontinuität, Kontinuum, Aufenthalt \\ 
\textbf{awen} && \textit{Verb, intransitiv}: bleiben, verbleiben, sich aufhalten, warten \\ 
\textbf{awen (e \dots)} && \textit{Verb, transitiv}: behalten; nicht weggeben \\ 
 && \\ % no-dictionary
\textbf{\dots e \dots} && \textit{Separator}: Ein \glqq e\grqq leitet das direkte Objekt ein. \\  && Verwende ein \glqq e\grqq nicht zusammmen mit einem der anderen Separatoren. \\ 
 && \\ % no-dictionary
\textbf{\dots en \dots} && \textit{Konjunktion}:  und (um Haupt-Substantive zu verbinden) \\ 
 && \\ % no-dictionary
\textbf{\dots esun} && \textit{Adjektiv}: kommerziell, gewerblich, kaufmännisch, verkäuflich \\ 
\textbf{esun} && \textit{Substantiv}: Markt, Shop \\ 
\textbf{esun (e \dots)} && \textit{Verb, transitiv}: kaufen, verkaufen, tauschen \\ 
 && \\ % no-dictionary
\textbf{\dots ijo} && \textit{Adjektiv}: für etwas \\ 
\textbf{\dots ijo} && \textit{Adverb}: für etwas \\ 
\textbf{ijo} && \textit{Substantiv}: Ding, Sache, Zeug, Krempel, Gegenstand, Objekt \\ 
\textbf{ijo (e \dots)} && \textit{Verb, transitiv}: objektivieren \\ 
 && \\ % no-dictionary
\textbf{\dots ike} && \textit{Adjektiv}: schlecht, schlimm, übel, böse, ungezogen, falsch, zu komplex, ungesund \\ 
\textbf{\dots ike} && \textit{Adverb}: schlecht, schlimm, übel, böse, ungezogen, falsch, zu komplex, ungesund \\ 
\textbf{ike!} && \textit{Interjektion}: Mist! Leider! Sch \dots ! \\ 
\textbf{ike} && \textit{Substantiv}: Negativität, Übel, Böse, Schlechtigkeit, Verderbtheit \\ 
\textbf{ike la \dots} && \textit{Substantiv}: wenn Übel ..., bei Verderbtheit, bei Boshaftigkeit \\ 
\textbf{ike} && \textit{Verb, intransitiv}: schlech sein, jdm. in den Arsch kriechen \\ 
\textbf{ike (e \dots)} && \textit{Verb, transitiv}: schlecht machen, verschlechtern, schlechten Einfluss auf \\ 
 && \\ % no-dictionary
\textbf{\dots ilo} && \textit{Adjektiv}: nützlich \\ 
\textbf{\dots ilo} && \textit{Adverb}: nutzbringend \\ 
\textbf{ilo} && \textit{Substantiv}: Werkzeug, Gerät, Tool, Vorrichtung, Apparat, Maschine \\ 
 && \\ % no-dictionary
\textbf{\dots insa} && \textit{Adjektiv}: intern, zentral \\ 
\textbf{insa} && \textit{Substantiv}: Innere, Innenseite, Zentrum, Magen \\ 
 && \\ % no-dictionary
\textbf{\dots jaki} && \textit{Adjektiv}: schmutzig, dreckig, schmierig, unsauber, schmuddelig, obszön \\ 
\textbf{\dots jaki} && \textit{Adverb}: schmutzig, dreckig, schmierig; unsauber; schmuddelig; versifft, grob\\ 
\textbf{jaki!} && \textit{Interjektion}: Igit! Eckelhaft! \\ 
\textbf{jaki} && \textit{Substantiv}: Dreck, Schmutz, Verschmutzung, Müll, Schrott, Kot \\ 
\textbf{jaki (e \dots)} && \textit{Verb, transitiv}: verschmutzen, verunreinigen, verpesten \\ 
 && \\ % no-dictionary
\textbf{\dots jan} && \textit{Adjektiv}: persönlich, human, jemandens \\ 
\textbf{\dots jan} && \textit{Adverb}: persönlich, human, jemandens \\ 
\textbf{jan} && \textit{Substantiv}: Mensch, Person, Leute, Jemand \\ 
\textbf{jan (e \dots)} && \textit{Verb, transitiv}: personifizieren, verkörpern, vermenschlichen, personalisieren \\ 
 && \\ % no-dictionary
\textbf{\dots jelo} && \textit{Adjektiv}: gelb, hellgrün \\ 
\textbf{jelo} && \textit{Substantiv}: Gelb, Hellgrün \\ 
 && \\ % no-dictionary
\textbf{\dots jo} && \textit{Adjektiv}: privat, persöhnlich \\ 
\textbf{jo} && \textit{Substantiv}: Besitz, Inhalt \\ 
\textbf{jo (e \dots)} && \textit{Verb, transitiv}: haben, enthalten, umfassen \\ 
kama \textbf{jo (e \dots)} && \textit{Verb, transitiv}: erhalten, bekommen, kriegen, abkriegen, nehmen \\ 
 && \\ % no-dictionary
\textbf{\dots kala} && \textit{Adjektiv}: fischig \\ 
\textbf{kala} && \textit{Substantiv}: Wasserlebewesen, Fisch, Qualle, Tintenfisch, Muschel \\ 
 && \\ % no-dictionary
\textbf{\dots kalama} && \textit{Adjektiv}: laut, geräuschvoll, lautstark \\
\textbf{kalama} && \textit{Substantiv}: Laut, Sound, Geräusch, Stimme \\ 
\textbf{kalama} && \textit{Verb, intransitiv}: Geräusch machen, laut sein \\ 
\textbf{kalama (e \dots)} && \textit{Verb, transitiv}: klingen, läuten, klingeln, ein Instrument spielen \\ 
 && \\ % no-dictionary
\textbf{\dots kama} && \textit{Adjektiv}: einkehrend, kommend, zukünftig \\ 
\textbf{\dots kama} && \textit{Adverb}: einkehrend, kommend, zukünftig \\ 
\textbf{kama} && \textit{Substantiv}:  Zukunft, Chance, Ankunft, Anreise, Anfang, Beginn \\ 
\textbf{kama} && \textit{Verb, intransitiv}: kommen, werden, anfangen, eintreffen, ankommen,  \\ 
\textbf{kama \dots} && \textit{Hilfsverb}: werde(n), er- \\ 
\textbf{kama (e \dots)} && \textit{Verb, transitiv}: bewerkstelligen, herbeiführen \\ 
 && \\ % no-dictionary
\textbf{\dots kasi} && \textit{Adjektiv}: vegetarisch, pflanzlich, biologisch \\ 
\textbf{kasi} && \textit{Substantiv}:  Planze, Blatt, Kraut, Baum, Strauch, Gras, Pilz \\ 
\textbf{kasi} && \textit{Verb, intransitiv}: (selber) wachsen \\ 
\textbf{kasi (e \dots)} && \textit{Verb, transitiv}: (ein-)pflanzen  \\ 
 && \\ % no-dictionary
\textbf{ken} && \textit{Substantiv}: Möglichkeit, Fähigkeit, Begabung, Erlaubnis, Genehmigung, Freigabe \\ 
\textbf{ken la \dots)} && \textit{Substantiv}: es besteht die Möglichkeit, bei Erlaubnis, bei Genehmigung \\ 
\textbf{ken} && \textit{Verb, intransitiv}: kann, ist fähig zu, ist bemächtigt zu, mögen; dürfen, ist möglich \\ 
\textbf{ken \dots} && \textit{Hilfsverb}: kann ... \\ 
\textbf{ken (e \dots)} && \textit{Verb, transitiv}:  möglich machen, anordnen, aktivieren, freigeben, erlauben \\ 
 && \\ % no-dictionary
\textbf{kepeken} && \textit{Substantiv}: Nutzung, Verwendung, Werkzeug \\ 
\textbf{\dots kepeken \dots} && \textit{Präposition}: mit, mittels, per \\ 
\textbf{kepeken} && \textit{Verb, intransitiv}: benutzen, verwenden, anwenden, nutzen, gebrauchen \\ 
\textbf{kepeken \dots} && \textit{Hilfsverb}: verwenden \\ 
 && \\ % no-dictionary
\textbf{\dots kili} && \textit{Adjektiv}: fruchtig \\ 
\textbf{\dots kili} && \textit{Adverb}: fruchtig \\ 
\textbf{kili} && \textit{Substantiv}: Frucht, Obst, Gemüse, Pilz \\ 
 && \\ % no-dictionary
\textbf{\dots kin} && \textit{Adjektiv}: auch, noch, ebenso, ferner, außerdem, gerade, wirklich, tatsächlich \\ && kin kann das letzte Wort in einer Adjectiv-Gruppe sein. \\
\textbf{\dots kin} && \textit{Adverb}: auch, noch, ebenso, ferner, außerdem, gerade, wirklich, tatsächlich \\ && kin kann das letzte Wort in einer Adverb-Gruppe sein. \\
\textbf{kin} && \textit{Substantiv}: Wirklichkeit, Tatsache, Fakt  \\ 
\textbf{kin la \dots} && \textit{Substantiv}: wenn Wirklichkeit, wenn Tatsache \\ 
%\textbf{kin!} && \textit{Interjektion}: Wirklich! \\ 
 && \\ % no-dictionary
\textbf{kipisi } && \textit{Substantiv}: Abschnitt, Fragment, Scheibe \\ 
\textbf{kipisi (e \dots)} && \textit{Verb, transitiv}: schneiden \\ 
 && \\ % no-dictionary
\textbf{\dots kiwen} && \textit{Adjektiv}: hart, schwer, heftig, fest, solid, stabil, robust, steinhart \\ 
\textbf{\dots kiwen} && \textit{Adverb}: hart, schwer, heftig, fest, solid, stabil, robust, steinhart \\ 
\textbf{kiwen} && \textit{Substantiv}: Felsbrocken, Fels, Stein, Metall, Mineral, Lehm \\ 
\textbf{kiwen (e \dots)} && \textit{Verb, transitiv}: verfestigen, verhärten, versteinern \\ 
 && \\ % no-dictionary
\textbf{ko} && \textit{Substantiv}: halbfeste Substanz, Paste, Puder, Gummi, Kleber, Mehl, Brei, Pulver \\ 
\textbf{ko (e \dots)} && \textit{Verb, transitiv}: zerquetschen, pulverisieren \\ 
 && \\ % no-dictionary
\textbf{\dots kon} && \textit{Adjektiv}: luftig, ätherisch, gasförmig, gasartig \\ 
\textbf{\dots kon} && \textit{Adverb}: luftig, ätherisch, gasförmig, gasartig \\ 
\textbf{kon} && \textit{Substantiv}: Luft, Atmosphäre, Wind, Odor, Geruch, Seele, Geist \\ 
\textbf{kon} && \textit{Verb, intransitiv}: atmen \\ 
\textbf{kon (e \dots)} && \textit{Verb, transitiv}: wegblasen \\ 
 && \\ % no-dictionary
\textbf{\dots kule} && \textit{Adjektiv}: farbenfreudig \\ 
\textbf{kule} && \textit{Substantiv}: Farbe, Lack \\ 
\textbf{kule (e \dots)} && \textit{Verb, transitiv}: färben, anstreichen, streichen \\ 
 && \\ % no-dictionary
\textbf{\dots kulupu} && \textit{Adjektiv}: kommunal, geteilt, öffentlich \\ 
\textbf{kulupu} && \textit{Substantiv}: Gruppe, Gemeinde, Gemeinschaft, Gemeinsamkeit, Gesellschaft \\ 
\textbf{kulupu (e \dots)} && \textit{Verb, transitiv}: montieren, zusammenbauen, zusammenrufen \\ 
 && \\ % no-dictionary
\textbf{\dots kute} && \textit{Adjektiv}: auditiv, feinhörig \\ 
\textbf{kute} && \textit{Substantiv}: Gehör, Hörsinn \\ 
\textbf{kute (e \dots)} && \textit{Verb, transitiv}: hören, zuhören \\ 
 && \\ % no-dictionary
\textbf{\dots la \dots} && \textit{Separator}: Ein \glqq la\grqq  trennt eine Konditional-Phrase von einem Satz. \\  && Verwende ein \glqq la\grqq nicht zusammmen mit einem der anderen Separatoren. \\ 
 && \\ % no-dictionary
\textbf{\dots lape} && \textit{Adjektiv}: schlafend \\ 
\textbf{\dots lape} && \textit{Adverb}: schlafend \\ 
\textbf{lape} && \textit{Substantiv}: Schlaf, Ruhe, Pause \\ 
\textbf{lape} && \textit{Verb, intransitiv}: schlafen, ruhen, dösen, liegen \\ 
\textbf{lape (e \dots)} && \textit{Verb, transitiv}: bewusstlos schlagen \\ 
 && \\ % no-dictionary
\textbf{\dots laso} && \textit{Adjektiv}: blau, cyan \\ 
\textbf{laso} && \textit{Substantiv}: Blau, Cyan \\ 
 && \\ % no-dictionary
\textbf{\dots lawa} && \textit{Adjektiv}: Haupt- , größt- , wichtigst-, Leit-, tonangebend \\ 
\textbf{\dots lawa} && \textit{Adverb}: führend,  \\ 
\textbf{lawa} && \textit{Substantiv}: Kopf, Sinn, Meinung, Ansicht \\ 
\textbf{lawa (e \dots)} && \textit{Verb, transitiv}: führen, lenken, kontrollieren, steuern, herrschen \\ 
 && \\ % no-dictionary
\textbf{\dots len} && \textit{Adjektiv}: gekleidet, kostümiert, verkleidet \\ 
\textbf{len} && \textit{Substantiv}: Kleidung, Bekleidung, Stoff, Tuch, Lappen, Gewebe, Netzwerk, Internet \\ 
\textbf{len (e \dots)} && \textit{Verb, transitiv}: (Kleidung) tragen, verkleiden \\ 
 && \\ % no-dictionary
\textbf{\dots lete} && \textit{Adjektiv}: kalt, frostig, roh, ungekocht \\ 
\textbf{\dots lete} && \textit{Adverb}: trostlos \\ 
\textbf{lete} && \textit{Substantiv}: Kälte \\ 
\textbf{lete (e \dots)} && \textit{Verb, transitiv}: erkalten, kühlen, gefrieren, vereisen \\ 
 && \\ % no-dictionary
\textbf{\dots li \dots} && \textit{Separator}: Er trennt die Subjekt-Phrase, außer \glqq mi\grqq und \glqq sina\grqqvon, von der Prädikat-Phrase. \\ && Verwende ein \glqq li\grqq nicht zusammmen mit einem der anderen Separatoren. \\  
 && \\ % no-dictionary
\textbf{\dots lili} && \textit{Adjektiv}: klein, gering, unbedeutend, wenig, kaum, schwerlich, jung, ein wenig, kurz \\ 
\textbf{\dots lili} && \textit{Adverb}: wenig \\ 
\textbf{lili} && \textit{Substantiv}: Kleinheit, Jugend, Unmündigkeit \\ 
\textbf{lili (e \dots)} && \textit{Verb, transitiv}: verkleinern, verringern, ermäßigen, schrumpfen\\ 
 && \\ % no-dictionary
\textbf{\dots linja} && \textit{Adjektiv}: längliche, rechteckig, lang \\ 
\textbf{linja} && \textit{Substantiv}: Faden, Kette, Saite, Schnur, Sehne, Leine, Haar, Zwirn \\ 
 && \\ % no-dictionary
\textbf{\dots lipu} && \textit{Adjektiv}: Buch-, Papier-, Karten-, Ticket-, Bogen-, Seiten- \\ 
\textbf{lipu} && \textit{Substantiv}: Schriftstück, z.B. Papier, Pappe, Karte, Ticket, Website, Blog, Liste \\ 
 && \\ % no-dictionary
\textbf{\dots loje} && \textit{Adjektiv}: rot, rötlich \\ 
\textbf{loje} && \textit{Substantiv}: Rot \\ 
 && \\ % no-dictionary
\textbf{\dots lon} && \textit{Adjektiv}: real, wahr, bestehend, richtig, echt \\ 
\textbf{lon} && \textit{Substantiv}: Existenz, Sein, Präsenz \\ 
\textbf{\dots lon \dots} && \textit{Präposition}: in, im, an, am, bei, auf \\ 
\textbf{lon} && \textit{Verb, intransitiv}: anwesend sein, vorhanden sein, enthalten sein, existieren, leben \\ 
 && \\ % no-dictionary
\textbf{\dots luka} && \textit{Adjektiv}: greifbar, Tast- \\ 
\textbf{\dots luka} && \textit{Adjektiv, Ziffer}: fünf, 5 \\ 
\textbf{luka} && \textit{Substantiv}: Hand, Arm \\ 
 && \\ % no-dictionary
\textbf{\dots lukin} && \textit{Adjektiv}: optisch, sichtbar, visuell \\ 
\textbf{\dots lukin} && \textit{Adverb}: optisch, sichtbar, visuell \\ 
\textbf{lukin} && \textit{Substantiv}: Sicht, Blick, Sehen, Vision \\ 
\textbf{lukin} && \textit{Verb, intransitiv}: aussehen, aufpassen (auf), Ausschau halten (nach), achten auf \\ 
\textbf{lukin (e \dots)} && \textit{Verb, transitiv}: sehen, betrachten, lesen \\ 
\textbf{lukin \dots} && \textit{Hilfsverb}: sehen \\ 
 && \\ % no-dictionary
\textbf{\dots lupa} && \textit{Adjektiv}: löchrig \\
\textbf{lupa} && \textit{Substantiv}: Loch, Höhle; Bau, Mündung, Öffnung, Ausguss, Fenster, Tür, Tor \\ 
\textbf{lupa (e \dots)} && \textit{Verb, transitiv}: durchbohren, erstechen, perforieren \\ 
 && \\ % no-dictionary
\textbf{\dots ma} && \textit{Adjektiv}: ländlich, im Freien, Open-Air- \\ 
\textbf{ma} && \textit{Substantiv}: Land, Gegend, Erde, Staat, Kontinent, Gebiet, Zone, Areal \\ 
 && \\ % no-dictionary
\textbf{\dots mama} && \textit{Adjektiv}: elterlich, mütterlich, väterlich \\ 
\textbf{mama} && \textit{Substantiv}: Eltern, Mutter, Vater \\ 
\textbf{mama (e \dots)} && \textit{Verb, transitiv}: bemuttern \\ 
 && \\ % no-dictionary
\textbf{\dots mani} && \textit{Adjektiv}: Finanz-, finanziell, währungs-, Vermögens- \\ 
\textbf{\dots mani} && \textit{Adverb}: finanziell \\ 
\textbf{mani} && \textit{Substantiv}: Geld, Währung, Euro, Kapital \\ 
 && \\ % no-dictionary
\textbf{\dots meli} && \textit{Adjektiv}: weiblich, feminin, fraulich \\ 
\textbf{meli} && \textit{Substantiv}: Frau, Dame, Lady, Mädchen, Weibchen, Weib, Ehefrau, Freundin \\ 
 && \\ % no-dictionary
\textbf{mi} && \textit{Personalpronomen}: ich, wir  \\ 
\textbf{\dots mi} && \textit{Possessivpronomen}: mein, unser \\  
\textbf{\dots e mi} && \textit{Reflexivpronomen}: mich, uns  \\ 
 && \\ % no-dictionary
\textbf{\dots mije} && \textit{Adjektiv}: männlich, maskulin \\ 
\textbf{mije} && \textit{Substantiv}: Mann, Kerl, Eheman, Liebhaber \\ 
 && \\ % no-dictionary
\textbf{\dots moku} && \textit{Adjektiv}: essend \\ 
\textbf{\dots moku} && \textit{Adverb}: essend \\ 
\textbf{moku} && \textit{Substantiv}: Essen, Lebensmittel, Nahrung, Nahrungsmittel, Mahlzeit, Fressen \\ 
\textbf{moku (e \dots)} && \textit{Verb, transitiv}: essen, trinken, schlucken, verzehren, verspeisen \\ 
 && \\ % no-dictionary
\textbf{\dots moli} && \textit{Adjektiv}: tot, tötlich, fatal, schwerwiegend \\ 
\textbf{\dots moli} && \textit{Adverb}: tödlich \\ 
\textbf{moli} && \textit{Substantiv}: Tod, Todesfall \\ 
\textbf{moli} && \textit{Verb, intransitiv}: tot sein \\ 
\textbf{moli (e \dots)} && \textit{Verb, transitiv}: töten \\ 
kama \textbf{moli} && \textit{Verb, intransitiv}: sterben \\ 
 && \\ % no-dictionary
\textbf{\dots monsi} && \textit{Adjektiv}: Rück-, Hinter- \\ 
\textbf{monsi} && \textit{Substantiv}: Rücken, Heck, Hintern, Po, Arsch \\ 
 && \\ % no-dictionary
\textbf{\dots monsuta} && \textit{Adjektiv}: ängstlich \\ 
\textbf{monsuta} && \textit{Substantiv}: Monster, Ungeheuer, Fabelwesen, Angst \\ 
 && \\ % no-dictionary
\textbf{\dots mu} && \textit{Adjektiv}: Tiergeräusch- \\ 
\textbf{\dots mu} && \textit{Adverb}: Tiergeräusch- \\ 
\textbf{mu!} && \textit{Interjektion}: Wau!, Miau!, Muh! (Tiergeräusch) \\ 
\textbf{mu} && \textit{Substantiv}: Tiergeräusch \\ 
\textbf{mu} && \textit{Verb, intransitiv}: tierisch kommunizieren \\ 
\textbf{mu (e \dots)} && \textit{Verb, transitiv}: Tiergeräusche machen \\ 
 && \\ % no-dictionary
\textbf{\dots mun} && \textit{Adjektiv}: Mond- \\ 
\textbf{mun} && \textit{Substantiv}: Mond, Stern, Planet \\ 
 && \\ % no-dictionary
\textbf{\dots musi} && \textit{Adjektiv}: kunstvoll, spaßig, Erholung-, Freizeit- \\ 
\textbf{\dots musi} && \textit{Adverb}: fröhlich \\ 
\textbf{musi} && \textit{Substantiv}: Spaß, Partie, Spiel, Freizeit, Erholung, Kunst, Unterhaltung \\ 
\textbf{musi} && \textit{Verb, intransitiv}: spielen, Spaß haben \\ 
\textbf{musi (e \dots)} && \textit{Verb, transitiv}: amüsieren, unterhalten, belustigen, bewirten \\ 
 && \\ % no-dictionary
\textbf{\dots mute} && \textit{Adjektiv}: viel, viele, mehrere, sehr, reichlich, zahlreich, weiter, \\ 
\textbf{\dots mute} && \textit{Adjektiv, Ziffer}: 20 (offizielles Toki Pona Buch) \\ 
\textbf{\dots mute} && \textit{Adverb}: viel, sehr, reichlich, zahlreich, mehr, weiter, \\ 
\textbf{mute} && \textit{Substantiv}: Betrag, Summe, Menge, Quantität \\ 
\textbf{mute (e \dots)} && \textit{Verb, transitiv}: viel machen, vermehren \\ 
 && \\ % no-dictionary
\textbf{\dots namako} && \textit{Adjektiv}: würzig, pikant \\ 
\textbf{namako} && \textit{Substantiv}: Gewürz, Salz, Nahrungsergänzung, Kräuter \\ 
\textbf{namako (e \dots)} && \textit{Verb, transitiv}: würzen \\ 
 && \\ % no-dictionary
\textbf{nanpa \dots} && \textit{Adjektiv, Ziffer}: Zum Bilden von Ordnungszahlen. \\ 
\textbf{nanpa} && \textit{Substantiv}: Zahl, Ziffer, Anzahl, Ordnungszahl \\ 
\textbf{nanpa (e \dots)} && \textit{Verb, transitiv}: zählen, nummerieren, beziffern, rechnen \\ 
 && \\ % no-dictionary
\textbf{\dots nasa} && \textit{Adjektiv}: dumm, doof, blöd, töricht, albern, verrückt, übergeschnappt \\ 
\textbf{\dots nasa} && \textit{Adverb}: töricht, affig, albern, verrückt, übergeschnappt \\ 
\textbf{nasa} && \textit{Substantiv}: Dummheit, Unsinn, Blödsinn, Stumpfheit, Schussel, Wirrkopf \\ 
\textbf{nasa (e \dots)} && \textit{Verb, transitiv}: verrückt machen \\ 
 && \\ % no-dictionary
\textbf{\dots nasin} && \textit{Adjektiv}: systematisch, gewöhnlich, üblich \\ 
\textbf{nasin} && \textit{Substantiv}: Weg, Straße, Strecke, Pfad, Methode, Doktrin, Lehre, System \\ 
 && \\ % no-dictionary
\textbf{\dots nena} && \textit{Adjektiv}: hügelig, hügelige, bergig, bucklige, bucklige, holprige \\ 
\textbf{nena} && \textit{Substantiv}: Beule, Hügel, Berg, Gebirge, Erhebung, Button, Nase, Rundung \\ 
 && \\ % no-dictionary
\textbf{\dots ni} && \textit{adjektivisches Demonstrativpronomen}: dies, diese, dieser, dieses (hier), (das) dort, jenes \\  
\textbf{ni} && \textit{substantivisches Demonstrativpronomen}: dies, diese, dieser, dieses (hier), (das) dort, jenes \\ 
 && \\ % no-dictionary
\textbf{nimi} && \textit{Substantiv}: Wort, Name \\ 
\textbf{nimi (e \dots) } && \textit{Verb, transitiv}: bennenen, Namen geben \\ 
 && \\ % no-dictionary
\textbf{\dots noka} && \textit{Adjektiv}: Fuss-, niedriger, unten \\ 
\textbf{\dots noka } && \textit{adverb}: zu Fuss \\ 
\textbf{noka} && \textit{Substantiv}: Bein, Fuß \\ 
 && \\ % no-dictionary
\textbf{o!} && \textit{Interjektion}: Hallo! (um jemandens Aufmerksamkeit zu erwirken) \\ 
\textbf{\dots o, \dots} && \textit{Interjektion}: Leute ansprechen \\ 
\textbf{o \dots !} && \textit{subject}: Ein \glqq o \grqq wird bei Befehlen verwendet und ersetzt das Subjekt. \\ 
\textbf{\dots o \dots !} && \textit{Separator}: Ein \glqq o \grqq wird bei Befehlen verwendet und ersetzt \glqq li \grqq. \\  
 && \\ % no-dictionary
\textbf{\dots oko} && \textit{Adjektiv}: optisch, Auge- \\ 
\textbf{oko} && \textit{Substantiv}: Auge \\ 
 && \\ % no-dictionary
\textbf{\dots olin} && \textit{Adjektiv}: Liebes- \\ 
\textbf{olin} && \textit{Substantiv}: Liebe \\ 
\textbf{olin (e \dots)} && \textit{Verb, transitiv}: (eine Person) lieben \\ 
 && \\ % no-dictionary
\textbf{ona} && \textit{Personalpronomen}: er, sie, es, sie (Plural)  \\ 
\textbf{\dots ona} && \textit{Possessivpronomen}: seins, ihres, seine, deren \\  
\textbf{\dots e ona} && \textit{Reflexivpronomen}: seins, ihres, seine, deren \\  
 && \\ % no-dictionary
\textbf{\dots open} && \textit{Adjektiv}: (er-)öffnend \\ 
\textbf{open} && \textit{Substantiv}: Anfang, Beginn, Start  \\ 
\textbf{open la \dots} && \textit{Substantiv}: zu Anfang, bei Beginn, beim Start \\ 
\textbf{open (e \dots)} && \textit{Verb, transitiv}: öffnen, andrehen, anmachen, anschalten, beginnen, anfangen \\ 
\textbf{open \dots } && \textit{Hilfsverb}: beginnen, starten \\ 
 && \\ % no-dictionary\dots
\textbf{\dots pakala} && \textit{Adjektiv}: ruiniert, zerstört, zerbrochen \\ 
\textbf{\dots pakala} && \textit{Adverb}: ruiniert, zerstört, zerbrochen \\ 
\textbf{pakala!} && \textit{Interjektion}: Verdammt! \\ 
\textbf{pakala} && \textit{Substantiv}: Fehler, Unfall, Havarie, Error, Vernichtung, Zerstörung, Schaden \\ 
\textbf{pakala} && \textit{Verb, intransitiv}: vermasseln, auseinander fallen, brechend,  \\ 
\textbf{pakala (e \dots)} && \textit{Verb, transitiv}: verpfuschen, zerstören, ruinieren, verletzen \\ 
 && \\ % no-dictionary
\textbf{\dots pali} && \textit{Adjektiv}: aktiv, tätig, Arbeits-, arbeitend, berufstätig, werktätig \\ 
\textbf{\dots pali} && \textit{Adverb}: aktiv, tätig, arbeitend, berufstätig, werktätig \\ 
\textbf{pali} && \textit{Substantiv}: Tätigkeit, Betätigung, Arbeit, Tat, Projekt \\ 
\textbf{pali} && \textit{Verb, intransitiv}: handeln, arbeiten, funktionieren \\ 
\textbf{pali (e \dots)} && \textit{Verb, transitiv}: tuen, machen, bauen, erstellen \\ 
 && \\ % no-dictionary
\textbf{\dots palisa} && \textit{Adjektiv}: lang, länglich \\
\textbf{palisa} && \textit{Substantiv}: Rute, Stab, Stange, Angel, Stock, Stiel, Ast, Stock \\ 
\textbf{palisa (e \dots)} && \textit{Verb, transitiv}: strecken, sexuell zu erregen \\ 
 && \\ % no-dictionary
\textbf{pan} && \textit{Substantiv}: Getreide, Körner; Gerste, Mais, Hafer, Reis, Weizen, Brot, Nudeln \\ 
\textbf{pan (e \dots)} && \textit{Verb, transitiv}: sähen, aussähen \\ 
 && \\ % no-dictionary
\textbf{\dots pana} && \textit{Adjektiv}: großzügig \\ 
\textbf{pana} && \textit{Substantiv}: Verabreichung, Transformation, Überführung, Umtausch \\ 
\textbf{pana (e \dots)} && \textit{Verb, transitiv}: geben, senden, entbinden, strahlen, emittieren \\ 
 && \\ % no-dictionary
\textbf{\dots pi \dots } && \textit{Separator}: \glqq pi \grqq ermöglicht komplexe Substantive. \\ &&  Es trennt ein Substantiv von einem anderen Substantiv, das mindestens ein Adjektiv hat. \\ && Nach \glqq pi \grqq kann nur ein Substantiv oder Pronomen folgen.  \\  && Verwende ein \glqq pi\grqq nicht zusammmen mit einem der anderen Separatoren. \\ 
 && \\ % no-dictionary
\textbf{\dots pilin} && \textit{Adjektiv}: sensitive, fühlend, empathische \\ 
\textbf{\dots pilin} && \textit{Adverb}: einfühlsam \\ 
\textbf{pilin} && \textit{Substantiv}: Gefühl, Emotion, Herz \\ 
\textbf{pilin} && \textit{Verb, intransitiv}: fühlen, spüren, verspüren, empfinden \\ 
\textbf{pilin (e \dots)} && \textit{Verb, transitiv}: denken, wahrnehmen, berühren, anfassen \\ 
 && \\ % no-dictionary
\textbf{\dots pimeja} && \textit{Adjektiv}: schwarz, dunkel \\ 
\textbf{pimeja} && \textit{Substantiv}: Dunkelheit, Finsternis, Schwärze, Schatten \\ 
\textbf{pimeja (e \dots)} && \textit{Verb, transitiv}: verdunkeln, schwärzen \\ 
 && \\ % no-dictionary
\textbf{\dots pini} && \textit{Adjektiv}: abgeschlossen, vollendet, beendet, erledigt, fertiggestellt  \\ 
\textbf{\dots pini} && \textit{Adverb}: vor, vorbei, vorüber, perfekt, vollendet \\ 
\textbf{pini} && \textit{Substantiv}: Ende, Zweck, Ziel, Tip \\ 
\textbf{pini (e \dots)} && \textit{Verb, transitiv}: beenden, erledigen, fertigstellen, vollenden  \\ 
\textbf{pini \dots } && \textit{Hilfsverb}: stoppen, beenden, unterbrechen \\ 
 && \\ % no-dictionary
\textbf{pipi} && \textit{Substantiv}: Insekt, Käfer, Spinne, Skorpion, Krebs, Krabbe, Schmetterling, Ameise \\ 
 && \\ % no-dictionary
\textbf{\dots poka} && \textit{Adjektiv}: angrenzend, benachbart \\ 
\textbf{poka} && \textit{Substantiv}: Seite, Hüfte, Nähe \\ 
% \textbf{\dots poka \dots} && \textit{Präposition}:  in Gesellschaft von, mit, neben \\ 
 && \\ % no-dictionary
\textbf{poki} && \textit{Substantiv}: Kontainer, Behälter, Box, Schüssel, Schale, Tasse, (Trink-) Glas \\ 
\textbf{poki (e \dots)} && \textit{Verb, transitiv}: einpacken, einwecken, einwickeln, einmachen  \\ 
 && \\ % no-dictionary
\textbf{\dots pona} && \textit{Adjektiv}: gut, einfach, positiv, schön, richtig, korrekt, toll \\ 
\textbf{\dots pona} && \textit{Adverb}: gut, einfach, schön, richtig, korrekt, toll \\ 
\textbf{pona!} && \textit{Interjektion}: Toll! Gut! Danke! OK! Cool! Fantastisch! \\ 
\textbf{pona} && \textit{Substantiv}: das Gute, Einfachheit, Positivismus \\ 
\textbf{pona la \dots} && \textit{Substantiv}: zum Glück, wenn Einfachheit \\ 
\textbf{pona (e \dots)} && \textit{Verb, transitiv}: verbessern, fixen, reparieren, festmachen \\ 
 && \\ % no-dictionary
\textbf{\dots pu} && \textit{Adjektiv}: kaufen und lesen des offiziellen Toki Pona Buchs \\ 
\textbf{pu} && \textit{Substantiv}: Einkauf und Lesung des offiziellen Toki Pona Buchs \\ 
\textbf{pu} && \textit{Verb, intransitiv}: kaufen und lesen des offiziellen Toki Pona Buch \\ 
\textbf{pu (e \dots)} && \textit{Verb, transitiv}: anwenden der Regeln des offiziellen Toki Pona Buches \\ 
\textbf{pu \dots} && \textit{Hilfsverb}: kaufen und lesen des offiziellen Toki Pona Buchs \\ 
 && \\ % no-dictionary
\textbf{\dots sama} && \textit{Adjektiv}: gleichaltrig, ähnlich, paritätisch \\ 
\textbf{\dots sama} && \textit{Adverb}: ähnlich \\ 
\textbf{sama} && \textit{Substantiv}: Gleichheit, Parität, Identität \\ 
\textbf{sama la \dots} && \textit{Substantiv}: bei Gleichheit, bei Parität, bei Identität  \\ 
\textbf{\dots sama \dots} && \textit{Präposition}: wie, dergleichen \\ 
\textbf{sama (e \dots)} && \textit{Verb, transitiv}: gleichzusetzen, gleich machen, nachäffen \\ 
 && \\ % no-dictionary
\textbf{\dots seli} && \textit{Adjektiv}: heiß, warm, gar \\ 
\textbf{\dots seli} && \textit{Adverb}: heiß, warm \\ 
\textbf{seli} && \textit{Substantiv}: Feuer, Wärme, Hitze \\ 
\textbf{seli (e \dots)} && \textit{Verb, transitiv}: erhitzen, ermärmen, aufwärmen, kochen \\ 
 && \\ % no-dictionary
\textbf{selo} && \textit{Substantiv}: Außenseite, Oberfläche, Haut, Schale, Rinde, Borke, Gestalt, Form \\ 
\textbf{selo (e \dots)} && \textit{Verb, transitiv}: schützen, bewachen \\ 
 && \\ % no-dictionary
\textbf{seme} && \textit{Fragepronomen}: wer, wie, was, welche, warum, wessen (Fragewort) \\ 
 && \\ % no-dictionary
\textbf{\dots sewi} && \textit{Adjektiv}: übergeordnet, oberer, erhöht, erhaben, religiös, gläubig, formell \\ 
\textbf{\dots sewi} && \textit{Adverb}: übergeordnet, erhöht, erhaben, gläubig, formell \\ 
\textbf{sewi} && \textit{Substantiv}: Höhe, Himmel, Dach, Gipfel, Spitze, Krone \\ 
\textbf{sewi} && \textit{Verb, intransitiv}: aufstehen \\ 
\textbf{sewi (e \dots)} && \textit{Verb, transitiv}: heben, anheben \\ 
 && \\ % no-dictionary
\textbf{\dots sijelo} && \textit{Adjektiv}:  körperlich, materiellen, fleischlichen \\ 
\textbf{\dots sijelo} && \textit{Adverb}: körperlich \\ 
\textbf{sijelo} && \textit{Substantiv}: Körper, Leib, Gesundheit, physischer Zustand \\ 
\textbf{sijelo (e \dots)} && \textit{Verb, transitiv}: heilen \\ 
 && \\ % no-dictionary
\textbf{\dots sike} && \textit{Adjektiv}:  rund, konjunkturell, periodisch \\ 
\textbf{\dots sike} && \textit{Adverb}: gedreht \\ 
\textbf{sike} && \textit{Substantiv}: Kreis, Rad, Kugel, Ball, Kreislauf, Periode, Zyklus \\ 
\textbf{sike (e \dots)} && \textit{Verb, transitiv}: kreisen, drehen \\ 
 && \\ % no-dictionary
\textbf{\dots sin} && \textit{Adjektiv}: neu, frisch, noch einer, ein anderes, mehr \\ 
\textbf{ \dots sin } && \textit{Adverb}: erneuernd \\ 
\textbf{sin} && \textit{Substantiv}: Nachrichten, Neuheit, Innovation \\ 
\textbf{sin (e \dots)} && \textit{Verb, transitiv}: erneuern, renovieren, auffrischen, beleben \\ 
 && \\ % no-dictionary
\textbf{sina} && \textit{Personalpronomen}: du, ihr (Plural)  \\ 
\textbf{\dots sina} && \textit{Possessivpronomen}: deins, euer  \\  
\textbf{\dots e sina} && \textit{Reflexivpronomen}: dich, euch  \\  
 && \\ % no-dictionary
\textbf{\dots sinpin} && \textit{Adjektiv}: Gesichts-, frontal, Vorder-, vertikal \\ 
\textbf{sinpin} && \textit{Substantiv}: Front, Vorderseite, Gesicht, Brust, Brustkorb, Rumpf, Wand, Mauer \\ 
 && \\ % no-dictionary
\textbf{\dots sitelen} && \textit{Adjektiv}: bildlich, figurative, bildhafte, metaphorisch \\ 
\textbf{\dots sitelen} && \textit{Adverb}: bildlich \\ 
\textbf{sitelen} && \textit{Substantiv}: Bild, Abbildung, Foto, Film, Darstellung, Gemälde \\ 
\textbf{sitelen (e \dots)} && \textit{Verb, transitiv}: zeichnen, malen, schreiben \\ 
 && \\ % no-dictionary
\textbf{\dots sona} && \textit{Adjektiv}: wissend, verständnisvoll, cognizant, klug, schlau, clever \\ 
\textbf{sona} && \textit{Substantiv}: Kenntnis, Wissen, Erkenntnis, Weisheit, Intelligenz, Verständnis \\ 
\textbf{sona} && \textit{Verb, intransitiv}: wissen, können, verstehen \\ 
\textbf{sona (e \dots)} && \textit{Verb, transitiv}: wissen, können, verstehen, zu etwas fähig sein \\ 
kama \textbf{sona (e \dots)} && \textit{Verb, transitiv}: lernen, studieren \\ 
\textbf{sona \dots} && \textit{Hilfsverb}: wissen \\ 
 && \\ % no-dictionary
\textbf{\dots soweli} && \textit{Adjektiv}: tierisch \\ 
\textbf{soweli} && \textit{Substantiv}: (Land-) Säugetier, Hund, Kuh, Maus, Elefant, Katze, Fleisch \\ 
 && \\ % no-dictionary
\textbf{\dots suli} && \textit{Adjektiv}: groß, schwer, wichtig, lang, erwachsen \\ 
\textbf{\dots suli} && \textit{Adverb}: schwer, wichtig, erwachsen \\ 
\textbf{suli} && \textit{Substantiv}: Größe, Format \\ 
\textbf{suli (e \dots)} && \textit{Verb, transitiv}: vergrößern, ausbauen, ausdehnen, verlängern \\ 
 && \\ % no-dictionary
\textbf{\dots suno} && \textit{Adjektiv}: sonnig \\ 
\textbf{\dots suno} && \textit{Adverb}: sonnig \\ 
\textbf{suno} && \textit{Substantiv}: Sonne, Licht, Helligkeit \\ 
\textbf{suno (e \dots)} && \textit{Verb, transitiv}: scheinen, beleuchten \\ 
 && \\ % no-dictionary
\textbf{\dots supa} && \textit{Adjektiv}: flach, eben, plan, platt, seicht, horizontal \\ 
\textbf{supa} && \textit{Substantiv}: horizontale Fläche, Möbel, Tisch, Stuhl, Kissen, Fußboden \\ 
 && \\ % no-dictionary
\textbf{\dots suwi} && \textit{Adjektiv}: süß, niedlich \\ 
\textbf{suwi} && \textit{Substantiv}: Süssigkeiten, Zucker, Schokolade, Bonbon \\ 
\textbf{suwi (e \dots)} && \textit{Verb, transitiv}: süßen, versüßen \\ 
 && \\ % no-dictionary
\textbf{\dots tan} && \textit{Adjektiv}: kausal, ursächlich \\ 
\textbf{tan} && \textit{Substantiv}: Ursprung, Abstammung, Herkunft, Entstehung, Grund, Ursache \\ 
\textbf{\dots tan \dots} && \textit{Präposition}: von, aus, durch, wegen, weil, seit \\ 
\textbf{tan} && \textit{Verb, intransitiv}: stammen aus, kommen aus \\ 
 && \\ % no-dictionary
\textbf{\dots taso} && \textit{Adjektiv}: nur, einzig, einzig, alleinig \\ 
\textbf{\dots taso} && \textit{Adverb}: nur, einzig, einzig, alleinig \\ 
\textbf{\dots taso \dots} && \textit{Konjunktion}: aber, sondern, doch \\ 
 && \\ % no-dictionary
\textbf{\dots tawa} && \textit{Adjektiv}: beweglich, mobil \\ 
\textbf{\dots tawa} && \textit{Adverb}: beweglich, mobil \\ 
\textbf{tawa} && \textit{Substantiv}: Bewegung, Regung, Transport \\ 
\textbf{\dots tawa \dots} && \textit{Präposition}: zu, um zu, zu hin, nach, für, bis, in \\ 
\textbf{tawa} && \textit{Verb, intransitiv}: gehen, spazieren, verlassen, abfahren, reisen \\ 
\textbf{tawa (e \dots)} && \textit{Verb, transitiv}: bewegen, verschieben, verlagern \\ 
 && \\ % no-dictionary
\textbf{\dots telo} && \textit{Adjektiv}: feucht, klebrig, verschwitzt \\ 
\textbf{\dots telo} && \textit{Adverb}: feucht, klebrig, verschwitzt \\ 
\textbf{telo} && \textit{Substantiv}: Wasser, Flüssigkeit, Feuchtigkeit, Saft, Soße \\ 
\textbf{telo (e \dots)} && \textit{Verb, transitiv}: begießen, bewässern, befeuchten, waschen, schmelzen \\ 
 && \\ % no-dictionary
\textbf{\dots tenpo} && \textit{Adjektiv}: zeitliche, chronologische, chronologisch \\ 
\textbf{\dots tenpo} && \textit{Adverb}: chronologisch \\ 
\textbf{tenpo} && \textit{Substantiv}: Zeit, Zeitperiode, Zeitabschnitt, Moment, Dauer, Situation \\ 
 && \\ % no-dictionary
\textbf{\dots toki} && \textit{Adjektiv}: eloquent, sprachlichen, verbal, grammatische \\ 
\textbf{\dots toki} && \textit{Adverb}: eloquent, sprachlichen, verbal, grammatische \\ 
\textbf{toki!} && \textit{Interjektion}: Hallo! Hi! \\ 
\textbf{toki} && \textit{Substantiv}: Sprache, Gespräch, Rede, Ansprache, Kommunikation, Plauderei \\ 
\textbf{toki} && \textit{Verb, intransitiv}: reden, sprechen, sich unterhalten, plaudern \\ 
\textbf{toki (e \dots)} && \textit{Verb, transitiv}: sagen \\ 
 && \\ % no-dictionary
\textbf{\dots tomo} && \textit{Adjektiv}: städtebaulich, Stadt-, häuslich \\ 
\textbf{\dots tomo} && \textit{Adverb}: häuslich \\ 
\textbf{tomo} && \textit{Substantiv}: Gebäude, Bauwerk, Haus, Raum, Konstruktion \\ 
\textbf{tomo (e \dots)} && \textit{Verb, transitiv}: bauen, konstruieren \\ 
 && \\ % no-dictionary
\textbf{\dots tu} && \textit{Adjektiv, Ziffer}: zwei, 2 \\ 
\textbf{tu} && \textit{Substantiv}: Duo, Paar \\ 
\textbf{tu (e \dots)} && \textit{Verb, transitiv}: verdoppeln, zweiteilen, teilen, zerschneiden \\ 
 && \\ % no-dictionary
\textbf{\dots unpa} && \textit{Adjektiv}:  erotisch, sexuell, sexy \\ 
\textbf{\dots unpa} && \textit{Adverb}:  erotisch, sexuell \\ 
\textbf{unpa} && \textit{Substantiv}: Geschlecht, Sex, Sexualität \\ 
\textbf{unpa} && \textit{Verb, intransitiv}: Sex haben \\ 
\textbf{unpa (e \dots)} && \textit{Verb, transitiv}: mit jemandem schlafen, mit jemandem Sex haben \\ 
 && \\ % no-dictionary
\textbf{\dots uta} && \textit{Adjektiv}: mündlich, oral \\ 
\textbf{\dots uta} && \textit{Adverb}: mündlich, oral \\ 
\textbf{uta} && \textit{Substantiv}: Mund, Maul, Schnabel \\ 
\textbf{uta (e \dots)} && \textit{Verb, transitiv}: küssen, saugen \\ 
 && \\ % no-dictionary
\textbf{\dots utala} && \textit{Adjektiv}: kämpfend \\ 
\textbf{\dots utala} && \textit{Adverb}: kämpfend \\ 
\textbf{utala} && \textit{Substantiv}: Konflikt, Disharmonie, Kampf, Krieg, Streit, Gewalt \\ 
\textbf{utala (e \dots)} && \textit{Verb, transitiv}: schlagen, stoßen, überfallen, kämpfen \\ 
 && \\ % no-dictionary
\textbf{\dots walo} && \textit{Adjektiv}: weiß, hell, bleich \\ 
\textbf{walo} && \textit{Substantiv}: Helligkeit, Weiße \\ 
\textbf{walo (e \dots)} && \textit{Verb, transitiv}: bleichen \\ 
 && \\ % no-dictionary
\textbf{\dots wan} && \textit{Adjektiv, Ziffer}: 1, eins, ein \\ 
\textbf{wan} && \textit{Substantiv}: Einheit, Element, Partikel, Teil, Abschnitt, Stück \\ 
\textbf{wan (e \dots)} && \textit{Verb, transitiv}: einigen, vereinigen, zusammenführen \\ 
 && \\ % no-dictionary
\textbf{\dots waso} && \textit{Adjektiv}: vogel-, vogelartig \\ 
\textbf{waso} && \textit{Substantiv}: Vogel, Fledermaus, Geflügel \\ 
 && \\ % no-dictionary
\textbf{\dots wawa} && \textit{Adjektiv}: energisch, tatkräftig, stark, hitzig, intensiv \\ 
\textbf{\dots wawa} && \textit{Adverb}: stark, kraftvoll \\ 
\textbf{wawa} && \textit{Substantiv}: Energie, Stärke, Mumm, Leistung, Kraft \\ 
\textbf{wawa (e \dots)} && \textit{Verb, transitiv}: bestärken, verstärken, ermächtigen \\ 
 && \\ % no-dictionary
\textbf{\dots weka} && \textit{Adjektiv}: weg, fort, entfernt, abwesend, fehlend, versäumend \\ 
\textbf{weka} && \textit{Substantiv}: Abwesenheit (von), Fehlen, Mangel, Absenz \\ 
\textbf{weka (e \dots)} && \textit{Verb, transitiv}: wegwerfen, entfernen, abtragen,  \\ 
 && \\ % no-dictionary
\textbf{\dots wile} && \textit{Adjektiv}: notwendig,  nötig, erforderlich \\ 
%\textbf{\dots wile} && \textit{Adverb}: notwendig \\ 
\textbf{wile} && \textit{Substantiv}: Begehren, Wunsch, Verlangen (nach), Lust (auf), Trieb, Zwang \\ 
\textbf{wile (e \dots)} && \textit{Verb, transitiv}: benötigen, brauchen, müssen, wünschen, wollen, sollen \\ 
\textbf{wile \dots} && \textit{Hilfsverb}: wollen, wünschen, müssen \\ 
\end{supertabular} \\
%
%%%%%%%%%%%%%%%%%%%%%%%%%%%%%%%%%%%%%%%%%%%%%%%%%%%%%%%%%%%%%%%%%%%%%%%%%%
% eof
