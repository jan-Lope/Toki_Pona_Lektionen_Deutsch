%%%%%%%%%%%%%%%%%%%%%%%%%%%%%%%%%%%%%%%%%%%%%%%%%%%%%%%%%%%%%%%%%%%%%%%%%%
\section{Andere Pr�positionen und ortsbezogene Substantive}
%%%%%%%%%%%%%%%%%%%%%%%%%%%%%%%%%%%%%%%%%%%%%%%%%%%%%%%%%%%%%%%%%%%%%%%%%%
\index{Pr�position}
\subsection*{Vokabeln}
%%%%%%%%%%%%%%%%%%%%%%%%%%%%%%%%%%%%%%%%%%%%%%%%%%%%%%%%%%%%%%%%%%%%%%%%%%
%
% \index{\textit{kan}}
\index{\textit{sama}}
\index{\textit{tan}}
\index{\textit{anpa}} 
\index{\textit{insa}}
\index{\textit{monsi}}
\index{\textit{poka}}
\index{\textit{sewi}}
\index{\textit{sona}}
\index{gleich}
\index{�hnlich}
\index{wie}
\index{Gleichheit}
\index{von}
\index{aus}
\index{weil}
\index{durch}
\index{seit}
\index{Ursache}
\index{Grund}
\index{Anla�}
\index{kennen}
\index{k�nnen}
\index{auskennen}
\index{wissen}
\index{Wissen}
\index{Weisheit}
\index{unten}
\index{tief}
\index{unterlegen}
\index{besiegen}
\index{absenken}
\index{Grund}
\index{Boden}
\index{zentral}
\index{innerhalb}
\index{Magen}
\index{Innereien}
\index{Bauchh�hle}
\index{Hinter-}
\index{R�cken}
\index{Hintern}
\index{neben}
\index{nahe}
\index{angrenzend}
\index{benachbart}
\index{N�he}
\index{H�fte}
\index{oberer}
\index{erh�ht}
\index{Dach}
\index{Spitze}
\index{Himmel}
\begin{supertabular}{p{5,5cm}|ll}
% kan && with, among, in the company of \\
sama && gleich, �hnlich, wie, Gleichheit \\
tan && von, aus, weil, durch, seit, Ursache, Grund, Anla� \\
sona && kennen, k�nnen, auskennen, wissen, Wissen, Weisheit \\
anpa && unten, tief, unterlegen, besiegen, absenken, Grund, Boden \\ 
insa && zentral, innerhalb, Magen, Innereien, Bauchh�hle \\
monsi && Hinter-, R�cken, Hintern \\
poka && neben, nahe, angrenzend, benachbart, N�he, H�fte \\
sewi && oberer, erh�ht, Dach, Spitze, Himmel \\
\end{supertabular} 
%
%%%%%%%%%%%%%%%%%%%%%%%%%%%%%%%%%%%%%%%%%%%%%%%%%%%%%%%%%%%%%%%%%%%%%%%%%%
\index{\textit{same}}
\subsection*{\textit{sama}}
%%%%%%%%%%%%%%%%%%%%%%%%%%%%%%%%%%%%%%%%%%%%%%%%%%%%%%%%%%%%%%%%%%%%%%%%%%
% This word can be used for several different parts of speech, but I don't think that it's too difficult to understand. 
%
\begin{supertabular}{p{5,5cm}|ll}
jan ni li ' \textbf{sama} mi. && Diese Person ist mir �hnlich. \\
ona li lukin, \textbf{sama} pipi. && Er guckt wie ein K�fer. \\
\textbf{sama} li ' ike. && Gleichheit ist schlecht. \\
\end{supertabular} 
%
%%%%%%%%%%%%%%%%%%%%%%%%%%%%%%%%%%%%%%%%%%%%%%%%%%%%%%%%%%%%%%%%%%%%%%%%%%
\index{\textit{tan}!Pr�position}
\index{Pr�position!\textit{tan}}
\subsection*{tan}
%%%%%%%%%%%%%%%%%%%%%%%%%%%%%%%%%%%%%%%%%%%%%%%%%%%%%%%%%%%%%%%%%%%%%%%%%%
\textbf{\textit{tan} als Pr�position} 

\begin{supertabular}{p{5,5cm}|ll}
mi moku, \textbf{tan} ni: mi wile moku. &&  Ich esse weil ich hungrig bin. \\
% mi \textbf{tan} ma ike.   &&       Ich komme aus einem schlechten Land. \\
\end{supertabular} 

\index{\textit{tan}!Substantiv}
\index{Substantiv!\textit{tan}}
\textbf{\textit{tan} als Substantiv}

Wenn \textit{tan} als Substantiv verwendet wird, bedeutet es 'Ursache' oder 'Grund'.

\begin{supertabular}{p{5,5cm}|ll}
mi sona e \textbf{tan}. && Ich kenne die Ursache. \\
\end{supertabular} 
%
\index{\textit{poka}}
\index{\textit{poka}!\textit{Substantiv}}
\index{Substantiv!\textit{poka}}
\subsection*{\textit{poka}}
%%%%%%%%%%%%%%%%%%%%%%%%%%%%%%%%%%%%%%%%%%%%%%%%%%%%%%%%%%%%%%%%%%%%%%%%%%
%
poka kann als Adjektiv, Substantiv oder Pr�position verwendet werden.

\textbf{\textit{poka} als Adjektiv bzw. zusammengesetztes Substantiv} \\

\begin{supertabular}{p{5,5cm}|ll}
jan \textbf{poka} && Nachbar, jemand neben mir \\
\textbf{poka} telo && Strand, Ufer \\
\end{supertabular} 

\textbf{\textit{poka} als Pr�position} \\
% \textit{poka} kann direkt als Pr�position verwendet werden.

\begin{supertabular}{p{5,5cm}|ll}
mi moku, \textbf{poka} jan pona mi. && Ich esse an der Seite meines Freundes. \\
\end{supertabular} 
   
%
\newpage
%%%%%%%%%%%%%%%%%%%%%%%%%%%%%%%%%%%%%%%%%%%%%%%%%%%%%%%%%%%%%%%%%%%%%%%%%%
\index{\textit{anpa}}
\index{\textit{insa}}
\index{\textit{monsi}}
\index{\textit{poka}}
\index{\textit{sewi}}
\subsection*{Die ortsbezogene Substantive \textit{anpa}, \textit{insa}, \textit{monsi}, \textit{sewi}}
%%%%%%%%%%%%%%%%%%%%%%%%%%%%%%%%%%%%%%%%%%%%%%%%%%%%%%%%%%%%%%%%%%%%%%%%%%
%
\index{\textit{anpa}!Substantiv}
\index{\textit{insa}!Substantiv}
\index{\textit{monsi}!Substantiv}
\index{\textit{sewi}!Substantiv}
\index{Substantiv!\textit{anpa}}
\index{Substantiv!\textit{insa}}
\index{Substantiv!\textit{monsi}}
\index{Substantiv!\textit{sewi}}
\textbf{Substantive} \\
M�glicherweise m�chtest man diese W�rter als Pr�positionen benutzen. 
Sie werden aber als Substantive verwendet. 
Hier eine w�rtliche �bersetzung.

\begin{supertabular}{p{5,5cm}|ll}
ona li \textbf{lon} sewi mi.    &&  Er ist in dem Bereich �ber mir. \\
pipi li \textbf{lon} anpa mi.   &&  Der K�fer ist an meiner Unterseite. \\
moku li \textbf{lon} insa mi.   &&  Das Essen ist in meinem Inneren. \\
mi moku \textbf{lon} poka sina. &&  Ich esse an deiner Seite. \\
\end{supertabular} 

\textit{anpa}, \textit{insa}, \textit{monsi} und \textit{sewi} sind hier also Substantive und \textit{mi} ist hier das besitzanzeigendes Possessiv-Pronomen.
\textit{sewi mi} hei�t etwa 'der Bereich �ber mir'. 
Substantive m�ssen ja immer mit einem Verb verwendet werden.
In den Beispielen ist \textit{lon} das Verb. 

\textbf{\textit{monsi} als K�rperteil} \\
\begin{supertabular}{p{5,5cm}|ll}
monsi && R�cken, Hintern, Po \\
\end{supertabular} 

\index{\textit{anpa}!Verb}
\index{Verb!\textit{anpa}}
\index{besiegen}
\textbf{\textit{anpa} als Verb} \\
\begin{supertabular}{p{5,5cm}|ll}
mi \textbf{anpa} e jan utala. && Ich besiegte den Soldaten. \\
\end{supertabular} 
%

%%%%%%%%%%%%%%%%%%%%%%%%%%%%%%%%%%%%%%%%%%%%%%%%%%%%%%%%%%%%%%%%%%%%%%%%%%
\subsection*{�bungen 7 (Antworten siehe Seite~\pageref{'other_prepositions'})}
%%%%%%%%%%%%%%%%%%%%%%%%%%%%%%%%%%%%%%%%%%%%%%%%%%%%%%%%%%%%%%%%%%%%%%%%%%
%
\begin{supertabular}{p{5,5cm}|ll}
Mein Freund ist an meiner Seite. && \\ % no-dictionary
Die Sonne ist �ber mir. && \\ % no-dictionary
Das Land ist unter mir. && \\ % no-dictionary
B�se Dinge liegen hinter mir. && \\ % no-dictionary
Mir geht es gut weil ich lebe. * && \\ % no-dictionary
Ich schaue mir das Land mit meinem Freund an. && \\ % no-dictionary
Leute sehen wie Ameisen aus. && \\ % no-dictionary
 && \\ % no-dictionary
poka mi li ' pakala. && \\ % no-dictionary
mi kepeken poki li kepeken ilo moku. && \\ % no-dictionary
jan li lon insa tomo. && \\ % no-dictionary
\end{supertabular} 

* \textit{lon} als Verb alleine bedeutet existieren bzw. real sein.
% 
%%%%%%%%%%%%%%%%%%%%%%%%%%%%%%%%%%%%%%%%%%%%%%%%%%%%%%%%%%%%%%%%%%%%%%%%%%
% eof
