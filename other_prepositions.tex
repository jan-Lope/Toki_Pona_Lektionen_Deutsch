%%%%%%%%%%%%%%%%%%%%%%%%%%%%%%%%%%%%%%%%%%%%%%%%%%%%%%%%%%%%%%%%%%%%%%%%%%
%
\section{Relative Ortsangaben}
%
%%%%%%%%%%%%%%%%%%%%%%%%%%%%%%%%%%%%%%%%%%%%%%%%%%%%%%%%%%%%%%%%%%%%%%%%%%
%
\subsection*{Vokabeln}
%%%%%%%%%%%%%%%%%%%%%%%%%%%%%%%%%%%%%%%%%%%%%%%%%%%%%%%%%%%%%%%%%%%%%%%%%%
%
\begin{supertabular}{p{2,5cm}|ll}
%
\index{anpa}
\textbf{\dots anpa} && \textit{Adjektiv}: unten, tief, niedrig \\ % no-dictionary
\textbf{\dots anpa} && \textit{Adverb}: tief, niedrig \\ % no-dictionary
\textbf{anpa} && \textit{Substantiv}: Boden, Erdboden, Grund, Talsohle \\ % no-dictionary
\textbf{anpa} && \textit{Verb, intransitiv}:  sich unterwerfen \\ % no-dictionary
\textbf{anpa (e \dots)} && \textit{Verb, transitiv}: niederlassen, absenken, abseilen, herunterholen, besiegen, bezwingen \\ % no-dictionary
 && \\ % no-dictionary
%
\index{insa}
\textbf{\dots insa} && \textit{Adjektiv}: intern, zentral \\ % no-dictionary
\textbf{insa} && \textit{Substantiv}: Innere, Innenseite, Zentrum, Magen \\ % no-dictionary
 && \\ % no-dictionary
%
\index{monsi}
\textbf{\dots monsi} && \textit{Adjektiv}: R�ck-, Hinter- \\ % no-dictionary
\textbf{monsi} && \textit{Substantiv}: R�cken, Heck, Hintern, Po, Arsch \\ % no-dictionary
 && \\ % no-dictionary
%
\index{noka}
\textbf{\dots noka} && \textit{Adjektiv}: Fuss-, niedriger, unten \\  % no-dictionary
\textbf{\dots noka } && \textit{adverb}: zu Fuss \\  % no-dictionary
\textbf{noka} && \textit{Substantiv}: Bein, Fu� \\  % no-dictionary
 && \\ % no-dictionary
%
\index{poka}
\textbf{\dots poka} && \textit{Adjektiv}: angrenzend, benachbart \\ % no-dictionary
\textbf{poka} && \textit{Substantiv}: Seite, H�fte, N�he \\ % no-dictionary
% \textbf{\dots , poka \dots} && \textit{Pr�position}:  in Gesellschaft von, mit, neben \\ % no-dictionary
 && \\ % no-dictionary
%
\index{sewi}
\textbf{\dots sewi} && \textit{Adjektiv}: �bergeordnet, oberer, erh�ht, erhaben, religi�s, gl�ubig, formell \\ 
\textbf{\dots sewi} && \textit{Adverb}: �bergeordnet, erh�ht, erhaben, gl�ubig, formell \\ 
\textbf{sewi} && \textit{Substantiv}: H�he, Himmel, Dach, Gipfel, Spitze, Krone \\ 
\textbf{sewi} && \textit{Verb, intransitiv}: aufstehen \\ 
\textbf{sewi (e \dots)} && \textit{Verb, transitiv}: heben, anheben \\ 
 && \\ % no-dictionary
%
\index{sinpin}
\textbf{\dots sinpin} && \textit{Adjektiv}: Gesichts-, frontal, Vorder-, vertikal \\  % no-dictionary
\textbf{sinpin} && \textit{Substantiv}: Front, Vorderseite, Gesicht, Brust, Brustkorb, Rumpf, Wand, Mauer \\  % no-dictionary
\end{supertabular}
%
%%%%%%%%%%%%%%%%%%%%%%%%%%%%%%%%%%%%%%%%%%%%%%%%%%%%%%%%%%%%%%%%%%%%%%%%%%
\newpage
%
\subsection*{Die ortsbezogene Substantive \textit{anpa}, \textit{insa}, \textit{monsi}, \textit{noka}, \textit{poka}, \textit{sewi} und \textit{sinpin}}
%
\index{Substantiv!ortsbezogen}
\index{ortsbezoges Substantiv}
\index{\textit{anpa}!ortsbezogenes Substantiv}
\index{\textit{insa}!ortsbezogenes Substantiv}
\index{\textit{monsi}!ortsbezogenes Substantiv}
\index{\textit{noka}!ortsbezogenes Substantiv}
\index{\textit{poka}!ortsbezogenes Substantiv}
\index{\textit{sewi}!ortsbezogenes Substantiv}
\index{\textit{sinpin}!ortsbezogenes Substantiv}
%%%%%%%%%%%%%%%%%%%%%%%%%%%%%%%%%%%%%%%%%%%%%%%%%%%%%%%%%%%%%%%%%%%%%%%%%%
%

In \textit{toki pona} werden relative Ortsangaben mit speziellen Substantiven gebildet. 
Diese speziellen Substantive nennt man 'ortsbezogene Substantive'. 
F�r die relativen Ortsangaben sind neben dem Substantiv noch Adjektive, Possessivpronomen oder Demonstrativpronomen notwendig. 
%
Vor ortsbezogenes Substantiven steht entweder ein intransitives Verb oder eine Pr�position. 
Das hei�t, relative Ortsangaben befinden sich entweder in eimem indirekten Objekt oder einem Pr�positionalobjekt und sind damit Teil einer Pr�dikatsphrase.

%
\subsubsection*{Ortsbezogene Substantive in einem indirekten Objekt}
%
\index{ortsbezoges Substantiv!indirektes Objekt}
%%%%%%%%%%%%%%%%%%%%%%%%%%%%%%%%%%%%%%%%%%%%%%%%%%%%%%%%%%%%%%%%%%%%%%%%%%
%
\index{\textit{lon}!intransitiven Verb}
\index{\textit{lon}!Pr�position}
%

Meist wird vor ortsbezogenes Substantive das intransitiven Verb \textit{lon} oder die Pr�position \textit{lon} verwendet. 
Ist vor \textit{lon} kein Verb vorhanden, kann \textit{lon} keine Pr�position sein. 
In diesen Beispielen wird das intransitive Verb \textit{lon} verwendet.

\begin{supertabular}{p{5,5cm}|ll}
pipi li lon anpa mi.       && Der K�fer ist an meiner Unterseite. \\
moku li lon insa mi.       && Das Essen ist in meinem Inneren. \\
telo suli li lon monsi mi. && Das Meer liegt hinter mir. \\
ma li lon noka mi.         && Land ist unter meinen F��en. \\
ona li lon sewi mi.        && Er ist in dem Bereich �ber mir. \\
tomo li lon sinpin mi.     && Das Haus befindet sich vor mir. \\
\end{supertabular} 

%
\subsubsection*{Ortsbezogene Substantive in einem Pr�positionalobjekt}
%
\index{ortsbezoges Substantiv!Pr�positionalobjekt}
%%%%%%%%%%%%%%%%%%%%%%%%%%%%%%%%%%%%%%%%%%%%%%%%%%%%%%%%%%%%%%%%%%%%%%%%%%
%

In den folgenden Beispielen ist ein Verb vorhanden. 
Also wird die Pr�position \textit{lon} verwendet. 

\begin{supertabular}{p{5,5cm}|ll}
mi moku, lon poka sina.    && Ich esse an deiner Seite. \\
ona li pona e ilo, lon tomo ona. && Er repariert das Werkzeug in seinem Haus. \\
\end{supertabular} 

In diesem Satz ist das zweite \textit{tawa} eine Pr�position und steht vor dem ortsbezogenen Substantiv \textit{noka}. 

\begin{supertabular}{p{5,5cm}|ll}
mi tawa e mi, tawa noka sina. && Ich verneige mich vor dir. \\
\end{supertabular} 

%
%
%
%%%%%%%%%%%%%%%%%%%%%%%%%%%%%%%%%%%%%%%%%%%%%%%%%%%%%%%%%%%%%%%%%%%%%%%%%%
\subsection*{Weitere Bedeutungen dieser W�rter}
%
%%%%%%%%%%%%%%%%%%%%%%%%%%%%%%%%%%%%%%%%%%%%%%%%%%%%%%%%%%%%%%%%%%%%%%%%%%
%
%
%%%%%%%%%%%%%%%%%%%%%%%%%%%%%%%%%%%%%%%%%%%%%%%%%%%%%%%%%%%%%%%%%%%%%%%%%%
\subsubsection*{Das transitive Verb \textit{anpa}}
%
\index{\textit{anpa}!Verb}
%%%%%%%%%%%%%%%%%%%%%%%%%%%%%%%%%%%%%%%%%%%%%%%%%%%%%%%%%%%%%%%%%%%%%%%%%%

\begin{supertabular}{p{5,5cm}|ll}
mi anpa e jan utala. && Ich besiegte den Soldaten. \\
\end{supertabular} 

%
%%%%%%%%%%%%%%%%%%%%%%%%%%%%%%%%%%%%%%%%%%%%%%%%%%%%%%%%%%%%%%%%%%%%%%%%%%
\subsubsection*{Das 'normale' Substantiv \textit{poka}}
%
\index{\textit{poka}!Substantiv}
%%%%%%%%%%%%%%%%%%%%%%%%%%%%%%%%%%%%%%%%%%%%%%%%%%%%%%%%%%%%%%%%%%%%%%%%%%

\begin{supertabular}{p{5,5cm}|ll}
poka telo && Strand, Ufer \\
\end{supertabular} 

%
%%%%%%%%%%%%%%%%%%%%%%%%%%%%%%%%%%%%%%%%%%%%%%%%%%%%%%%%%%%%%%%%%%%%%%%%%%
\subsubsection*{Das Adjektiv \textit{poka}}
%
\index{\textit{poka}!Adjektiv}
%%%%%%%%%%%%%%%%%%%%%%%%%%%%%%%%%%%%%%%%%%%%%%%%%%%%%%%%%%%%%%%%%%%%%%%%%%

\begin{supertabular}{p{5,5cm}|ll}
jan poka && Nachbar \\
\end{supertabular} 

%
%
%
%%%%%%%%%%%%%%%%%%%%%%%%%%%%%%%%%%%%%%%%%%%%%%%%%%%%%%%%%%%%%%%%%%%%%%%%%%
\newpage
%
\subsection*{�bungen (Antworten siehe Seite~\pageref{'other_prepositions'})}
%%%%%%%%%%%%%%%%%%%%%%%%%%%%%%%%%%%%%%%%%%%%%%%%%%%%%%%%%%%%%%%%%%%%%%%%%%
%
Schreibe bitte die Antworten auf einen Zettel und �berpr�fe sie anschlie�end. 

\begin{supertabular}{p{5,5cm}|ll}
Wie bildet man relative Ortsangaben in \textit{toki pona}? &&  \\ % no-dictionary
Was ist ein Possessivpronomen? &&  \\ % no-dictionary
Wo ist ein Slot f�r ein substantivisches Demonstrativpronomen m�glich? &&  \\ % no-dictionary
Welcher Separator steht am Ende eines Aussagesatzes? &&  \\ % no-dictionary
Was ist ein Pr�dikatsadjektiv? &&  \\ % no-dictionary
In welchen Satzphrasen k�nnen sich ortsbezogenen Substantive befinden? &&  \\ % no-dictionary
\end{supertabular}

Versuche diese S�tze zu �bersetzen. 
Mit dem Tool \textit{Toki Pona Parser} (\cite{www:rowa:02}) kann man Rechtschreibung und Grammatik �berpr�fen. 

\begin{supertabular}{p{5,5cm}|ll}
Mein Freund ist an meiner Seite. && \\ % no-dictionary
Die Sonne ist �ber mir. && \\ % no-dictionary
Das Land ist unter mir. && \\ % no-dictionary
B�se Dinge liegen hinter mir. && \\ % no-dictionary
Mir geht es gut, weil ich lebe. * && \\ % no-dictionary
Ich schaue mir das Land mit dir an. && \\ % no-dictionary
\end{supertabular}

\begin{supertabular}{p{5,5cm}|ll}
poka mi li ' pakala. && \\ % no-dictionary
mi kepeken poki li kepeken ilo moku. && \\ % no-dictionary
jan li lon insa tomo. && \\ % no-dictionary
\end{supertabular} 

* \textit{lon} als Verb alleine bedeutet existieren bzw. real sein.
% 
%%%%%%%%%%%%%%%%%%%%%%%%%%%%%%%%%%%%%%%%%%%%%%%%%%%%%%%%%%%%%%%%%%%%%%%%%%
% eof
