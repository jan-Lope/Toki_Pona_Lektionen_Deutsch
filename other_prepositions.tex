%%%%%%%%%%%%%%%%%%%%%%%%%%%%%%%%%%%%%%%%%%%%%%%%%%%%%%%%%%%%%%%%%%%%%%%%%%
\section{Ortsbezogene Substantive}
%%%%%%%%%%%%%%%%%%%%%%%%%%%%%%%%%%%%%%%%%%%%%%%%%%%%%%%%%%%%%%%%%%%%%%%%%%
\index{Pr�position}
\subsection*{Vokabeln}
%%%%%%%%%%%%%%%%%%%%%%%%%%%%%%%%%%%%%%%%%%%%%%%%%%%%%%%%%%%%%%%%%%%%%%%%%%
%
% \index{\textit{kan}}
\index{\textit{sama}}
\index{\textit{tan}}
\index{\textit{anpa}} 
\index{\textit{insa}}
\index{\textit{monsi}}
\index{\textit{poka}}
\index{\textit{sewi}}
\index{\textit{sona}}
\index{\textit{pipi}}
\index{gleich}
\index{�hnlich}
\index{wie}
\index{Gleichheit}
\index{von}
\index{aus}
\index{weil}
\index{durch}
\index{seit}
\index{Ursache}
\index{Grund}
\index{Anla�}
\index{kennen}
\index{k�nnen}
\index{auskennen}
\index{wissen}
\index{Wissen}
\index{Weisheit}
\index{unten}
\index{tief}
\index{unterlegen}
\index{besiegen}
\index{absenken}
\index{Grund}
\index{Boden}
\index{zentral}
\index{innerhalb}
\index{Magen}
\index{Innereien}
\index{Bauchh�hle}
\index{Hinter-}
\index{R�cken}
\index{Hintern}
\index{neben}
\index{nahe}
\index{angrenzend}
\index{benachbart}
\index{N�he}
\index{H�fte}
\index{oberer}
\index{erh�ht}
\index{Dach}
\index{Spitze}
\index{Himmel}
\begin{supertabular}{p{2,5cm}|ll}
\textbf{\dots anpa} && \textit{Adjektiv}: unten, tief, niedrig \\ % no-dictionary
\textbf{\dots anpa} && \textit{Adverb}: tief, niedrig \\ % no-dictionary
\textbf{anpa} && \textit{Substantiv}: Boden, Erdboden, Grund, Talsohle \\ % no-dictionary
\textbf{anpa} && \textit{Verb, intransitiv}:  sich unterwerfen \\ % no-dictionary
\textbf{anpa (e \dots)} && \textit{Verb, transitiv}: niederlassen, absenken, abseilen, herunterholen, besiegen, bezwingen \\ % no-dictionary
 && \\ % no-dictionary
\textbf{\dots insa} && \textit{Adjektiv}: intern, zentral \\ % no-dictionary
\textbf{insa} && \textit{Substantiv}: Innere, Innenseite, Zentrum, Magen \\ % no-dictionary
 && \\ % no-dictionary
\textbf{\dots monsi} && \textit{Adjektiv}: R�ck-, Hinter- \\ % no-dictionary
\textbf{monsi} && \textit{Substantiv}: R�cken, Heck, Hintern, Po, Arsch \\ % no-dictionary
 && \\ % no-dictionary
\textbf{\dots noka} && \textit{Adjektiv}: Fuss-, niedriger, unten \\  % no-dictionary
\textbf{\dots noka } && \textit{adverb}: zu Fuss \\  % no-dictionary
\textbf{noka} && \textit{Substantiv}: Bein, Fu� \\  % no-dictionary
 && \\ % no-dictionary
\textbf{\dots poka} && \textit{Adjektiv}: angrenzend, benachbart \\ % no-dictionary
\textbf{poka} && \textit{Substantiv}: Seite, H�fte, N�he \\ % no-dictionary
% \textbf{\dots , poka \dots} && \textit{Pr�position}:  in Gesellschaft von, mit, neben \\ % no-dictionary
 && \\ % no-dictionary
\textbf{\dots sewi} && \textit{Adjektiv}: �bergeordnet, oberer, erh�ht, erhaben, religi�s, gl�ubig, formell \\ 
\textbf{\dots sewi} && \textit{Adverb}: �bergeordnet, erh�ht, erhaben, gl�ubig, formell \\ 
\textbf{sewi} && \textit{Substantiv}: H�he, Himmel, Dach, Gipfel, Spitze, Krone \\ 
\textbf{sewi} && \textit{Verb, intransitiv}: aufstehen \\ 
\textbf{sewi (e \dots)} && \textit{Verb, transitiv}: heben, anheben \\ 
 && \\ % no-dictionary
\textbf{\dots sinpin} && \textit{Adjektiv}: Gesichts-, frontal, Vorder-, vertikal \\  % no-dictionary
\textbf{sinpin} && \textit{Substantiv}: Front, Vorderseite, Gesicht, Brust, Brustkorb, Rumpf, Wand, Mauer \\  % no-dictionary
\end{supertabular}
%
%%%%%%%%%%%%%%%%%%%%%%%%%%%%%%%%%%%%%%%%%%%%%%%%%%%%%%%%%%%%%%%%%%%%%%%%%%
\newpage
\index{Substantiv!ortsbezogen}
\index{\textit{anpa}}
\index{\textit{insa}}
\index{\textit{monsi}}
\index{\textit{poka}}
\index{\textit{sewi}}
\index{\textit{noka}}
\index{\textit{poka}}
\index{\textit{sinpin}}
\index{\textit{anpa}!ortsbezogenes Substantiv}
\index{\textit{insa}!ortsbezogenes Substantiv}
\index{\textit{monsi}!ortsbezogenes Substantiv}
\index{\textit{sewi}!ortsbezogenes Substantiv}
\index{\textit{noka}!ortsbezogenes Substantiv}
\index{\textit{poka}!ortsbezogenes Substantiv}
\index{\textit{sinpin}!ortsbezogenes Substantiv}
\index{Substantiv!\textit{anpa}}
\index{Substantiv!\textit{insa}}
\index{Substantiv!\textit{monsi}}
\index{Substantiv!\textit{sewi}}
\index{Substantiv!\textit{noka}}
\index{Substantiv!\textit{poka}}
\index{Substantiv!\textit{sinpin}}
\subsection*{Die ortsbezogene Substantive \textit{anpa}, \textit{insa}, \textit{monsi}, \textit{noka}, \textit{poka}, \textit{sewi} und \textit{sinpin}}
%%%%%%%%%%%%%%%%%%%%%%%%%%%%%%%%%%%%%%%%%%%%%%%%%%%%%%%%%%%%%%%%%%%%%%%%%%
%
M�glicherweise m�chtest man diese W�rter als Pr�positionen benutzen. 
Es sind aber Substantive zur Beschreibung eines Ortes.
Oft werden sie  mit dem intrasitiven Verb \textit{lon} verwendet. 

\begin{supertabular}{p{5,5cm}|ll}
pipi li lon anpa mi.       && Der K�fer ist an meiner Unterseite. \\
moku li lon insa mi.       && Das Essen ist in meinem Inneren. \\
telo suli li lon monsi mi. && Das Meer liegt hinter mir. \\
ma li lon noka mi.         && Land ist unter meinen F��en. \\
mi moku, lon poka sina.    && Ich esse an deiner Seite. \\
ona li lon sewi mi.        && Er ist in dem Bereich �ber mir. \\
tomo li lon sinpin mi.     && Das Haus befindet sich vor mir. \\
\end{supertabular} 
%

%
%%%%%%%%%%%%%%%%%%%%%%%%%%%%%%%%%%%%%%%%%%%%%%%%%%%%%%%%%%%%%%%%%%%%%%%%%%
\subsection*{Weitere Bedeutungen dieser W�rter}

\textit{anpa} kann auch als transitives Verb verwendet werden.

\begin{supertabular}{p{5,5cm}|ll}
mi anpa e jan utala. && Ich besiegte den Soldaten. \\
\end{supertabular} 


\textit{poka} kann als 'normales' Substantiv verwendet werden.

\begin{supertabular}{p{5,5cm}|ll}
poka telo && Strand, Ufer \\
\end{supertabular} 

\textbf{\textit{poka} als Adjektiv} \\
\begin{supertabular}{p{5,5cm}|ll}
jan poka && Nachbar \\
\end{supertabular} 

%
%%%%%%%%%%%%%%%%%%%%%%%%%%%%%%%%%%%%%%%%%%%%%%%%%%%%%%%%%%%%%%%%%%%%%%%%%%
\newpage
\subsection*{�bungen (Antworten siehe Seite~\pageref{'other_prepositions'})}
%%%%%%%%%%%%%%%%%%%%%%%%%%%%%%%%%%%%%%%%%%%%%%%%%%%%%%%%%%%%%%%%%%%%%%%%%%
%
Schreibe bitte die Antworten auf einen Zettel und �berpr�fe sie anschlie�end. 








Versuche diese S�tze zu �bersetzen. 
Mit dem Tool \textit{Toki Pona Parser} (\cite{www:rowa:02}) kann man Rechtschreibung und Grammatik �berpr�fen. 

\begin{supertabular}{p{5,5cm}|ll}
Mein Freund ist an meiner Seite. && \\ % no-dictionary
Die Sonne ist �ber mir. && \\ % no-dictionary
Das Land ist unter mir. && \\ % no-dictionary
B�se Dinge liegen hinter mir. && \\ % no-dictionary
Mir geht es gut weil ich lebe. * && \\ % no-dictionary
Ich schaue mir das Land mit dir an. && \\ % no-dictionary

 && \\ % no-dictionary
poka mi li ' pakala. && \\ % no-dictionary
mi kepeken poki li kepeken ilo moku. && \\ % no-dictionary
jan li lon insa tomo. && \\ % no-dictionary
\end{supertabular} 

* \textit{lon} als Verb alleine bedeutet existieren bzw. real sein.
% 
%%%%%%%%%%%%%%%%%%%%%%%%%%%%%%%%%%%%%%%%%%%%%%%%%%%%%%%%%%%%%%%%%%%%%%%%%%
% eof
