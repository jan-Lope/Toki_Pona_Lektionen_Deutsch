%%%%%%%%%%%%%%%%%%%%%%%%%%%%%%%%%%%%%%%%%%%%%%%%%%%%%%%%%%%%%%%%%%%%%%%%%%
%%%%%%%%%%%%%%%%%%%%%%%%%%%%%%%%%%%%%%%%%%%%%%%%%%%%%%%%%%%%%%%%%%%%%%%%%%
\section{Antworten}
%%%%%%%%%%%%%%%%%%%%%%%%%%%%%%%%%%%%%%%%%%%%%%%%%%%%%%%%%%%%%%%%%%%%%%%%%%
% \subsection*{Introduction} \\
% --
%%%%%%%%%%%%%%%%%%%%%%%%%%%%%%%%%%%%%%%%%%%%%%%%%%%%%%%%%%%%%%%%%%%%%%%%%%
% \subsection*{Pronunciation and the Alphabet} \\
%--

%%%%%%%%%%%%%%%%%%%%%%%%%%%%%%%%%%%%%%%%%%%%%%%%%%%%%%%%%%%%%%%%%%%%%%%%%%
\subsection*{Aussprache, Alphabet und Satzzeichen} 
\label{'pronunciation_alphabet'}

\begin{supertabular}{p{5,5cm}|ll}
Was sind Separatoren? && Separatoren trennen Phrasen voneinander.  \\ % no-dictionary
Welche Phrase hat kein Satzzeichen am Ende? && Eine Überschrift (headline) hat kein Satzzeichen am Ende. \\ % no-dictionary
Welcher Separator steht am Ende eines Aussagesatzes? && Ein Punkt. \\ % no-dictionary
Wann werden offiziellen \textit{toki pona} Wörter groß geschrieben? && Niemals. \\ % no-dictionary
Was darf vor oder nach einem Separator meist nicht stehen? && Ein weiterer Separator. \\ % no-dictionary
\end{supertabular} 

%%%%%%%%%%%%%%%%%%%%%%%%%%%%%%%%%%%%%%%%%%%%%%%%%%%%%%%%%%%%%%%%%%%%%%%%%%
\newpage
%
\subsection*{Einfache Sätze} 
\label{'basic_sentences'}

\begin{supertabular}{p{5,5cm}|ll}
Was ist ein Verb? && Ein Verb beschreibt eine Aktion. \\  % no-dictionary
Was ist ein Substantiv? && Ein Substantiv ist ein Wort für eine Person, Ort oder Ding. \\  % no-dictionary
Wozu dient \textit{li}? && Es trennt die Subjekt-Phrase von der Prädikat-Phrase.  \\ % no-dictionary
Was vertritt ein Personalpronomen? && Es vertritt ein Substantiv. \\ % no-dictionary
Wie erkennt man Substantive, Pronomen, Verben und Adjektive in \textit{toki pona}? && An ihrer Position im Satz. \\  % no-dictionary
Was ist ein Subjekt? && Es ist der Träger der Handlung, des Vorgangs oder des Zustands. \\ % no-dictionary
Nach welchen Subjekt-Phrasen wird \textit{li} nicht benutzt? && Es wird nur benutzt, wenn die Subjekt-Phrase nicht \textit{mi} oder \textit{sina} ist.  \\  % no-dictionary
Wo steht das Subjekt im Satz? && In Toki Pona steht es immer am Anfang des Satzes. \\  % no-dictionary
Kann ein leerer Verb-Slot alleine ein Prädikat bilden? && Nein!  \\  % no-dictionary
Wann kann ein Verb-Slot leer sein? && Wenn das Prädikat durch ein Substantiv oder Adjektiv gebildet wird.  \\  % no-dictionary
Was ist ein Prädikat? && Es ist ein Kernbestandteil in einem Satz und die Ausage des Satzes. \\  % no-dictionary
Ein vollständiger Satz in \textit{toki pona} enthält immer\dots && \dots eine Subjekt- und eine Prädikatsphrase.  \\  % no-dictionary
Aus welchen Wortarten kann in \textit{toki pona} ein Prädikat gebildet werden? && Verben, Substantive oder Adjektive.  \\  % no-dictionary
Was ist ein Adjektiv? && Ein Adjektiv ist ein Wort, das ein Substantiv beschreibt.  \\ % no-dictionary
Wo befinden sich mögliche Adjektiv-Slots? && Nach einem Substantiv, nach einem Pronomen und nach \textit{li}.  \\  % no-dictionary
Warum kann nach \textit{li} ein Satz nicht beendet werden? && Da dann das Prädikat fehlt. \\ % no-dictionary
\end{supertabular} 

\begin{supertabular}{p{5,5cm}|ll}
sina - sina pona. && Personalpronomen \\ % no-dictionary
moku - moku li ' pona. && Substantiv \\ % no-dictionary
ona - ona li ' moku. && Personalpronomen \\ % no-dictionary
li - moku li ' pona. && Separator \\ % no-dictionary
\end{supertabular}

\begin{supertabular}{p{5,5cm}|ll}
Menschen sind gut. && jan li ' pona. \\ % & English - Toki Pona
Ich esse. && mi moku. \\ % & English - Toki Pona
Du bist groß. && sina ' suli. \\ % & English - Toki Pona
Wasser ist einfach. && telo li ' pona. \\ % & English - Toki Pona
Der See ist groß. && telo li ' suli. \\ % & English - Toki Pona
\end{supertabular}  

\begin{supertabular}{p{5,5cm}|ll}
suno li ' suli. && Die Sonne ist groß. \\
mi ' suli. && Ich bin wichtig. / Ich bin fett. \\
jan li moku. && Jemand isst. \\
\end{supertabular} 

%%%%%%%%%%%%%%%%%%%%%%%%%%%%%%%%%%%%%%%%%%%%%%%%%%%%%%%%%%%%%%%%%%%%%%%%%%
\newpage
%
\subsection*{Direkte Objekte} 
\label{'direct_objects_compund_sentences'}

\begin{supertabular}{p{5,5cm}|ll}
Wie erfragt man das direkte Objekt? && Mit 'Wen' oder 'Was'. \\ % no-dictionary
Welche Wortart hat ein Prädikat vor dem Separator \textit{e}? && Es ist immer ein transitives Verb. \\ % no-dictionary
Zu welcher Phrase im Satz gehört ein direktes Objekt? && Zur Prädikat-Phrase. \\ % no-dictionary
Welche Wortarten sind nach dem Separator \textit{e} möglich? && Ein Substantiv oder Pronomen. \\ % no-dictionary
Was ist ein Prädikatsnomen?  && Ein Substantiv als Prädikat. \\ % no-dictionary
Wo befinden sich mögliche Slots für Reflexivpronomen? && Nach dem Separator \textit{e}. \\ % no-dictionary
Kann man mit mehreren \textit{e} mehrere Eigenschaften eines Subjektes beschreiben?  &&  Nein, da \textit{e} nach einem transitives Verb steht. \\ % no-dictionary
Wie kann man mehrere Prädikat-Phrasen in einem Satz bilden? && Mit mehreren Separatoren \textit{li}. \\ % no-dictionary
\end{supertabular}

\begin{supertabular}{p{5,5cm}|ll}
e - mi moku e kili. && Separator \\ % no-dictionary
pona - mi pona e ijo. && transitives Verb \\ % no-dictionary
Das zweite sina - sina telo e sina. &&  Reflexivpronomen  \\ % no-dictionary
ilo - ona li pona e ilo. && Substantiv \\ % no-dictionary
\end{supertabular}

\begin{supertabular}{p{5,5cm}|ll}
Ich habe ein Werkzeug. && mi jo e ilo. \\ % & English - Toki Pona
Sie isst Früchte.  && ona li moku e kili. \\ % & English - Toki Pona
Etwas schaut mich an. && ijo li lukin e mi. \\ % & English - Toki Pona
Ananas sind Nahrung und gut. && kili li ' moku li ' pona. \\ % & English - Toki Pona
Er wäscht sich. && ona li telo e ona.  \\ % & English - Toki Pona
\end{supertabular}  

\begin{supertabular}{p{5,5cm}|ll}
mi ' jan li ' suli. && Ich bin ein Mensch und bedeutend. \\
\end{supertabular} 


%%%%%%%%%%%%%%%%%%%%%%%%%%%%%%%%%%%%%%%%%%%%%%%%%%%%%%%%%%%%%%%%%%%%%%%%%%
\newpage
%
\subsection*{Verben, Adverbien, Hilfsverben} 
\label{'adverbs'}

\begin{supertabular}{p{5,5cm}|ll}
Was sind Adverbien? && Adverbien beschreiben eine Handlung (Verb). \\ % no-dictionary
Kann ein Adverb nach einem Prädikatsnomen stehen? && Nein.  \\ % no-dictionary
Wo befinden sich Slots für Adverbien? 	&&  Nur nach Verben. \\ % no-dictionary
Durch welche Wortart wird eine Handlung beschrieben? && Durch Verben. \\ % no-dictionary
Wann enthält eine Prädikats-Phrase Slots für Adverbien? && Wenn die Prädikats-Phrase ein Verb enthält. \\ % no-dictionary
Wozu dient ein Hilfsverb? && Es ergänzt das Vollverb.  \\ % no-dictionary
Zu welcher Phrase im Satz gehört ein Hilfsverb? && Ein Hilfsverb gehört zur Prädikat-Phrase.  \\ % no-dictionary
\end{supertabular} 

\begin{supertabular}{p{5,5cm}|ll}
kama - mi kama jo e telo. && Hilfsverb \\ % no-dictionary
wile - mi wile lukin e ma. && Hilfsverb, transitives Verb \\ % no-dictionary
ike - mi lawa ike e jan. && Adverb \\ % no-dictionary
jan - mi ' jan.  && Adjectiv, Substantiv \\ % no-dictionary
\end{supertabular} 


\begin{supertabular}{p{5,5cm}|ll}
jan li pona ilo e ilo. && Der Kerl verbessert nutzbringend das Werkzeug. \\ % & English - Toki Pona
sina lukin unpa mute e mi. && Du betrachtest mich sehr sexy.  \\ % & English - Toki Pona
jaki li jaki lili e mi. && Der Müll beschmutze mich ein wenig. \\ % & English - Toki Pona
sina len nasa jaki e sina. && Du kleidest dich ekelhaft albern. \\ % & English - Toki Pona
ilo li sewi sewi e sewi. && Die Maschine hebt das Dach hoch. \\ % & English - Toki Pona
ona li lawa utala e utala. && Er führt kämpfend die Schlacht. \\ % & English - Toki Pona
mi wile unpa e ona. && Ich möchte Sex mit ihr haben.  \\ % & English - Toki Pona
jan li wile jo e ma. && Die Leute möchten Land besitzen. \\ % & English - Toki Pona
\end{supertabular}  

\begin{supertabular}{p{5,5cm}|ll}
Sie vermehrt sehr schlecht den Besitz. && ona li mute ike mute e jo. \\
Ich will viel Sex mit dir haben. && mi wile unpa mute e sina. \\
Sie kleidete sich wenig. && ona li len lili e ona. \\
Die Sonne bestrahlt warm das Land. && suno li suno seli e ma. \\
Sie ist gut. && ona li ' pona. \\
Er will das Werkzeug zerstören. && ona li wile pakala e ilo. \\ % & English - Toki Pona
Sie hat Durst. && ona li wile moku e telo. \\ % & English - Toki Pona
\end{supertabular} 

%%%%%%%%%%%%%%%%%%%%%%%%%%%%%%%%%%%%%%%%%%%%%%%%%%%%%%%%%%%%%%%%%%%%%%%%%%
\newpage
%
\subsection*{Substantive, Adjektive, Pronomen} 
\label{'adjectives'}

\begin{supertabular}{p{5,5cm}|ll}
Was vertritt ein Possessivpronomen? && Es vertritt ein Adjektiv. \\ % no-dictionary
Welche Arten von Demonstrativpronomen gibt es? && Adjektivische und substantivische.  \\ % no-dictionary
Was ist komplexer in \textit{toki pona}: Adjektive oder Adverbien? && Adjektive.  \\ % no-dictionary
Mit welcher Wortart werden Substantive beschrieben? && Mit Adjektiven.  \\ % no-dictionary
Was ist der Unterschied zwischen Adverbien und Adjektiven? && Adverbien beschreiben Verben - Adjektive beschreiben Substantive. \\ % no-dictionary
Wo befinden sich Slots für Adjektive? && Nur nach Substantiven oder als Prädikatsadjektiv. \\ % no-dictionary
Kann ein Adjektiv nach einem Prädikatsnomen stehen? && Ja, da ein Prädikatsnomen ein Substantiv ist.  \\ % no-dictionary
\end{supertabular}

\begin{supertabular}{p{5,5cm}|ll}
mi jo e kili.  &&  Ich habe Früchte. \\
ona li ' pona li ' lili. && Sie sind gut und klein. \\
mi moku lili e kili lili.  && Ich nasche ein wenig von den kleinen Früchten. \\
\end{supertabular}  

\begin{supertabular}{p{5,5cm}|ll}
Der Chef trank dreckiges Wasser. && jan lawa li moku e telo jaki. \\ % & English - Toki Pona
Ich brauche eine Gabel. && mi wile e ilo moku. \\ % & English - Toki Pona
Ein Feind greift sie an.  && jan ike li utala e ona mute. \\ % & English - Toki Pona
Diese schlechte Person hat eigenartige Kleidung.  && jan ike ni li jo e len nasa. \\ % & English - Toki Pona
Wir tranken viel Wodka.  && mi mute li moku e telo nasa mute. \\ % & English - Toki Pona
Die Kinder beobachten die Erwachsenen.  && jan lili li lukin e jan suli. \\ % & English - Toki Pona
\end{supertabular}  

\begin{supertabular}{p{5,5cm}|ll}
mi lukin e ni. && Ich betrachte es. \\
mi lukin sewi e tomo suli.  && Ich schaue hoch zum Wolkenkratzer. \\
seli suno li seli e tomo mi.  && Die Sonnenhitze wärmt mein Haus. \\
jan lili li wile e telo kili.  && Die Kinder wollen Fruchtsaft. \\
ona mute li nasa e jan suli.  && Sie nerven die Eltern. \\
mi kama e pakala. && Ich verursachte einen Unfall. \\
\end{supertabular} 

%%%%%%%%%%%%%%%%%%%%%%%%%%%%%%%%%%%%%%%%%%%%%%%%%%%%%%%%%%%%%%%%%%%%%%%%%%
\newpage
%
\subsection*{Indirekte Objekte} 
\label{'indirect_objects'}

\begin{supertabular}{p{5,5cm}|ll}
Wie kann man nicht ein indirektes Objekt erfragen?  && Man kann es nicht mit 'Wen' oder 'Was' erfragen. \\ % no-dictionary
Welche Objektart wird stark vom Prädikat beeinflußt? && Das direkte Objekt.  \\ % no-dictionary
Zu welcher Phrase im Satz gehört das indirekte Objekt? && Zur Prädikat-Phrase. \\ % no-dictionary
Was für ein Slot steht an erster Position in einem indirekten Objekt? && Ein Substantiv- oder Pronomen-Slot. \\ % no-dictionary
Wie nennt man Verben, die kein Objekt beeinflussen? && Es sind intransitive Verben.  \\ % no-dictionary
Was steht vor einem indirekten Objekt in \textit{toki pona}? && Ein intransitives Verb. \\ % no-dictionary
Wo ist ein Slot für ein adjektivisches Demonstrativpronomen möglich? && Nach einem Substantiv. \\  % no-dictionary
Wo ist ein Slot für ein Hilfsverb? && Ein Hilfsverb steht vor dem Vollverb. \\ % no-dictionary
\end{supertabular}

\begin{supertabular}{p{5,5cm}|ll}
Dies ist für meinen Freund.   && ni li tawa jan pona mi. \\ % & English - Toki Pona
Die Werkzeuge sind im Container.  && ilo li lon poki. \\ % & English - Toki Pona
Diese Flasche ist im Dreck.  && poki ni li lon jaki. \\ % & English - Toki Pona
Sie streiten.  && ona mute li utala toki. \\ % & English - Toki Pona
Die Frau gebahr ihr Kind. && meli li lon e jan lili ona. \\ % & English - Toki Pona
\end{supertabular} 

%%%%%%%%%%%%%%%%%%%%%%%%%%%%%%%%%%%%%%%%%%%%%%%%%%%%%%%%%%%%%%%%%%%%%%%%%%
\newpage
%
\subsection*{Präpositionalobjekte} 
\label{'prepositional_objects'}

\begin{supertabular}{p{5,5cm}|ll}
Mit was ist eine Präposition eng verbunden? && Eine Präposition ist eng mit dem Verb verbunden. \\ % no-dictionary
Zu welcher Phrase im Satz gehört das Präpositionalobjekt? && Es ist optionaler Bestandteil einer Prädikat-Phrase.  \\ % no-dictionary
Wo befinden sich Slots für Präpositionen? && Am Anfang eines Präpositionalobjektes. \\ % no-dictionary
An welcher Position im Satz kann ein Präpositionalobjekt stehen? && Am Ende eines Satzes. \\ % no-dictionary
Mit welchen Separatoren kann man zusammengesetzte Sätze bilden? && Mit den Separatoren \textit{li} und \textit{e}. \\ % no-dictionary
Welche Slots sind an zweiter Stelle im Präpositionalobjekt möglich? && Ein Substantiv- oder Pronomen-Slot. \\ % no-dictionary
\end{supertabular}

\begin{supertabular}{p{5,5cm}|ll}
Ich reparierte die Taschenlampe mit einem kleinen Werkzeug.  && mi pona e ilo suno, kepeken ilo lili. \\ % & English - Toki Pona
Ich mag \textit{toki pona}.  && toki pona li ' pona, tawa mi. \\ % & English - Toki Pona
Wir gaben ihnen Essen.  && mi mute li pana e moku, tawa ona mute. \\ % & English - Toki Pona
Ich möchte mit meinem Auto zu seinem Haus fahren.  && mi wile tawa tomo ona, kepeken tomo tawa mi. \\ % & English - Toki Pona
Leute sehen wie Ameisen aus.  && jan li lukin, sama pipi. \\ % & English - Toki Pona
\end{supertabular}  

\begin{supertabular}{p{5,5cm}|ll}
sina wile kama, tawa tomo toki.  && Du solltest in den Chat-Room kommen. \\
jan li toki, kepeken toki pona, lon tomo toki.  && Leute sprechen \textit{toki pona} in dem Chat-Room. \\
mi tawa, tawa tomo toki. ona li pona, tawa mi.  && Ich gehe in den Chat-Room. Es gefällt mir. \\
sina kama jo e jan pona, lon ni.  && Du bekommst Freunde dort. \\
sama li ' ike. && Gleichheit ist schlecht. \\
mi sona e tan. && Ich kenne die Ursache. \\
\end{supertabular} 

%%%%%%%%%%%%%%%%%%%%%%%%%%%%%%%%%%%%%%%%%%%%%%%%%%%%%%%%%%%%%%%%%%%%%%%%%%
\newpage
%
\subsection*{Relative Ortsangaben} 
\label{'other_prepositions'}

\begin{supertabular}{p{5,5cm}|ll}
Wie bildet man relative Ortsangaben in \textit{toki pona}? && Indirekten Verb/Präposition + ortsbezogenen Substantiv. \\ % no-dictionary
Was ist ein Possessivpronomen? && Es drückt eine Eigenschaft oder Zugehörigkeit aus.  \\ % no-dictionary
Wo ist ein Slot für ein substantivisches Demonstrativpronomen möglich? && An Stelle eines Substantivs. \\ % no-dictionary
Welcher Separator steht am Ende eines Aussagesatzes? && Ein Punkt. \\ % no-dictionary
Was ist ein Prädikatsadjektiv? && Ein Adjektiv, dass als Prädikat verwendet wird. \\ % no-dictionary
In welchen Satzphrasen können sich ortsbezogenen Substantive befinden? && In einem indirekten Objekt oder Präpositioanlobjekt.  \\ % no-dictionary
\end{supertabular}

\begin{supertabular}{p{5,5cm}|ll}
Mein Freund ist an meiner Seite.  && jan pona mi li lon poka mi. \\ % & English - Toki Pona
Die Sonne ist über mir.  && suno li lon sewi mi. \\ % & English - Toki Pona
Das Land ist unter mir.  && ma li lon anpa mi. \\ % & English - Toki Pona
Böse Dinge liegen hinter mir.  && ijo ike li lon monsi mi. \\ % & English - Toki Pona
Mir geht es gut weil ich lebe.  && mi ' pona, tan ni: mi lon. \\ % & English - Toki Pona
Ich schaue mir das Land mit dir an.  && mi lukin e ma, lon poka sina. \\ % & English - Toki Pona
\end{supertabular}  

\begin{supertabular}{p{5,5cm}|ll}
poka mi li ' pakala.  && Meine Hüften schmerzen. \\
mi kepeken poki li kepeken ilo moku.  && Ich verwende eine Schüssel und einen Löffel.\\
jan li lon insa tomo.  && Jemand ist im Haus. \\
\end{supertabular} 


%%%%%%%%%%%%%%%%%%%%%%%%%%%%%%%%%%%%%%%%%%%%%%%%%%%%%%%%%%%%%%%%%%%%%%%%%%
\newpage
%
\subsection*{Verneinung, Ja/Nein-Fragen} 
\label{'negation_yes_no_questions'}

\begin{supertabular}{p{5,5cm}|ll}
Welcher Separator steht am Ende einer Frage? && Ein fragezeichen. \\ % no-dictionary
Wie wird in \textit{toki pona} eine Ja/Nein-Frage formuliert? && Das Adverb \textit{ala} wird an das Verb angefügt und das Verb wiederholt.  \\ % no-dictionary
Was ist bei einem Prädikat ohne Verb zu beachten? && Es lassen sich keine Ja/Nein-Frage mit dem Adverb \textit{ala} formulieren. \\ % no-dictionary
Wie wird in \textit{toki pona} ein Verb negiert? && Indem das Adverb \textit{ala} hinter das Verb gesetzt wird.  \\ % no-dictionary
Wie antwortet man in \textit{toki pona} negativ auf eine Ja/Nein-Frage? && Man wiederholt das Verb der Frage und fügt das Adverb \textit{ala} an. \\ % no-dictionary
Wie antwortet man in \textit{toki pona} positiv auf eine Ja/Nein-Frage? && Man wiederholt das Verb der Frage. \\ % no-dictionary
\end{supertabular}

\begin{supertabular}{p{5,5cm}|ll}
Du mußt mir sagen warum!  && sina wile toki e tan, tawa mi. \\ % & English - Toki Pona
Ist ein Käfer neben mir?  && pipi li lon ala lon poka mi? \\ % & English - Toki Pona
Ich kann nicht schlafen.  && mi ken ala lape. \\ % & English - Toki Pona
Ich möchte nicht mit dir reden.  && mi wile ala toki, tawa sina. \\ % & English - Toki Pona
Er ging nicht zum See.  && ona li tawa ala, tawa telo. \\ % & English - Toki Pona
\end{supertabular}  

\begin{supertabular}{p{5,5cm}|ll}
sina wile ala wile pali? wile ala.  && Willst du arbeiten? Nein. \\
jan utala li seli ala seli e tomo?  && Brannten die Krieger das Haus nieder? \\
jan lili li ken ala moku e telo nasa.  && Kindern dürfen kein Bier trinken. \\
sina kepeken ala kepeken ni?  && Hast du dies hier verwendet? \\
sina ken ala ken kama?  && Kannst du kommen? \\
sina pona ala pona? && Reparierst du (etwas)? \\
\end{supertabular} 


%%%%%%%%%%%%%%%%%%%%%%%%%%%%%%%%%%%%%%%%%%%%%%%%%%%%%%%%%%%%%%%%%%%%%%%%%%
\newpage
%
\subsection*{Inoffizielle Wörter} 
\label{'unofficial_words'}

\begin{supertabular}{p{5,5cm}|ll}
Was sind Eigennamen in \textit{toki pona}? && Inoffizielle Wörter, Adjektive \\ % no-dictionary
Wo befinden sich Slots für Prädikatsadjektive? && Nach dem Separator \textit{li}. \\ % no-dictionary
Wie werden Namen in \textit{toki pona} hervorgehoben? && Der erste Buchstabe ist ein Großbuchstabe. \\ % no-dictionary
Wie wird die Originalschreibweise eines Namens gekennzeichnet? && Durch Anführungszeichen.  \\ % no-dictionary
Welche Slots können inoffizielle Wörter füllen? && Adjektiv-Slots.  \\ % no-dictionary
Mit welcher Wort-Art müssen inoffizielle Wörter zusammen verwendet werden? && Mit einem Substantiv. \\ % no-dictionary
\end{supertabular}

\begin{supertabular}{p{5,5cm}|ll}
Susan ist verrückt.   && jan Susan li ' nasa. \\ % & English - Toki Pona
Ich komme aus Europa.  && mi kama, tan ma suli Elopa. \\ % & English - Toki Pona
Mein Name ist Ken.  && mi ' jan Ken. / nimi mi li Ken. \\ % & English - Toki Pona
Hallo Lisa!  && jan Lisa o, toki! \\ % & English - Toki Pona
Ich möchte nach Australien gehen.  && mi wile tawa, tawa ma suli Oselija. \\  % & English - Toki Pona
\end{supertabular}  

\begin{supertabular}{p{5,5cm}|ll}
mi wile kama sona e toki Inli.  && Ich möchte Englisch lernen. \\
jan Ana o pana e moku, tawa mi!  && Ana, gib mir Essen! \\
jan Mose o lawa e mi mute, tawa ma pona!  && Moses führe uns in das gelobte Land! \\
\end{supertabular} 


%%%%%%%%%%%%%%%%%%%%%%%%%%%%%%%%%%%%%%%%%%%%%%%%%%%%%%%%%%%%%%%%%%%%%%%%%%
\newpage
%
\subsection*{Vokativ, Ausrufe, Befehle} 
\label{'commands_interjections'}

\begin{supertabular}{p{5,5cm}|ll}
Mit welchem Separator endet ein Befehlssatz (Imperativ)? && Mit einem Ausrufungszeichen. \\ % no-dictionary
Woraus besteht das Subjekt bei der Befehlsform wenn niemand direkt angesprochen wird? && Aus dem Interjektion-Wort \textit{o}. \\ % no-dictionary
Wie spricht man Leute mit ihrem Namen an? && \textit{jan Name o, ...} \\ % no-dictionary
Woraus bestehen oft Interjektionen (Ausrufe)?? && Substantiv / Interjektion-Wort + Ausrufungszeichen. \\ % no-dictionary
Welcher Separator leitet das Prädikat ein, wenn bei einem Befehl jemand direkt angesprochen wird? && Der Separator \textit{o}. \\ % no-dictionary
Mit welchem Separator endet eine Interjektion (Ausruf)? && Mit einem Ausrufungszeichen. \\ % no-dictionary
\end{supertabular}

\begin{supertabular}{p{5,5cm}|ll}
Geh!  && o tawa! \\ % & English - Toki Pona
Warte Mutti!  && mama meli o awen! \\ % & English - Toki Pona
Hahaha! Das ist lustig.  && a a a! ni li ' musi. \\ % & English - Toki Pona
Scheiße!  && pakala! \\ % & English - Toki Pona
Tschüss!  && mi tawa!  \\ % & English - Toki Pona
\end{supertabular}  

\begin{supertabular}{p{5,5cm}|ll}
mu!  && Muh! Wau! Miau! \\
o tawa musi, lon poka mi!  && Tanze mit mir! \\
tawa pona!  && Tschüß!. (Das sagt der, der geht.) \\
o pu! && Kaufe und lese das offiziellen Toki Pona Buch! \\
\end{supertabular} 

%%%%%%%%%%%%%%%%%%%%%%%%%%%%%%%%%%%%%%%%%%%%%%%%%%%%%%%%%%%%%%%%%%%%%%%%%%
\newpage
%
\subsection*{Fragen} 
\label{'questions_using_seme'}

\begin{supertabular}{p{5,5cm}|ll}
Wie ändert sich der Satzaufbau bei einer Frage in \textit{toki pona}? && Der Satzaufbau ändert sich nicht. \\ % no-dictionary
Welche Wortart hat das Wort \textit{seme}? && Es ist ein Fragepronomen.  \\ % no-dictionary
Was ist ein Reflexivpronomen? && Ein Reflexivpronomen repräsentiert das Subjekt im direkten Objekt. \\ % no-dictionary
Was kann das Wort \textit{seme} vertreten? && Satzteile beziwhuzúngsweise alle Wortarten (außer Separatoren).  \\ % no-dictionary
Wie wird nach einer Person gefragt (Wer, Wen, Wem)? && Mit dem Substantiv \textit{jan} und \textit{seme}. \\ % no-dictionary
Wie wird eine Warum-Frage gestellt? && Mit der Präposition \textit{tan} und \textit{seme} als Präpositionalobjekt. \\ % no-dictionary
Wie wird nach einem indirekten Objekt gefragt? && Wenn \textit{seme} nach einem intransitiven Verb folgt. \\ % no-dictionary
Wie wird nach einem Präpositionalobjekt gefragt? && Wenn \textit{seme} nach einer Präposition folgt. \\ % no-dictionary
Gibt es verschachtelte Nebensätze in \textit{toki pona}? &&  Nein. \\ % no-dictionary
\end{supertabular}

\begin{supertabular}{p{5,5cm}|ll}
Was möchtest du tun?  && sina wile pali e seme? \\ % & English - Toki Pona
Wer liebt dich?  && jan seme li olin e sina? \\ % & English - Toki Pona
Versüsst es? && ni li suwi ala suwi? \\ % & English - Toki Pona
Ich gehe ins Bett.  && mi tawa supa lape. \\ % & English - Toki Pona
Kommen noch mehr Leute?  && jan sin li kama ala kama? \\ % & English - Toki Pona
Gib mir einen Lutscher!  && o pana e suwi, tawa mi! \\ % & English - Toki Pona
Wer ist da?  && jan seme li lon? / jan seme li lon ni? \\ % & English - Toki Pona
Welches Insekt hat dir weh getan?  && pipi seme li pakala e sina? \\ % & English - Toki Pona
Er liebt es zu essen.  && moku li pona, tawa ona. \\ % & English - Toki Pona
Wie bitte? && seme?  \\ % & English - Toki Pona
\end{supertabular}  

\begin{supertabular}{p{5,5cm}|ll}
jan Ken o, mi olin e sina.  && Ken, ich liebe dich. \\
ni li ' jan seme?  && Wer ist das? \\
sina lon seme?  && Wo bist du? \\ 
mi lon, tan seme?  && Warum bin ich hier? \\ 
jan seme li ' meli sina?  && Wer ist deine Freundin/Frau? \\
sina tawa ma tomo, tan seme?  && Warum gingst du in die Stadt? \\
sina wile tawa, tawa  ma seme?  && In welches Land willst du gehen? \\
\end{supertabular} 

%%%%%%%%%%%%%%%%%%%%%%%%%%%%%%%%%%%%%%%%%%%%%%%%%%%%%%%%%%%%%%%%%%%%%%%%%%
\newpage
%
\subsection*{Zusammengesetzte Substantive} 
\label{'pi'}

\begin{supertabular}{p{5,5cm}|ll}
Kann der Separator \textit{pi} zum Trennen von Adjektiven verwendet werden? && Nein. \\ % no-dictionary
Wo steht in \textit{toki pona} bei einem zusammengesetzen Substantiv das Haupt-Substantiv? && Am Anfang. \\ % no-dictionary
Wieviel Wörter müssen mindenstens zwischen dem Separator \textit{pi} und dem nächsten Separator stehen? && Zwei Wörter. \\ % no-dictionary
Wo können sich Adjektiv-Slots nach dem Separator \textit{pi} befinden? && Ab der zweiten Stelle nach Separator \textit{pi}. \\ % no-dictionary
Wie fragt man nach dem Eigentümer eines Gegenstandes? && Gegenstand + \textit{pi} + \textit{jan} + \textit{seme}  \\ % no-dictionary
\end{supertabular}

\begin{supertabular}{p{5,5cm}|ll}
Kelis Kind ist lustig.  && jan lili pi jan Keli li ' musi. \\ % & English - Toki Pona
Ich bin ein Toki-Ponianer.  && mi ' jan pi toki pona. \\ % & English - Toki Pona
Er ist ein guter Musiker.  && ona li ' jan pona pi kalama musi. \\ % & English - Toki Pona
Der Kapitän des Schiffes ißt.  && jan lawa pi tomo tawa telo li moku. \\ % & English - Toki Pona
Määh (Schaf).  && mu! \\ % & English - Toki Pona
Enyas Musik ist gut.  && kalama musi pi jan Enja li ' pona. \\ % & English - Toki Pona
Welche Leute dieser Gruppe sind wichtig?  && jan seme pi kulupu ni li suli? \\ % & English - Toki Pona
Unser Haus ist verwüstet.  && tomo pi mi mute li ' pakala. \\ % & English - Toki Pona
Wie hat sie das gemacht?   && ona li pali e ni, kepeken nasin seme? \\ % & English - Toki Pona
Ich schaue mir das Land mit meinem Freund an.  && mi lukin e ma, lon poka pi jan pona mi. \\ % & English - Toki Pona
Mit wem bist du gegangen?  && sina tawa, lon poka pi jan seme? \\ % & English - Toki Pona
\end{supertabular}  

\begin{supertabular}{p{5,5cm}|ll}
pipi pi ma mama mi li ' lili.  && Die Käfer meines Heimatlandes sind klein.\\ 
kili pi jan Linta li ' ike.  && Lindas Früchte sind schlecht. \\
len pi jan Susan li ' jaki.  && Susans Kleider sind dreckig. \\
mi sona ala e nimi pi ona mute.  && Ich kenne nicht ihre Namen. \\
mi wile toki meli.  && Ich möchte über Mädchen sprechen. \\
sina pakala e ilo, kepeken nasin seme?  && Wie hast du das Werkzeug zerbrochen? \\
jan Wasintan [Washington] li ' jan lawa pona pi ma Mewika.  && Washington war ein guter Präsident der USA. \\
wile pi jan ike li pakala e ijo.  && Die Wünsche schlechter Menschen zerstören Dinge. \\
\end{supertabular}  

%%%%%%%%%%%%%%%%%%%%%%%%%%%%%%%%%%%%%%%%%%%%%%%%%%%%%%%%%%%%%%%%%%%%%%%%%%
\newpage
%
\subsection*{Konjunktionen und Temperatur} 
\label{'conjunctions_temperature'}

\begin{supertabular}{p{5,5cm}|ll}
Was sind Konjunktionen? && Konjunktionen verbinden Wörter und Phrasen. \\ % no-dictionary
Was ist eine Antwortfrage? && Dabei ist die Antwort ist bereits in der Frage enthalten. \\ % no-dictionary
Was ist der Unterschied zwischen Konjunktionen und Präpositionen? && Konjunktionen bewirken keine Fälle. \\ % no-dictionary
Wie wird in \textit{toki pona} eine Antwortfrage gebildet? && Die Konjunktion \textit{anu} und das Fragepronomen \textit{seme} wird anfügt. \\ % no-dictionary
Kommt vor oder nach der Konjunktion \textit{taso} ein Komma? && Nein. \\ % no-dictionary
Was sind Alternativfragen? && Es wird nach einer Auswahl von mehreren Möglichkeiten gefragt.  \\ % no-dictionary
Was verbindet die Konjunktion \textit{taso}? && Damit bezieht man sich auf den vorherigen Satz.  \\ % no-dictionary
Was verbindet die Konjunktion \textit{en}? && Es verbindet (zusammengesetzte) Substantive bzw. Pronomen. \\ % no-dictionary
Wie wird in \textit{toki pona} eine Alternativfrage gebildet? && Mit der Konjunktion \textit{anu}. \\ % no-dictionary
Wie wird in \textit{toki pona} eine Ja/Nein-Frage mit Prädikatsnomen oder Prädikatsadjektive gebildet? && Es wird eine Antwortfrage formuliert. \\ % no-dictionary
\end{supertabular}

\begin{supertabular}{p{5,5cm}|ll}
Möchtest du kommen oder was?  && sina wile kama anu seme? \\ % & English - Toki Pona
Willst du das Essen oder das Wasser?  && sina wile e moku anu telo? \\ % & English - Toki Pona
Ich will noch in mein Haus gehen.  && mi wile kin tawa, tawa tomo mi. \\ % & English - Toki Pona
Dieses Papier fühlt sich kalt an.  && lipu ni li ' lete, tawa mi. \\ % & English - Toki Pona
Ich mag die Währungen anderer Länder.  && mani pi ma ante li ' pona, tawa mi. \\ % & English - Toki Pona
Ich möchte gehen, aber ich kann nicht.  && mi wile tawa. taso mi ken ala. \\ % & English - Toki Pona
Nur ich bin da. && mi taso li lon. \\ % & English - Toki Pona
Magst du mich?  && mi ' pona, tawa sina anu seme? \\ % & English - Toki Pona
Dieser See ist kalt. && telo ni li ' lete, tawa mi. \\ % & English - Toki Pona
\end{supertabular}  

\begin{supertabular}{p{5,5cm}|ll}
mi olin kin e sina.  && Ich liebe dich auch / immer noch.\\
mi pilin e ni: ona li jo ala e mani.  && Ich denke, dass er kein Geld hat. \\
mi wile lukin e ma ante.  && Ich möchte andere Länder sehen. \\
mi wile ala e ijo. mi lukin taso.  && Ich möchte nichts. Ich schaue nur. \\
mi pilin lete.  && Ich friere. \\
sina wile toki, tawa mije anu meli?  && Möchtest du zu einem Mann oder einer Frau sprechen? \\
\end{supertabular} 

%%%%%%%%%%%%%%%%%%%%%%%%%%%%%%%%%%%%%%%%%%%%%%%%%%%%%%%%%%%%%%%%%%%%%%%%%%
\newpage
%
\subsection*{Farben} 
\label{'colors'}

\begin{supertabular}{p{5,5cm}|ll}
Der Slot nach der Konjunktion \textit{en} kann welche Worart(en) beinhalten? && Substantiv oder Pronomen.  \\ % no-dictionary
Wie werden Farbmuster eines Gegenstandes in \textit{toki pona} beschrieben? && Gegenstand + \textit{pi} + 1.Farbe + \textit{en} + 2.Farbe \dots \\ % no-dictionary
Wie werden Farbtöne beschrieben, für die es kein Wort in \textit{toki pona} gibt? && Durch mehrere Wörter. \\ % no-dictionary
Der Slot nach dem Separator \textit{pi} kann welche Worart(en) beinhalten? && Substantiv oder Pronomen.  \\ % no-dictionary
Welche Wortarten haben die Wörter für Farben in \textit{toki pona}? && Adjektive und Substantive. \\ % no-dictionary
\end{supertabular}

\begin{supertabular}{p{5,5cm}|ll}
Ich sehe die blaue Tasche nicht.  && mi lukin ala e poki laso. \\ % & English - Toki Pona
Kleine grüne Menschen kamen vom Himmel.  && jan laso jelo lili li kama, tan sewi.  \\ % & English - Toki Pona
Ich mag die Farbe Lila.  && kule loje laso li ' pona, tawa mi. / \\ % & English - Toki Pona
Der Himmel ist blau.  && sewi li ' laso. \\ % & English - Toki Pona
Siehe diesen roten Käfer.  && o lukin e pipi loje ni.  \\ % & English - Toki Pona
Ich brauche die Landkarte.  && mi wile e sitelen ma. \\ % & English - Toki Pona
Schaust du dir 'Akte-X' an?   && sina lukin ala lukin e sitelen tawa X-Files? \\ % & English - Toki Pona
Welche Farbe magst du?  && kule seme li ' pona, tawa sina? \\ % & English - Toki Pona
Ist es rot? && ona li ' loje anu seme? \\ % & English - Toki Pona
\end{supertabular}  

\begin{supertabular}{p{5,5cm}|ll}
ni li pimeja ala pimeja e suno? && Verdunkelt dies die Sonne? \\
suno li ' jelo.  && Die Sonne ist gelb. \\
telo suli li ' laso.  && Das Meer ist blau. \\
mi wile moku e kili loje.  && Ich möchte eine rote Frucht essen. \\
ona li kule e tomo tawa.  && Er lakiert das Auto. \\
len pi loje en laso pi meli sina li ' pona, tawa mi. && Das rot und blau gemusterte Kleid deiner Frau geällt mir. \\
\end{supertabular}  

\begin{supertabular}{p{5,5cm}|ll}
ma mi li ' pimeja. && Mein Land ist dunkel. \\
kalama ala li lon && Kein Geräsch ist da.\\
mi lape. mi sona. && Ich schlafe. Ich weiss. \\
\end{supertabular} 

%%%%%%%%%%%%%%%%%%%%%%%%%%%%%%%%%%%%%%%%%%%%%%%%%%%%%%%%%%%%%%%%%%%%%%%%%%
\newpage
%
\subsection*{Lebewesen} 
\label{'living_things'}

\begin{supertabular}{p{5,5cm}|ll}
Welches Trennzeichen steht am Ende einer Frage? && Ein Fragezeichen. \\ % no-dictionary
In welchen Fällen wird ein Komma verwendet? && Leute anzusprechen: nach \textit{o}. Optional vor Präpositionen. \\ % no-dictionary
In welchen Fällen wird ein Doppelpunkt verwendet? && Ein Doppelpunkt trennt ein Hinweis-Satz von und einem Satz. \\ % no-dictionary
Wo sind mögliche Slots für Präpositions in einem Satz? && Am Anfang eines Präpositionalobjektes.  \\ % no-dictionary
\end{supertabular}

\begin{supertabular}{p{5,5cm}|ll}
Ist dies ein Säugetier? && ni li ' soweli anu seme?  \\ % & English - Toki Pona
Ich möchte einen Hamster haben.  && mi wile e soweli lili. \\ % & English - Toki Pona
Oh! Der Dinosaurier will mich fressen.  && a! akesi li wile moku e mi! \\ % & English - Toki Pona
Ein Moskito hat mich gestochen.  && pipi li moku e mi.  \\ % & English - Toki Pona
Kühe sagen Muuh.  && soweli li toki e mu. \\ % & English - Toki Pona
Vögel fliegen in der Luft.  && waso li tawa, lon kon. \\ % & English - Toki Pona
Lass uns Fisch essen.  && mi mute o moku e kala. \\ % & English - Toki Pona
Blumen sind schön.   && kasi kule li ' pona lukin. \\ % & English - Toki Pona
Ich mag Blumen.  && kasi li ' pona, tawa mi. \\ % & English - Toki Pona
Hast du dich verbessert? && sina pona ala pona e sina? sina pona e sina anu seme? \\ % & English - Toki Pona
\end{supertabular}  

\begin{supertabular}{p{5,5cm}|ll}
mama ona li kepeken kasi nasa.  && Seine Mutter verwendete Marihuana. \\
akesi li pana e telo moli.  && Die Schlange gab Gift. \\
pipi li moku e kasi.  && Käfer fressen Pflanzen. \\
soweli mi li kama moli.  && Mein Hund stirbt. \\
jan Pawe o, mi wile ala moli.  && Paul, ich möchte nicht sterben. \\
mi lon ma kasi.  && Ich bin im Wald. \\
ona li kasi ala kasi? && Wächst es? \\
\end{supertabular} 

%%%%%%%%%%%%%%%%%%%%%%%%%%%%%%%%%%%%%%%%%%%%%%%%%%%%%%%%%%%%%%%%%%%%%%%%%%
\newpage
%
\subsection*{Der Körper} 
\label{'the_body'}

\begin{supertabular}{p{5,5cm}|ll}
kepeken - mi kepeken ilo. && intranstives Verb, Substantiv \\ % no-dictionary
sina - sina pona ala pona? && transitives Verb \\ % no-dictionary
kama - mi kama jo e tomo tawa. && Hilfsverb \\ % no-dictionary
lon - mi lon tomo. && intranstives Verb, Adverb, Adjektiv, Substantiv \\ % no-dictionary
kepeken - mi pali e ni, kepeken ilo. && Präposition  \\ % no-dictionary
\end{supertabular}

\begin{supertabular}{p{5,5cm}|ll}
Küss mich.  && o pilin e uta mi, kepeken uta sina. \\ % & English - Toki Pona
Ich muß pullern.  && mi wile pana e telo jelo. \\ % & English - Toki Pona
Mein Haar ist naß.  && linja mi li ' telo. \\ % & English - Toki Pona
Etwas ist in meinem Auge.  && ijo li lon oko mi. \\ % & English - Toki Pona
Ich kann nicht hören was du sagst.  && mi ken ala kute e toki sina. \\ % & English - Toki Pona
Ich muß kacken.  && mi wile pana e ko jaki. \\ % & English - Toki Pona
Das Loch ist groß.  && lupa ni li ' suli. \\ % & English - Toki Pona
Ist es eine Kette? && ona li ' linja anu seme? \\ % & English - Toki Pona
\end{supertabular}  

\begin{supertabular}{p{5,5cm}|ll}
selo pi jelo en laso pi akesi lili li ' pona, tawa mi. && Die grün-blau gescheckte Haut der kleinen Echse gefällt mir. \\ % & English - Toki Pona
a! telo sijelo loje li kama tan nena kute mi!  && Ahh! Blut läuft aus meinem Ohr! \\
selo mi li wile e ni: mi pilin e ona.  && Meine Haut juckt. \\
%%    && Anders ausgerückt: Meine Haut juckt.This is how we say that our skin itches. \\  % no-dictionary
o pilin e nena.  && Drücke den Knopf. \\
o moli e pipi, kepeken palisa.  && Töte den Skorpion mit dem Stock. \\
luka mi li ' jaki. mi wile telo e ona.  && Meine Hände sind dreckig. Ich möchte sie waschen. \\
o pana e sike, tawa mi.  && Gib mir den Ball. \\
mi pilin e seli sijelo sina.  && Ich fühle deine Körperwärme. \\
ona li selo ala selo? && Schützt es? \\
\end{supertabular} 

%%%%%%%%%%%%%%%%%%%%%%%%%%%%%%%%%%%%%%%%%%%%%%%%%%%%%%%%%%%%%%%%%%%%%%%%%%
\newpage
%
\subsection*{Zahlen} 
\label{'numbers'}

\begin{supertabular}{p{5,5cm}|ll}
Wie werden Ordnungszahlen gebildet? && Mit dem Adjektiv \textit{nanpa} vor den Ziffern.  \\ % no-dictionary
Kann eine Ziffer direkt nach dem Separator \textit{li} stehen? && Ja, als Prädikatsadjektiv.  \\ % no-dictionary 
Mit welcher Wortarten werden Ziffern gebildet? && Adjektive \\ % no-dictionary
Wie werden große Zahlen dargestellt? && Mit dem Adjektiv \textit{mute}. \\ % no-dictionary
Welche Wortart kann in einem zusammengesetzten Substantiv nach Ziffern verwendet werden? && Possessivpronomen. \\ % no-dictionary
Wie bildet man Summen? && Mit der Konjunktion \textit{en}. \\ % no-dictionary 
\end{supertabular}

\begin{supertabular}{p{5,5cm}|ll}
nanpa - ona li ' jan nanpa wan. && Adjektiv \\ % no-dictionary
wan - mi wan. && transitives Verb, Adjektiv (Ziffer), Substantiv \\ % no-dictionary
luka - ni li ' luka tu. && Adjektiv, Adjektiv (Ziffer), Substantiv \\ % no-dictionary
luka - ni li ' luka # tu. && Adjektiv, Substantiv \\ % no-dictionary
nanpa - sina nanpa e kili. && transitives Verb \\ % no-dictionary
weka - sina tawa weka e sina. && Adverb \\ % no-dictionary
esun - o esun e ni! && transitives Verb \\ % no-dictionary
\end{supertabular}

\begin{supertabular}{p{5,5cm}|ll}
Ich sah drei Vögel.  && mi lukin e waso tu wan. \\ % & English - Toki Pona
Viele Leute kommen.  && jan mute li kama. \\ % & English - Toki Pona
Der Erste ist da.   && jan pi nanpa wan li lon. \\ % & English - Toki Pona
Ich besitze zwei Autos.  && mi jo e tomo tawa tu. \\ % & English - Toki Pona
Einige Leute kommen.   && jan mute lili li kama. \\ % & English - Toki Pona
Vereinigt euch!   && o wan! \\ % & English - Toki Pona
Ist dies ein Teil? && ni li ' wan anu seme? \\ % & English - Toki Pona
\end{supertabular}  

\begin{supertabular}{p{5,5cm}|ll}
mi weka e ijo tu ni.  && Ich entfernte diese beiden Dinge. \\
o tu.  && Teile. Zerbreche. \\
mi lukin e soweli luka.  && Ich sah fünf Pferde. \\
mi ' weka.  && Ich war weg. \\
ona li sike ala sike? && Dreht es (sich)? \\
\end{supertabular} 

%%%%%%%%%%%%%%%%%%%%%%%%%%%%%%%%%%%%%%%%%%%%%%%%%%%%%%%%%%%%%%%%%%%%%%%%%%
\newpage
%
\subsection*{Konditionalsätze} 
\label{'la'}


\begin{supertabular}{p{5,5cm}|ll}
Was ist eine Konditional-Phrase? && Sie stellt eine Bedingung dar. \\  % no-dictionary
Was folgt dem Separator \textit{la}? && Ein vollständiger Hauptsatz. \\  % no-dictionary
Woraus kann eine Konditional-Phrase bestehen? && Aus einem (zusammengesetzten) Substantiv/Pronoun \\ && oder eine vollständiger Satz. \\  % no-dictionary
Welche Wortarten können am Anfang einer Konditional-Phrase stehen? && Substantiv oder Pronomen. \\ &&  Optional kann eine Konjunktion davor sein. \\  % no-dictionary
Kann das Fragepronomen \textit{seme} in einer Konditional-Phrase stehen? && Ja, bei einem Fragesatz. \\  % no-dictionary
\end{supertabular}




%\begin{supertabular}{p{5,5cm}|ll}
%la - ken la ni li pona. && Separator \\ % no-dictionary
%ken - ken la mi tawa. && Substantiv \\ % no-dictionary
%\end{supertabular}



\begin{supertabular}{p{5,5cm}|ll}
Vielleicht wird Susan kommen.  && ken la jan Susan li kama. \\ % & English - Toki Pona
Letzte Nacht sah ich 'Akte-X'.  && tenpo pimeja pini la mi lukin e sitelen tawa X-Files. \\ % & English - Toki Pona
Wenn der Feind kommt, verbrenne diese Papiere.  && jan ike li kama la o seli e lipu ni. \\ % & English - Toki Pona
Vielleicht ist er in der Schule.  && ken la ona li lon tomo sona. \\ % & English - Toki Pona
Ich muß morgen arbeiten.  && tenpo suno kama la mi wile pali. \\ % & English - Toki Pona
Wenn es warm ist schwitze ich.  && seli li lon la mi pana e telo, tan selo mi. \\ % & English - Toki Pona
Öffne die Tür!  && o open e lupa. \\ % & English - Toki Pona
Ist der Mond groß heute Nacht?  && tenpo pimeja ni la mun li ' suli anu seme? \\ % & English - Toki Pona
Unter welchen Bedingungen werden Sie das tun? && seme la sina pali e ni?  \\ % & English - Toki Pona
\end{supertabular}  

\begin{supertabular}{p{5,5cm}|ll}
tenpo suno ni la mun li pimeja ala pimeja e suno? && Gibt es heute eine Sonnenfinsterniss? \\ 
ken la jan lili li wile moku e telo.  && Möglicherweise ist hat das Baby durst. \\
tenpo ali la o kama sona!  && Lerne immer!  \\
sina sona e toki ni la sina sona e toki pona!  && Finde dies selber raus  :o) \\
open la ala li lon! && Am Anfang war nichts! \\
ken la tomo pi ona en sina pi jelo en loje li ' ike, tawa mi. && Vielleicht mag ich das gelb-rot gemusterte Haus von ihr und dir nicht. \\
sina wile jo e ilo moli la sina wile moli e jan. && Wenn du eine Waffe möchtest, willst du Menschen töten. \\
jan nasa pi ilo moli li ken pana e ike. && Waffennarren können Schlechtes bringen. \\
\end{supertabular}  

%%%%%%%%%%%%%%%%%%%%%%%%%%%%%%%%%%%%%%%%%%%%%%%%%%%%%%%%%%%%%%%%%%%%%%%%%%
%%%%%%%%%%%%%%%%%%%%%%%%%%%%%%%%%%%%%%%%%%%%%%%%%%%%%%%%%%%%%%%%%%%%%%%%%%
% eof
