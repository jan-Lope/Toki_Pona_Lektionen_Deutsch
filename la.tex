%%%%%%%%%%%%%%%%%%%%%%%%%%%%%%%%%%%%%%%%%%%%%%%%%%%%%%%%%%%%%%%%%%%%%%%%%%
\section{la}
%%%%%%%%%%%%%%%%%%%%%%%%%%%%%%%%%%%%%%%%%%%%%%%%%%%%%%%%%%%%%%%%%%%%%%%%%%
%
%%%%%%%%%%%%%%%%%%%%%%%%%%%%%%%%%%%%%%%%%%%%%%%%%%%%%%%%%%%%%%%%%%%%%%%%%%
\subsection*{Vokabeln}
%%%%%%%%%%%%%%%%%%%%%%%%%%%%%%%%%%%%%%%%%%%%%%%%%%%%%%%%%%%%%%%%%%%%%%%%%%
%
\index{\textit{la}}
\index{\textit{mun}}
\index{\textit{open}}
\index{\textit{pini}}
\index{\textit{tenpo}}
\index{\textit{kipisi}}
\index{Trennwort!\textit{la}}
\index{Mond}
\index{�ffnen}
\index{anfangen}
\index{beginnen}
\index{einschalten}
\index{beenden}
\index{aufh�ren}
\index{ausschalten}
\index{Ende}
\index{erledigt}
\index{Ziel}
\index{Zeit}
\index{Moment}
\index{Zeitabschnitt}
\index{schneiden}
\begin{supertabular}{p{2,5cm}|ll}
\textbf{\dots la \dots} && \textit{Separator}: Ein \glqq la\grqq  trennt eine Phrase (Satz, Halbsatz oder Substantiv) von einem Satz. \\  && Verwende ein \glqq la\grqq nicht zusammmen mit einem der anderen Separatoren. \\  % no-dictionary
 && \\ % no-dictionary
\textbf{\dots mun} && \textit{Adjektiv}: Mond- \\  % no-dictionary
\textbf{mun} && \textit{Substantiv}: Mond, Stern, Planet \\  % no-dictionary
 && \\ % no-dictionary
\textbf{\dots open} && \textit{Adjektiv}: (er-)�ffnend \\  % no-dictionary
\textbf{open} && \textit{Substantiv}: Anfang, Beginn, Start  \\  % no-dictionary
\textbf{open la \dots} && \textit{Substantiv}: zu Anfang, bei Beginn, beim Start \\  % no-dictionary
\textbf{open} && \textit{Substantiv}: start, beginning, opening \\  % no-dictionary
\textbf{open (e \dots)} && \textit{Verb, transitiv}: �ffnen, andrehen, anmachen, anschalten, beginnen, anfangen \\  % no-dictionary
\textbf{open \dots } && \textit{Hilfsverb}: beginnen, starten \\  % no-dictionary
 && \\ % no-dictionary\dots
\textbf{\dots pini} && \textit{Adjektiv}: abgeschlossen, vollendet, beendet, erledigt, fertiggestellt  \\  % no-dictionary
\textbf{\dots pini} && \textit{Adverb}: vor, vorbei, vor�ber, perfekt, vollendet \\  % no-dictionary
\textbf{pini} && \textit{Substantiv}: Ende, Zweck, Ziel, Tip \\  % no-dictionary
\textbf{pini (e \dots)} && \textit{Verb, transitiv}: beenden, erledigen, fertigstellen, vollenden  \\  % no-dictionary
\textbf{pini \dots } && \textit{Hilfsverb}: stoppen, beenden, unterbrechen \\  % no-dictionary
 && \\ % no-dictionary
\textbf{\dots tenpo} && \textit{Adjektiv}: zeitliche, chronologische, chronologisch \\  % no-dictionary
\textbf{\dots tenpo} && \textit{Adverb}: chronologisch \\  % no-dictionary
\textbf{tenpo} && \textit{Substantiv}: Zeit, Zeitperiode, Zeitabschnitt, Moment, Dauer, Situation \\  % no-dictionary
 && \\ % no-dictionary
\textbf{kipisi } && \textit{Substantiv}: Abschnitt, Fragment, Scheibe \\  % no-dictionary
\textbf{kipisi (e \dots)} && \textit{Verb, transitiv}: schneiden \\  % no-dictionary
 && \\ % no-dictionary
\textbf{ante la \dots} && \textit{Substantiv}: Wenn Unterschied, bei Abweichung, bei Differenz  \\  % no-dictionary
 && \\ % no-dictionary
\textbf{ike la \dots} && \textit{Substantiv}: wenn �bel ..., bei Verderbtheit, bei Boshaftigkeit \\  % no-dictionary
 && \\ % no-dictionary
\textbf{ken la \dots} && \textit{Substantiv}: es besteht die M�glichkeit, bei Erlaubnis, bei Genehmigung \\  % no-dictionary
 && \\ % no-dictionary
\textbf{kin la \dots} && \textit{Substantiv}: wenn Wirklichkeit, wenn Tatsache \\ % no-dictionary % no-dictionary
 && \\ % no-dictionary
\textbf{pona la \dots} && \textit{Substantiv}: zum Gl�ck, wenn Einfachheit \\  % no-dictionary
\end{supertabular} \\
%
%%%%%%%%%%%%%%%%%%%%%%%%%%%%%%%%%%%%%%%%%%%%%%%%%%%%%%%%%%%%%%%%%%%%%%%%%%
\index{\textit{la}}
\subsection*{\textit{la}}
%%%%%%%%%%%%%%%%%%%%%%%%%%%%%%%%%%%%%%%%%%%%%%%%%%%%%%%%%%%%%%%%%%%%%%%%%%
%
%%%%%%%%%%%%%%%%%%%%%%%%%%%%%%%%%%%%%%%%%%%%%%%%%%%%%%%%%%%%%%%%%%%%%%%%%%
\index{vielleicht}
\index{\textit{la}!\textit{ken}}
\index{\textit{ken}!\textit{la}}
\subsubsection*{\textit{la} nach einem einzelnen Wort}
%%%%%%%%%%%%%%%%%%%%%%%%%%%%%%%%%%%%%%%%%%%%%%%%%%%%%%%%%%%%%%%%%%%%%%%%%%
%
Betrachten wir uns wie sich ein einfacher Satz durch \textit{la} ver�ndert.

\begin{supertabular}{p{5,5cm}|ll}
ilo li ' pakala. && Das Werkzeug ist kaputt. \\
\textbf{ken la} ilo li ' pakala. && Vielleicht ist das Werkzeug kaputt. \\
\end{supertabular} 

Es wird mit der \textit{la}-Phrase begonnen. 
Nach \textit{la} beginnt ein vollst�ndiger Hauptsatz.

Eine \textit{la}-Phrase kann aus einem einzelnen Wort bestehen. Dieses einzelne Wort kann nur ein Substantiv bzw. Pronoun sein.
Eine \textit{la}-Phrase kann aus einem zusammengesetzten Substantiv bzw. Pronoun betehen. Da hei�t, nach dem Substantiv bzw. Pronoun folgen ein oder mehrere Adjektive bzw. \textit{pi}-Phrasen. 
Optional kann vor dem Substantiv bzw. Pronoun eine Konjunktion (\textit{anu}, \textit{en}, \textit{taso}) sein. 
Ein \textit{la}-Phrase kann aber auch ein kompletter Satz sein. 

Das Substantiv \textit{ken} bedeutet ja 'M�glichkeit'. 
\textit{ken la} bedeutet also 'Wenn es eine M�glichkeit gibt' oder besser 'vielleicht'. 

\begin{supertabular}{p{5,5cm}|ll}
\textbf{ken la} jan Lisa li jo e ona. && Vielleicht hat Lisa es. \\
\textbf{ken la} ona li lape. && Vielleicht schl�ft er. \\
\textbf{ken la} mi ken tawa ma Elopa. && Vielleicht kann ich nach Europa gehen. \\
\end{supertabular} 
%
%%%%%%%%%%%%%%%%%%%%%%%%%%%%%%%%%%%%%%%%%%%%%%%%%%%%%%%%%%%%%%%%%%%%%%%%%%
\index{\textit{la}!Zeit}
\index{Zeit!\textit{la}}
\index{\textit{la}!\textit{tenpo}}
\index{\textit{tenpo}!\textit{la}}
\subsubsection*{Zeiten und \textit{la}}
%%%%%%%%%%%%%%%%%%%%%%%%%%%%%%%%%%%%%%%%%%%%%%%%%%%%%%%%%%%%%%%%%%%%%%%%%%
%
Bevor wir die Kombination \textit{tenpo ... la} behandeln, m�ssen wir erst
zusammengesetzte Substantive mit \textit{tenpo} kennenlernen. 

\index{Tag}
\index{Nacht}
\index{jetzt}
\index{Gegenwart}
\index{heute}
\index{Nacht!heute}
\index{Zukunft}
\index{bald}
\index{Vergangenheit}
\index{gestern}
\index{Nacht!letzte}
\index{morgen}
\index{oft}
\begin{supertabular}{p{5,5cm}|ll}
tenpo suno && Tag \\
tenpo pimeja && Nacht \\
tenpo ni && jetzt, Gegenwart \\
tenpo suno ni && heute \\
tenpo pimeja ni && heute Nacht \\
tenpo kama && Zukunft \\
tenpo kama lili && bald \\
tenpo pini && Vergangenheit \\
tenpo suno pini && gestern \\
tenpo pimeja pini && letzte Nacht \\
tenpo suno kama && morgen \\
tenpo mute && oft \\
\end{supertabular} 

Jetzt k�nnen wir diese Phrasen vor \textit{la} stellen, um Zeitangaben zu machen.

\index{betrunken}
\begin{supertabular}{p{5,5cm}|ll}
\textbf{tenpo} pini \textbf{la} mi ' weka. && Ich war weg. \\
\textbf{tenpo} ni \textbf{la} mi lon. && Jetzt bin ich hier. \\
\textbf{tenpo} kama \textbf{la} mi lape. && Ich werde schlafen. \\
\textbf{tenpo} pimeja pini \textbf{la} mi kama nasa. && Letzte Nacht war ich betrunken. \\
\end{supertabular} 

\index{Alter}
\index{alt}
\index{wie!alt}
\index{Jahr}
\index{Geburtstag}
Wir benutzen auch eine \textit{tenpo ... la} Phrase um �ber das Alter zu sprechen. 
Nur als Warnung: 
�ber das Alter sprechen wir mit einem lustigen Ausdruck. 

\begin{supertabular}{p{5,5cm}|ll}
\textbf{tenpo} pi mute seme \textbf{la} sina sike e suno? && Wie alt bist du? \\
\end{supertabular} 

Geburtstag hat man ja einmal im Jahr und jedes Jahr umkreist die Erde einmal die Sonne.
Um zu antworten ersetze einfach \textit{pi mute seme} mit deinem Alter.

\begin{supertabular}{p{5,5cm}|ll}
\textbf{tenpo} tu tu \textbf{la} mi sike e suno. && Ich bin vier Jahre alt. \\
\end{supertabular} 

Das Pr�positional-Objekt nach der Pr�position \textbf{lon} kann vor \textbf{la} mit der gleichen Bedeutung platziert werden.

\begin{supertabular}{p{5,5cm}|ll}
mi moku e telo \textbf{lon} tenpo ni. && Ich trinke jetzt. \\
tenpo ni \textbf{la} mi moku e telo. &&  Ich trinke jetzt. \\
\end{supertabular} 
%
%%%%%%%%%%%%%%%%%%%%%%%%%%%%%%%%%%%%%%%%%%%%%%%%%%%%%%%%%%%%%%%%%%%%%%%%%%
\index{wenn!\textit{la}}
\index{\textit{la}!wenn}
\subsubsection*{Wenn-Ausdr�cke mit \textit{la}}
%%%%%%%%%%%%%%%%%%%%%%%%%%%%%%%%%%%%%%%%%%%%%%%%%%%%%%%%%%%%%%%%%%%%%%%%%%
%
\begin{supertabular}{p{5,5cm}|ll}
mama mi li ' moli \textbf{la} mi pilin ike. && Wenn meine Eltern sterben, f�hlte ich mich schlecht. \\
\end{supertabular} 

Alle \textit{la}-S�tze haben den gleichen Aufbau.

\begin{supertabular}{p{5,5cm}|ll}
1 la 2. && Wenn 1 passiert, passiert auch 2. \\  % no-dictionary
\end{supertabular} 

\begin{supertabular}{p{5,5cm}|ll}
mi lape \textbf{la} ali li ' pona. && Wenn ich schlafe ist alles gut. \\
sina moku e telo nasa \textbf{la} sina ' nasa. && Wenn du Bier trinkst bist du albern. \\
sina ' moli \textbf{la} sina ken ala toki. && Wenn du tot bist dann kannst du nicht sprechen. \\
mi pali mute \textbf{la} mi pilin ike. && Wenn ich viel arbeite f�hle ich mich schlecht. \\
sama pi ni en ona \textbf{la} mi wile jo e ni tu. && Wenn dieses und jenes gleich ist, will ich beides. \\
tawa mi \textbf{la} mi pilin pona. && Bin ich in Bewegung, ich f�hle mich gut. \\
tan ni \textbf{la} mi sona e nasin. && Wenn dies die Ursache ist, wissen wir die L�sung. \\

lon ona \textbf{la} mi ken lukin e ona. && Hat es ein Sein, k�nnen wir es auch sehen. \\


\end{supertabular} 

Es sind auch zwei \textit{la} in einem Satz m�glich. Bitte aber bitte nicht mehr als zwei.

\begin{supertabular}{p{5,5cm}|ll}
ken \textbf{la} tenpo pimeja \textbf{la} ni li ' pona. && Vielleicht, wenn es Nacht ist, ist dies gut. \\  
\end{supertabular} 

Da \textit{la} ein Trennwort ist, sind Kommas zusammen mit \textit{la} weder notwendig noch sinnvoll. 
%
%%%%%%%%%%%%%%%%%%%%%%%%%%%%%%%%%%%%%%%%%%%%%%%%%%%%%%%%%%%%%%%%%%%%%%%%%%
\index{Steigerung}
\index{Komparativ}
\index{Superlativ}
\index{Adjektiv!Steigerung}
\index{Adjektiv!Komparativ}
\index{Adjektiv!Superlativ}
\subsection*{Verschiedenes}
%%%%%%%%%%%%%%%%%%%%%%%%%%%%%%%%%%%%%%%%%%%%%%%%%%%%%%%%%%%%%%%%%%%%%%%%%%
\subsubsection*{Steigerungsformen} 
%%%%%%%%%%%%%%%%%%%%%%%%%%%%%%%%%%%%%%%%%%%%%%%%%%%%%%%%%%%%%%%%%%%%%%%%%%
%
Zum Schlu� wollen wir uns Komparative und Superlative anschauen. 
Steigerungsformen werden in \textit{toki pona} durch zwei S�tze gebildet.
So z.B. 'Lisa ist besser als Susan.':

\begin{supertabular}{p{5,5cm}|ll}
jan Lisa li ' pona mute. ...  && Lisa ist sehr gut ... \\ % no-dictionary
... jan Susan li ' pona lili. && ... Susan ist ein bi�chen gut. \\ % no-dictionary
\end{supertabular} 

Man verwendet einfach zwei S�tze mit unterschiedlichen Gewichtungen.

\begin{supertabular}{p{5,5cm}|ll}
mi ' suli mute. sina ' suli lili. && Ich bin gr��er als du. \\
mi moku mute. sina moku lili. && Ich esse mehr als du. \\
\end{supertabular} 

%%%%%%%%%%%%%%%%%%%%%%%%%%%%%%%%%%%%%%%%%%%%%%%%%%%%%%%%%%%%%%%%%%%%%%%%%%
\index{�berschriften}
\subsubsection*{�berschriften} 
%%%%%%%%%%%%%%%%%%%%%%%%%%%%%%%%%%%%%%%%%%%%%%%%%%%%%%%%%%%%%%%%%%%%%%%%%%
%
�berschriften k�nnen wie in anderen Sprachen
unvollst�ndige S�tze sein und m�ssen nicht mit einem Satzzeichen enden.

\textit{tenpo mun nanpa luka luka wan} \\
tenpo ni li ike kin, lon ma Tosi. \\
suno li suli lili kin. \\ 
telo li kama, lon sewi. \\
kasi li moli. \\
waso li tawa. \\
tenpo seli o kama! 
%
%%%%%%%%%%%%%%%%%%%%%%%%%%%%%%%%%%%%%%%%%%%%%%%%%%%%%%%%%%%%%%%%%%%%%%%%%%
\newpage
\subsection*{�bungen 17 (Antworten siehe Seite ~\pageref{'la'})}
%%%%%%%%%%%%%%%%%%%%%%%%%%%%%%%%%%%%%%%%%%%%%%%%%%%%%%%%%%%%%%%%%%%%%%%%%%
%
\begin{supertabular}{p{5,5cm}|ll}
Vielleicht wird Susan kommen.  && \\ % no-dictionary
Letzte Nacht sah ich 'Akte-X'.  &&   \\ % no-dictionary
Wenn der Feind kommt, verbrenne diese Papiere.  &&   \\ % no-dictionary
Vielleicht ist er in der Schule.  &&   \\ % no-dictionary
Ich mu� morgen arbeiten.  &&   \\ % no-dictionary
Wenn es warm ist schwitze ich. *  &&  \\ % no-dictionary
�ffne die T�r!   &&  \\ % no-dictionary
Der Mond ist gro� heute Nacht.  &&  \\ % no-dictionary
 && \\ % no-dictionary
ken la jan lili li wile moku e telo.  &&   \\ % no-dictionary
tenpo ali la o kama sona!   &&  \\ % no-dictionary
sina sona e toki ni la sina sona e toki pona!   &&  \\ % no-dictionary
\end{supertabular}

* 'Hitze ist hier, Ich lasse Fl�ssigkeit aus meiner Haut.'
%
%%%%%%%%%%%%%%%%%%%%%%%%%%%%%%%%%%%%%%%%%%%%%%%%%%%%%%%%%%%%%%%%%%%%%%%%%%
% eof
