%%%%%%%%%%%%%%%%%%%%%%%%%%%%%%%%%%%%%%%%%%%%%%%%%%%%%%%%%%%%%%%%%%%%%%%%%%
\section{Direkte Objekte und zusammengesetze S�tze}
%
%%%%%%%%%%%%%%%%%%%%%%%%%%%%%%%%%%%%%%%%%%%%%%%%%%%%%%%%%%%%%%%%%%%%%%%%%%
\subsection*{Vokabeln}
%%%%%%%%%%%%%%%%%%%%%%%%%%%%%%%%%%%%%%%%%%%%%%%%%%%%%%%%%%%%%%%%%%%%%%%%%%
%
\index{\textit{ilo}}
\index{\textit{kili}}
\index{\textit{ni}}
\index{\textit{ona}}
\index{\textit{pipi}}
\index{\textit{ma}}
\index{\textit{ijo}} 
\index{\textit{jo}}
\index{\textit{lukin}}
\index{\textit{pakala}}
\index{\textit{unpa}}
\index{\textit{wile}}
\index{\textit{e}}
\index{er}
\index{sie}
\index{es}
\index{seins}
\index{seine}
\index{ihrs}
\index{sie!Plural}
\index{deren}
\index{Frucht}
\index{Obst}
\index{Gem�se}
\index{Pilz}
\index{Trennwort!\textit{e}}
\index{wollen}
\index{brauchen}
\index{m�ssen}
\index{begehren}
\index{Wunsch}
\index{Zwang}
\index{Werkzeug}
\index{Ger�t}
\index{Maschine}
\index{Zeug}
\index{Sache}
\index{Ding}
\index{dieses}
\index{jenes}
\index{K�fer}
\index{Insekt}
\index{Spinne}
\index{Skorpion}
\index{Land}
\index{Region}
\index{Staat}
\index{Kontinent}
\index{Erde}
\index{haben}
\index{besitzen}
\index{Besitz}
\index{Eigentum}
\index{sehen}
\index{schauen}
\index{betrachten}
\index{Sicht}
\index{Vision}
\index{sichtbar}
\index{verletzen}
\index{zerst�ren}
\index{Unfall}
\index{Crash}
\index{Fehler}
\index{Sex!mit jemanden haben}
\index{sexuell}
\index{sexy}
\begin{supertabular}{p{5,5cm}|ll}
ona &&  er, sie, es, seins, seine, ihrs, sie (Plural), deren  \\
kili && Frucht, Obst, Gem�se, Pilz  \\
e &&  Trennwort: Es leitet ein direktes Objekt ein.  \\
wile &&  wollen, brauchen, m�ssen, begehren, Wunsch, Zwang \\
ilo && Werkzeug, Ger�t, Maschine \\
ijo &&  Zeug, Sache, Ding  \\
ni &&  dieses, jenes  \\
pipi &&  K�fer, Insekt, Spinne, Skorpion \\
ma &&  Land, Region, Staat, Kontinent, Erde  \\
jo &&  haben, besitzen, Besitz, Eigentum \\
lukin &&  sehen, schauen, betrachten, Sicht, Vision, sichtbar \\
pakala &&  verletzen, zerst�ren, Unfall, Crash, Fehler \\
unpa && mit jemanden Sex haben, sexuell, sexy \\
\end{supertabular}  

%%%%%%%%%%%%%%%%%%%%%%%%%%%%%%%%%%%%%%%%%%%%%%%%%%%%%%%%%%%%%%%%%%%%%%%%%%
\index{\textit{e}}
\index{Objekt!direktes}
\subsection*{Einleitung eines direkten Objektes mit \textit{e}}
%%%%%%%%%%%%%%%%%%%%%%%%%%%%%%%%%%%%%%%%%%%%%%%%%%%%%%%%%%%%%%%%%%%%%%%%%%
%
Wir haben gesehen wie S�tze, wie \textit{mi moku}, zwei potentielle Bedeutungen haben
k�nnen: 
'Ich esse.' oder 'Ich bin Nahrung'. 
Die genaue Bedeutung l��t sich nur aus dem Kontext erschlie�en. 
Man kann dies aber exakter sagen.

\begin{supertabular}{p{5,5cm}|ll}
mi moku \textbf{e} kili. && Ich esse Fr�chte. \\
ona li lukin \textbf{e} pipi. && Er betrachtet einen K�fer. \\
\end{supertabular} 

In \textit{toki pona} wird das direkte Objekt mit \textit{e} vom Verb getrennt.
 
Auch haben wir gesehen, wie \textit{sina pona} zwei m�gliche Bedeutungen hat: 
'Du bist gut.' oder 'Du reparierst'. 
Wenn kein Objekt angegeben wird, kann man wohl annehmen, da� es 'Du bist
gut.' bedeutet.
Wenn man sagt 'Du reparierst.', sagt man auch meist was man repariert.
Und dazu verwendet man \textit{e}.

\begin{supertabular}{p{5,5cm}|ll}
ona li pona \textbf{e} ilo. && Sie repariert die Maschine. \\
mi pona \textbf{e} ijo. && Ich repariere etwas. \\
\end{supertabular} 

%%%%%%%%%%%%%%%%%%%%%%%%%%%%%%%%%%%%%%%%%%%%%%%%%%%%%%%%%%%%%%%%%%%%%%%%%%
% \newpage
\index{\textit{e}!\textit{wile}}
\index{\textit{wile}!\textit{e}}
\subsection*{Direkte Objekte mit \textit{e} und \textit{wile}}
%%%%%%%%%%%%%%%%%%%%%%%%%%%%%%%%%%%%%%%%%%%%%%%%%%%%%%%%%%%%%%%%%%%%%%%%%%
%
Um zu sagen, da� du etwas bestimmtes tun willst, verwende \textit{wile} und \textit{e}.

\begin{supertabular}{p{5,5cm}|ll}
mi \textbf{wile} lukin \textbf{e} ma. && Ich m�chte die Landschaft sehen. \\
mi \textbf{wile} pakala \textbf{e} sina. && Ich mu� dich vernichten. \\
ona li \textbf{wile} jo \textbf{e} ilo. && Er will das Werkzeug haben. \\
\end{supertabular} 

%%%%%%%%%%%%%%%%%%%%%%%%%%%%%%%%%%%%%%%%%%%%%%%%%%%%%%%%%%%%%%%%%%%%%%%%%%
\newpage
\index{S�tze!zusammengesetzt}
\index{S�tze!zusammengesetzt!\textit{li}}
\index{S�tze!zusammengesetzt!\textit{e}}
\index{\textit{li}!zusammengesetzte S�tze}
\index{\textit{e}!zusammengesetzte S�tze}
\index{und}
\label{'multiple_li'}
\subsection*{Zusammengesetzte S�tze}
%%%%%%%%%%%%%%%%%%%%%%%%%%%%%%%%%%%%%%%%%%%%%%%%%%%%%%%%%%%%%%%%%%%%%%%%%%
%
Im Deutschen bildet man zusammengesetzte S�tze oft mit 'und'.
In \textit{toki pona} gibt es zwei Arten, zusammengesetzte S�tze zu bilden. 
Eine M�glichkeit verwendet \textit{li}, die andere Variante verwendet \textit{e}. 
%
\subsubsection*{Zusammengesetzte S�tze mit \textit{li}}
%%%%%%%%%%%%%%%%%%%%%%%%%%%%%%%%%%%%%%%%%%%%%%%%%%%%%%%%%%%%%%%%%%%%%%%%%%
%
Mehrere \textit{li} werden bei mehreren Handlungen verwendet.

\begin{supertabular}{p{5,5cm}|ll}
pipi \textbf{li} lukin \textbf{li} unpa. && Der K�fer schaut und hat Sex. \\
\end{supertabular} 

\textit{li} wird vor jedes Verb gesetzt, damit alle Handlungen eingeleitet werden.

\begin{supertabular}{p{5,5cm}|ll}
mi moku \textbf{li} pakala. && Ich esse und zerst�re. \\
\end{supertabular} 

\textit{li} wird hier nicht zwischen \textit{mi} und \textit{moku} verwendet, da ja nach dem Subjekt \textit{mi} (und \textit{sina}) kein \textit{li} folgt. 
Das zweite Verb \textit{pakala} mu� aber mit einem \textit{li} eingeleitet werden, da ansonsten
der Satz chaotisch und verwirrend w�re. 
Es gibt keine verschachtelten \textbf{li}-Phrasen. Man kann die Reihenfolge vertauschen. \\
ona \textbf{li} moku \textbf{li} pona. = ona \textbf{li} pona \textbf{li} moku. 
%
%%%%%%%%%%%%%%%%%%%%%%%%%%%%%%%%%%%%%%%%%%%%%%%%%%%%%%%%%%%%%%%%%%%%%%%%%%
\label{'multiple_e'}
\subsubsection*{Zusammengesetzte S�tze mit \textit{e}}
%%%%%%%%%%%%%%%%%%%%%%%%%%%%%%%%%%%%%%%%%%%%%%%%%%%%%%%%%%%%%%%%%%%%%%%%%%
%
Wenn sich eine Handlung auf mehrere direkte Objekte bezieht, dann
werden mehrere \textit{e} verwendet.  

\begin{supertabular}{p{5,5cm}|ll}
mi moku \textbf{e} kili \textbf{e} telo. && Ich verspeise/trinke die Frucht und das Wasser. \\
mi wile lukin \textbf{e} ma \textbf{e} suno. && Ich will das Land und die Sonne sehen. \\
\end{supertabular} 
Es gibt keine verschachtelten \textbf{e}-Phrasen. Man kann die Reihenfolge vertauschen. \\
mi moku \textbf{e} moku \textbf{e} telo. = mi moku \textbf{e} telo \textbf{e} moku. 

%%%%%%%%%%%%%%%%%%%%%%%%%%%%%%%%%%%%%%%%%%%%%%%%%%%%%%%%%%%%%%%%%%%%%%%%%%
\subsection*{�bungen 4 (Antworten siehe Seite ~\pageref{'direct_objects_compund_sentences'})}
%%%%%%%%%%%%%%%%%%%%%%%%%%%%%%%%%%%%%%%%%%%%%%%%%%%%%%%%%%%%%%%%%%%%%%%%%%
%
\begin{supertabular}{p{5,5cm}|ll}
Ich habe ein Werkzeug. &&  \\ % no-dictionary
Sie isst Fr�chte. &&  \\ % no-dictionary
Etwas schaut mich an. &&  \\ % no-dictionary
Er will die Spinne zerquetschen. &&  \\ % no-dictionary
Ananas sind Nahrung und gut. &&  \\ % no-dictionary
Der K�fer hat Durst. && \\ % no-dictionary
  && \\ % no-dictionary
mi lukin e ni. &&  \\ % no-dictionary
mi wile unpa e ona. &&   \\ % no-dictionary
jan li wile jo e ma. &&  \\ % no-dictionary
mi jan li suli. &&  \\ % no-dictionary
\end{supertabular} 
%%%%%%%%%%%%%%%%%%%%%%%%%%%%%%%%%%%%%%%%%%%%%%%%%%%%%%%%%%%%%%%%%%%%%%%%%%
% eof
