%%%%%%%%%%%%%%%%%%%%%%%%%%%%%%%%%%%%%%%%%%%%%%%%%%%%%%%%%%%%%%%%%%%%%%%%%%
\section{Indirekte Objekte}
%%%%%%%%%%%%%%%%%%%%%%%%%%%%%%%%%%%%%%%%%%%%%%%%%%%%%%%%%%%%%%%%%%%%%%%%%%
%
%%%%%%%%%%%%%%%%%%%%%%%%%%%%%%%%%%%%%%%%%%%%%%%%%%%%%%%%%%%%%%%%%%%%%%%%%%
\subsection*{Vokabeln}
%%%%%%%%%%%%%%%%%%%%%%%%%%%%%%%%%%%%%%%%%%%%%%%%%%%%%%%%%%%%%%%%%%%%%%%%%%
\begin{supertabular}{p{2,5cm}|ll}
%
\index{kepeken}
\textbf{kepeken} && \textit{Substantiv}: Nutzung, Verwendung, Werkzeug \\ % no-dictionary
\textbf{\dots , kepeken \dots} && \textit{Pr�position}: mit, mittels, per \\ % no-dictionary
\textbf{kepeken} && \textit{Verb, intransitiv}: benutzen, verwenden, anwenden, nutzen, gebrauchen \\ % no-dictionary
% \textbf{kepeken \dots} && \textit{Hilfsverb}: verwenden \\ % no-dictionary
 && \\ % no-dictionary
%
\index{kiwen}
\textbf{\dots kiwen} && \textit{Adjektiv}: hart, schwer, heftig, fest, solid, stabil, robust, steinhart \\ % no-dictionary
\textbf{\dots kiwen} && \textit{Adverb}: hart, schwer, heftig, fest, solid, stabil, robust, steinhart \\ % no-dictionary
\textbf{kiwen} && \textit{Substantiv}: Felsbrocken, Fels, Stein, Metall, Mineral, Lehm \\ % no-dictionary
\textbf{kiwen (e \dots)} && \textit{Verb, transitiv}: verfestigen, verh�rten, versteinern \\ % no-dictionary
%
\index{kon}
\textbf{\dots kon} && \textit{Adjektiv}: luftig, �therisch, gasf�rmig, gasartig \\ % no-dictionary
\textbf{\dots kon} && \textit{Adverb}: luftig, �therisch, gasf�rmig, gasartig \\ % no-dictionary
\textbf{kon} && \textit{Substantiv}: Luft, Atmosph�re, Wind, Odor, Geruch, Seele, Geist \\ % no-dictionary
\textbf{kon} && \textit{Verb, intransitiv}: atmen \\ % no-dictionary
\textbf{kon (e \dots)} && \textit{Verb, transitiv}: wegblasen \\ % no-dictionary
 && \\ % no-dictionary
%
\index{lon}
\textbf{\dots lon} && \textit{Adjektiv}: real, wahr, bestehend, richtig, echt \\ % no-dictionary
\textbf{lon} && \textit{Substantiv}: Existenz, Sein, Pr�senz \\ % no-dictionary
\textbf{\dots , lon \dots} && \textit{Pr�position}: in, im, an, am, bei, auf \\ % no-dictionary
\textbf{lon} && \textit{Verb, intransitiv}: anwesend sein, vorhanden sein, enthalten sein, existieren, leben \\ % no-dictionary
\textbf{lon (e \dots)} && \textit{Verb, transitiv}: geb�ren, erschaffen \\ % no-dictionary
 && \\ % no-dictionary
%
\index{pana}
\textbf{\dots pana} && \textit{Adjektiv}: gro�z�gig \\ % no-dictionary
\textbf{pana} && \textit{Substantiv}: Verabreichung, Transformation, �berf�hrung, Umtausch \\ % no-dictionary
\textbf{pana (e \dots)} && \textit{Verb, transitiv}: geben, senden, entbinden, strahlen, emittieren \\ % no-dictionary
 && \\ % no-dictionary
%
\index{poki}
\textbf{poki} && \textit{Substantiv}: Container, Beh�lter, Box, Sch�ssel, Schale, Tasse, (Trink-) Glas \\ % no-dictionary
\textbf{poki (e \dots)} && \textit{Verb, transitiv}: einpacken, einwecken, einwickeln, einmachen  \\ % no-dictionary
 && \\ % no-dictionary
%
\index{tawa}
\textbf{\dots tawa} && \textit{Adjektiv}: beweglich, mobil \\ % no-dictionary
\textbf{\dots tawa} && \textit{Adverb}: beweglich, mobil \\ % no-dictionary
\textbf{tawa} && \textit{Substantiv}: Bewegung, Regung, Transport \\ % no-dictionary
\textbf{\dots , tawa \dots} && \textit{Pr�position}: zu, um zu, zu hin, nach, f�r, bis, in \\ % no-dictionary
\textbf{tawa} && \textit{Verb, intransitiv}: gehen, spazieren, verlassen, abfahren, reisen, besuchen \\ % no-dictionary
\textbf{tawa (e \dots)} && \textit{Verb, transitiv}: bewegen, verschieben, verlagern \\ % no-dictionary
\end{supertabular} \\
%
%%%%%%%%%%%%%%%%%%%%%%%%%%%%%%%%%%%%%%%%%%%%%%%%%%%%%%%%%%%%%%%%%%%%%%%%%%
\newpage{}
%
\subsection*{Indirekte Objekte und intransitive Verben}
%
\index{Objekt!indirekt}
\index{Verb!intransitive}
\index{Pr�dikat-Phrase}
%%%%%%%%%%%%%%%%%%%%%%%%%%%%%%%%%%%%%%%%%%%%%%%%%%%%%%%%%%%%%%%%%%%%%%%%%%
%
Wir haben schon direkte Objekte kennengelernt. 
Ein direktes Objekt ist von der Handlung (also dem transitiven Verb) am st�rksten beeinflusst. 
Das direkte Objekt (Akkusativobjekt) kann man mit 'Wen' oder 'Was' erfragen ('Was repariert sie?').
Aber in dem Satz 'Ich bin im Haus.' ist 'im Haus' ein indirektes Objekt, da man es nicht mit 'Wen' oder 'Was' erfragen kann.
Auch wird es nicht direkt vom Pr�dikat beeinflusst. 
Ein indirektes Objekt ist auch Teil der Pr�dikat-Phrase. 
Im indirekten Objekt ist der erste Slot immer ein Substantiv- oder Pronomen-Slot.
Danach sind optionale Slots f�r Adjektive, Possessivpronomen und Demonstrativpronomen m�glich. 

Wir haben bereits transitive Verben kennen gelernt. 
Ein transitives Verb macht etwas mit dem direkten Objekt. 
Dagegen werden Verben, die kein Objekt beeinflussen, intransitive Verben genannt. 
Nach einem intransitiven Verb folgt entweder kein Objekt oder ein indirektes Objekt. 
In den S�tzen 'Ich bin.' und 'Ich bin im Haus.' ist 'bin' ein intransitives Verb. 
In \textit{toki pona} steht zwischen dem intrasitiven Verb und indirekten Objekt kein \textit{e}.

%
%%%%%%%%%%%%%%%%%%%%%%%%%%%%%%%%%%%%%%%%%%%%%%%%%%%%%%%%%%%%%%%%%%%%%%%%%%
\index{\textit{lon}!intransitives Verb}
%
In diesen Beispielen wird das intransitive Verb \textit{lon} verwendet. 
Da vor \textit{lon} kein anderes Pr�dikat steht, muss \textit{lon} ein Verb sein.

\begin{supertabular}{p{5,5cm}|ll}
mi lon tomo. && Ich bin im Haus. \\
suno li lon sewi. && Die Sonne ist am Himmel. \\
kili li lon poki. && Die Fr�chte sind im Korb. \\
\end{supertabular} 

%%%%%%%%%%%%%%%%%%%%%%%%%%%%%%%%%%%%%%%%%%%%%%%%%%%%%%%%%%%%%%%%%%%%%%%%%%
\index{\textit{kepeken}!intransitives Verb}
%
Hier wird das intransitive Verb \textit{kepeken} verwendet. 

\begin{supertabular}{p{5,5cm}|ll}
mi kepeken ilo. && Ich benutze ein Werkzeug. \\
sina wile kepeken ilo. && Du mu�t die Hilfsmittel verwenden. \\
mi kepeken poki ni. && Ich nehme diese Tasse. \\
\end{supertabular} 

In einigen anderen Lektionen wird das transitive Verb \textit{kepeken} verwendet. 
Dies liegt sicherlich daran, dass man mit 'Was' nach dem Objekt nach \textit{kepeken} fragen kann. 
Da aber das Objekt nicht unmittelbar vom Verb \textit{kepeken} beeinflusst wird, ist es ein indirektes Objekt und \textit{kepeken} ein intransitives Verb. 

%%%%%%%%%%%%%%%%%%%%%%%%%%%%%%%%%%%%%%%%%%%%%%%%%%%%%%%%%%%%%%%%%%%%%%%%%%
\index{\textit{kon}!intransitives Verb}
%
Das intransitive Verb \textit{kon} bedeudet 'atmen'.

\begin{supertabular}{p{5,5cm}|ll}
jan ni li kon ike. && Dieser Mensch atmet schlecht. \\
\end{supertabular}

Dagegen bedeutet das transitive Verb \textit{kon} 'wegblasen'.

\begin{supertabular}{p{5,5cm}|ll}
mi kon e ilo suno. && Ich puste die Kerze aus. \\
\end{supertabular}

%%%%%%%%%%%%%%%%%%%%%%%%%%%%%%%%%%%%%%%%%%%%%%%%%%%%%%%%%%%%%%%%%%%%%%%%%%
\index{\textit{kama}!intransitives Verb}
%
Das intransitive Verb \textit{kama} bedeutet 'kommen' oder 'ankommen'.

\begin{supertabular}{p{5,5cm}|ll}
pona li kama. && Das Gute wird kommen. \\
\end{supertabular}

%%%%%%%%%%%%%%%%%%%%%%%%%%%%%%%%%%%%%%%%%%%%%%%%%%%%%%%%%%%%%%%%%%%%%%%%%%
\index{\textit{pakala}!intransitives Verb}
%
Das intransitive Verb \textit{pakala} bedeutet 'vermasseln', 'auseinander fallen' oder 'brechen'.

\begin{supertabular}{p{5,5cm}|ll}
tomo ni li pakala. && Dieses Haus f�llt auseinander. \\	
\end{supertabular}

%%%%%%%%%%%%%%%%%%%%%%%%%%%%%%%%%%%%%%%%%%%%%%%%%%%%%%%%%%%%%%%%%%%%%%%%%%
\index{\textit{sewi}!intransitives Verb}
%
Das intransitive Verb \textit{sewi} bedeutet 'aufstehen'.

\begin{supertabular}{p{5,5cm}|ll}
mi sewi. && Ich stehe auf. \\	
\end{supertabular}

%
%
%%%%%%%%%%%%%%%%%%%%%%%%%%%%%%%%%%%%%%%%%%%%%%%%%%%%%%%%%%%%%%%%%%%%%%%%%%
\subsection*{Intransitive Verben, Adverbien und Hilfsverben}
%
\index{Adverb}
\index{Umstandswort}
\index{Verb!Hilfs-}
\index{Hilfsverb}
%

Wir haben gelernt, dass ein Verb von einem Adverb modifiziert werden kann.
Dies gilt nat�rlich auch f�r intransitive Verben. 
In diesem Beispiel modifiziert das Adverb \textit{mute} das intransitive Verb \textit{lon.}

\begin{supertabular}{p{5,5cm}|ll}
mi lon mute tomo. && Ich bin oft im Haus. \\
\end{supertabular} 

Vor einem intransitiven Verb kann nat�rlich auch ein Hilfsverb stehen.

\begin{supertabular}{p{5,5cm}|ll}
mi wile lon tomo. && Ich m�chte im Haus sein. \\
\end{supertabular} 

%
%%%%%%%%%%%%%%%%%%%%%%%%%%%%%%%%%%%%%%%%%%%%%%%%%%%%%%%%%%%%%%%%%%%%%%%%%%
\newpage{}
%
\subsection*{�bungen (Antworten siehe Seite~\pageref{'indirect_objects'})}
%%%%%%%%%%%%%%%%%%%%%%%%%%%%%%%%%%%%%%%%%%%%%%%%%%%%%%%%%%%%%%%%%%%%%%%%%%
%
Schreibe bitte die Antworten auf einen Zettel und �berpr�fe sie anschlie�end. 

\begin{supertabular}{p{5,5cm}|ll}
Wie kann man nicht ein indirektes Objekt erfragen?  &&  \\ % no-dictionary
Welche Objektart wird stark vom Pr�dikat beeinflu�t? &&  \\ % no-dictionary
Zu welcher Phrase im Satz geh�rt das indirekte Objekt? &&  \\ % no-dictionary
Was f�r ein Slot steht an erster Position in einem indirekten Objekt? &&  \\ % no-dictionary
Wie nennt man Verben, die kein Objekt beeinflussen? &&   \\ % no-dictionary
Was steht vor einem indirekten Objekt in \textit{toki pona}? &&  \\ % no-dictionary
Wo ist ein Slot f�r ein adjektivisches Demonstrativpronomen m�glich? && . \\  % no-dictionary
Wo ist ein Slot f�r ein Hilfsverb? &&  \\ % no-dictionary
\end{supertabular}

Versuche diese S�tze zu �bersetzen. 
Mit dem Tool \textit{Toki Pona Parser} (\cite{www:rowa:02}) kann man Rechtschreibung und Grammatik �berpr�fen. 

\begin{supertabular}{p{5,5cm}|ll}
Dies ist f�r meinen Freund.   &&  \\  % no-dictionary
Die Werkzeuge sind im Container.  &&  \\  % no-dictionary
Diese Flasche ist im Dreck.  &&  \\  % no-dictionary
Sie streiten.  &&  \\  % no-dictionary
Die Frau gebar ihr Kind. &&   \\  % no-dictionary
\end{supertabular} 

%
%%%%%%%%%%%%%%%%%%%%%%%%%%%%%%%%%%%%%%%%%%%%%%%%%%%%%%%%%%%%%%%%%%%%%%%%%%
% eof
