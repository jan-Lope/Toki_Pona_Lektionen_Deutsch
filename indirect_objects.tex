%%%%%%%%%%%%%%%%%%%%%%%%%%%%%%%%%%%%%%%%%%%%%%%%%%%%%%%%%%%%%%%%%%%%%%%%%%
\section{Indirekte Objekte}
%%%%%%%%%%%%%%%%%%%%%%%%%%%%%%%%%%%%%%%%%%%%%%%%%%%%%%%%%%%%%%%%%%%%%%%%%%
\index{Indirektes Objekt}
\index{\textit{lon}}
\index{\textit{kepeken}}
\index{\textit{tawa}}
\index{\textit{kama}}
\index{\textit{kiwen}}
\index{\textit{kon}}
\index{\textit{pana}}
\index{\textit{poki}}
\index{\textit{toki}}
\index{in}
\index{im}
\index{an}
\index{auf}
\index{am}
\index{existieren}
\index{leben}
\index{benutzen}
\index{verwenden}
\index{mit}
\index{mittels}
\index{bewegen}
\index{fahren}
\index{gehen!zu etwas}
\index{f�r}
\index{zu}
\index{nach}
\index{bis}
\index{Bewegung}
\index{beweglich}
\index{kommen}
\index{passieren}
\index{verursachen}
\index{Zukunft}
\index{Anfang}
\index{schwer}
\index{fest}
\index{steinhart}
\index{Stein}
\index{Fels}
\index{Metall}
\index{Luft}
\index{Atmosph�re}
\index{Wind}
\index{Geist}
\index{Seele}
\index{luftig}
\index{gasf�rmig}
\index{geben}
\index{senden}
\index{freigeben}
\index{ausstrahlen}
\index{Verabreichung}
\index{Umtausch}
\index{Beh�lter}
\index{Container}
\index{Sch�ssel}
\index{Glas}
\index{Tasse}
\index{Kiste}
\index{sprechen}
\index{reden}
\index{tratschen}
\index{Sprache}
\index{Kommunikation}
%%%%%%%%%%%%%%%%%%%%%%%%%%%%%%%%%%%%%%%%%%%%%%%%%%%%%%%%%%%%%%%%%%%%%%%%%%
\subsection*{Vokabeln}
%%%%%%%%%%%%%%%%%%%%%%%%%%%%%%%%%%%%%%%%%%%%%%%%%%%%%%%%%%%%%%%%%%%%%%%%%%
\begin{supertabular}{p{2,5cm}|ll}
\textbf{kepeken} && \textit{Substantiv}: Nutzung, Verwendung, Werkzeug \\ % no-dictionary
\textbf{\dots , kepeken \dots} && \textit{Pr�position}: mit, mittels, per \\ % no-dictionary
\textbf{kepeken} && \textit{Verb, intransitiv}: benutzen, verwenden, anwenden, nutzen, gebrauchen \\ % no-dictionary
\textbf{kepeken \dots} && \textit{Hilfsverb}: verwenden \\ % no-dictionary
 && \\ % no-dictionary
\textbf{\dots kiwen} && \textit{Adjektiv}: hart, schwer, heftig, fest, solid, stabil, robust, steinhart \\ % no-dictionary
\textbf{\dots kiwen} && \textit{Adverb}: hart, schwer, heftig, fest, solid, stabil, robust, steinhart \\ % no-dictionary
\textbf{kiwen} && \textit{Substantiv}: Felsbrocken, Fels, Stein, Metall, Mineral, Lehm \\ % no-dictionary
\textbf{kiwen (e \dots)} && \textit{Verb, transitiv}: verfestigen, verh�rten, versteinern \\ % no-dictionary
-dictionary
\textbf{\dots kon} && \textit{Adjektiv}: luftig, �therisch, gasf�rmig, gasartig \\ % no-dictionary
\textbf{\dots kon} && \textit{Adverb}: luftig, �therisch, gasf�rmig, gasartig \\ % no-dictionary
\textbf{kon} && \textit{Substantiv}: Luft, Atmosph�re, Wind, Odor, Geruch, Seele, Geist \\ % no-dictionary
\textbf{kon} && \textit{Verb, intransitiv}: atmen \\ % no-dictionary
\textbf{kon (e \dots)} && \textit{Verb, transitiv}: wegblasen \\ % no-dictionary
 && \\ % no-dictionary
\textbf{\dots lon} && \textit{Adjektiv}: real, wahr, bestehend, richtig, echt \\ % no-dictionary
\textbf{lon} && \textit{Substantiv}: Existenz, Sein, Pr�senz \\ % no-dictionary
\textbf{\dots , lon \dots} && \textit{Pr�position}: in, im, an, am, bei, auf \\ % no-dictionary
\textbf{lon} && \textit{Verb, intransitiv}: anwesend sein, vorhanden sein, enthalten sein, existieren, leben \\ % no-dictionary
 && \\ % no-dictionary
\textbf{\dots pana} && \textit{Adjektiv}: gro�z�gig \\ % no-dictionary
\textbf{pana} && \textit{Substantiv}: Verabreichung, Transformation, �berf�hrung, Umtausch \\ % no-dictionary
\textbf{pana (e \dots)} && \textit{Verb, transitiv}: geben, senden, entbinden, strahlen, emittieren \\ % no-dictionary
 && \\ % no-dictionary
\textbf{poki} && \textit{Substantiv}: Kontainer, Beh�lter, Box, Sch�ssel, Schale, Tasse, (Trink-) Glas \\ % no-dictionary
\textbf{poki (e \dots)} && \textit{Verb, transitiv}: einpacken, einwecken, einwickeln, einmachen  \\ % no-dictionary
 && \\ % no-dictionary
\textbf{\dots tawa} && \textit{Adjektiv}: beweglich, mobil \\ % no-dictionary
\textbf{\dots tawa} && \textit{Adverb}: beweglich, mobil \\ % no-dictionary
\textbf{tawa} && \textit{Substantiv}: Bewegung, Regung, Transport \\ % no-dictionary
\textbf{\dots , tawa \dots} && \textit{Pr�position}: zu, um zu, zu hin, nach, f�r, bis, in \\ % no-dictionary
\textbf{tawa} && \textit{Verb, intransitiv}: gehen, spazieren, verlassen, abfahren, reisen \\ % no-dictionary
\textbf{tawa (e \dots)} && \textit{Verb, transitiv}: bewegen, verschieben, verlagern \\ % no-dictionary
\end{supertabular} \\
%
%%%%%%%%%%%%%%%%%%%%%%%%%%%%%%%%%%%%%%%%%%%%%%%%%%%%%%%%%%%%%%%%%%%%%%%%%%
\newpage
\subsection*{Indirekte Objekte und transitive Verben}
\index{Objekt!indirekte}
\index{Verb!intransitive}
\index{Verb!transitive}
%%%%%%%%%%%%%%%%%%%%%%%%%%%%%%%%%%%%%%%%%%%%%%%%%%%%%%%%%%%%%%%%%%%%%%%%%%
%
Wir haben schon direkte Objekte kennengelernt. 
Ein direktes Objekt ist von der Handlung (also dem transitiven Verb) am st�rksten beeinflusst. 
Das direkte Objekt (Akkusativobjekt) kann man mit 'Wen' oder 'Was' erfragen ('Was repariert Sie?').
Aber in dem Satz 'Ich bin im Haus.' ist 'im Haus' ein indirektes Objekt, da man kann es nicht mit 'Wen' oder 'Was' erfragen kann.
Auch wird es nicht direkt vom Pr�dikat beeinflusst. 
Ein indirektes Objekt ist auch Teil der Pr�dikat-Phrase. 
Im indirekten Objekt ist der erste Slot ist immer ein Substantiv- oder Pronomen-Slot.
Danach sind optionale Slots f�r Adjektive, Possessivpronomen und Demonstrativpronomen m�glich. 

Wir haben bereits transitive Verben kennen gelernt. 
Ein transitiven Verb macht etwas mit dem direkten Objekt. 
Dagegen werden Verben, die kein Objekt beinflussen, intransitive Verben genannt. 
Nach einem intransitiven Verb folgt entweder kein Objekt oder ein indirektes Objekt. 
In dem S�tzen 'Ich bin.' und 'Ich bin im Haus.' ist 'bin' ein intransitives Verb. 
In \textit{toki pona} steht zwischen intrasitiven Verb und indirekten Objekt kein \textit{e}.

%
%%%%%%%%%%%%%%%%%%%%%%%%%%%%%%%%%%%%%%%%%%%%%%%%%%%%%%%%%%%%%%%%%%%%%%%%%%
\index{Verb!\textit{lon}}
\index{\textit{lon}!Verb}
%
In diesem Beispielen wird \textit{lon} als intransitives Verb benutzt. 
Da vor \textit{lon} kein anderes Predikat steht, muss \textit{lon} ein Verb sein.

\begin{supertabular}{p{5,5cm}|ll}
mi lon tomo. && Ich bin im Haus. \\
suno li lon sewi. && Die Sonne ist am Himmel. \\
kili li lon poki. && Die Fr�chte sind im Korb. \\
\end{supertabular} 

%%%%%%%%%%%%%%%%%%%%%%%%%%%%%%%%%%%%%%%%%%%%%%%%%%%%%%%%%%%%%%%%%%%%%%%%%%
\index{Verb!\textit{kepeken}}
\index{\textit{kepeken}!Verb}
%
Hier dient \textit{kepeken} als intransitives Verb. 

\begin{supertabular}{p{5,5cm}|ll}
mi kepeken ilo. && Ich benutze ein Werkzeug. \\
sina wile kepeken ilo. && Du mu�t die Hilfsmittel verwenden. \\
mi kepeken poki ni. && Ich nehme diese Tasse. \\
\end{supertabular} 

In einigen anderen Lektionen wird \textit{kepeken} als transitives Verb verwendet. 
Dies liegt sicherlich daran, dass man mit 'Was' nach dem Objekt nach \textit{kepeken} fragen kann. 
Da aber das Objekt nicht unmittelbar vom Verb \textit{kepeken} beeinflusst wird, ist es ein indirektes Objekt und \textit{kepeken} ein intransitives Verb. 







%
%%%%%%%%%%%%%%%%%%%%%%%%%%%%%%%%%%%%%%%%%%%%%%%%%%%%%%%%%%%%%%%%%%%%%%%%%%
\newpage
\subsection*{�bungen (Antworten siehe Seite~\pageref{'indirect_objects'})}
%%%%%%%%%%%%%%%%%%%%%%%%%%%%%%%%%%%%%%%%%%%%%%%%%%%%%%%%%%%%%%%%%%%%%%%%%%
%
Schreibe bitte die Antworten auf einen Zettel und �berpr�fe sie anschlie�end. 



Versuche diese S�tze zu �bersetzen. 
Mit dem Tool \textit{Toki Pona Parser} (\cite{www:rowa:02}) kann man Rechtschreibung und Grammatik �berpr�fen. 

\begin{supertabular}{p{5,5cm}|ll}
Dies ist f�r meinen Freund.   &&  \\  % no-dictionary
Die Werkzeuge sind im Container.  &&  \\  % no-dictionary
Diese Flasche ist im Dreck.  &&  \\  % no-dictionary
Sie streiten.  &&  \\  % no-dictionary
\end{supertabular} 

%
%%%%%%%%%%%%%%%%%%%%%%%%%%%%%%%%%%%%%%%%%%%%%%%%%%%%%%%%%%%%%%%%%%%%%%%%%%
% eof
