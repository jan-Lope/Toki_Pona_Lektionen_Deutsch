%%%%%%%%%%%%%%%%%%%%%%%%%%%%%%%%%%%%%%%%%%%%%%%%%%%%%%%%%%%%%%%%%%%%%%%%%%
\section{Verneinung, Ja/Nein-Fragen}
%%%%%%%%%%%%%%%%%%%%%%%%%%%%%%%%%%%%%%%%%%%%%%%%%%%%%%%%%%%%%%%%%%%%%%%%%%
\subsection*{Vokabeln}
%%%%%%%%%%%%%%%%%%%%%%%%%%%%%%%%%%%%%%%%%%%%%%%%%%%%%%%%%%%%%%%%%%%%%%%%%%
\index{\textit{ala}}
\index{\textit{ali}}
\index{\textit{ken}}
\index{\textit{lape}}
\index{\textit{musi}}
\index{\textit{pali}}
\index{\textit{wawa}}
\index{nicht}
\index{nichts}
\index{kein}
\index{Negation}
\index{Null}
\index{nein}
\index{alles}
\index{alle}
\index{Universum}
\index{je}
\index{jeweils}
\index{k�nnen}
\index{M�glichkeit}
\index{F�higkeit}
\index{schlafen}
\index{liegen}
\index{schlafend}
\index{Schlaf}
\index{Spa�}
\index{am�sieren!sich}
\index{Spiel}
\index{Erholung}
\index{tun}
\index{machen}
\index{arbeiten}
\index{Aktivit�t}
\index{Arbeit}
\index{Projekt}
\index{stark}
\index{intensiv}
\index{Energie}
\index{Kraft}
\index{verst�rken}
\index{einschalten}
\begin{supertabular}{p{5,5cm}|ll}
ala && nicht, nichts, kein, Negation, Null, nein \\
ali && alles, alle, Universum, jeweils \\
ken && k�nnen, M�glichkeit, F�higkeit \\
lape && schlafen, liegen, schlafend, Schlaf \\
musi && Spa� (haben), sich am�sieren, Spiel, Erholung \\
pali && tun, machen, arbeiten, Aktivit�t, Arbeit, Projekt \\
wawa && stark, intensiv, Energie, Kraft, verst�rken, einschalten \\
\end{supertabular}  
%
%%%%%%%%%%%%%%%%%%%%%%%%%%%%%%%%%%%%%%%%%%%%%%%%%%%%%%%%%%%%%%%%%%%%%%%%%%
\index{Negation}
\index{nicht}
\index{\textit{ala}!Negation}
\subsection*{Verneinung}
%%%%%%%%%%%%%%%%%%%%%%%%%%%%%%%%%%%%%%%%%%%%%%%%%%%%%%%%%%%%%%%%%%%%%%%%%%
%
\textbf{ala als Adverb}

Im Deutschen kann ein Verb negiert werden, indem man ein 'nicht' hinzuf�gt. 
In \textit{toki pona} ist es ebenso.

\index{faul}
\begin{supertabular}{p{5,5cm}|ll}
mi lape \textbf{ala}. && Ich schlafe nicht. \\ 
mi musi \textbf{ala}. && Ich am�siere mich nicht. \\
mi wawa \textbf{ala}. && Ich bin nicht stark. \\
mi wile \textbf{ala} tawa musi. && Ich m�chte nicht tanzen. \\
tawa musi && tanzen \\
mi wile \textbf{ala} pali. && Ich bin faul. \\
\end{supertabular} 

\index{\textit{ala}!Adjektiv}
\index{Adjektiv!\textit{ala}}
\textbf{\textit{ala} als Adjektiv}

\index{schweigen}
\begin{supertabular}{p{5,5cm}|ll}
jan li toki ala. && Niemand redet. \\
\end{supertabular} 

\index{\textit{ala}!\textit{ijo}}
\index{\textit{ijo}!\textit{ala}}

Benutze niemals \textit{ala} mit \textit{ijo}.
Wenn es nicht hinter einem Verb steht, bedeutet ala einfach 'nichts'. 
Es w�re also unsinnig es mit \textit{ijo} zu benutzen.

\begin{supertabular}{p{5,5cm}|ll}
\textbf{ala} li jaki. && Nichts ist dreckig. \\
\end{supertabular} 
%
%%%%%%%%%%%%%%%%%%%%%%%%%%%%%%%%%%%%%%%%%%%%%%%%%%%%%%%%%%%%%%%%%%%%%%%%%%
\index{\textit{ali}}
\subsection*{\textit{ali}}
%%%%%%%%%%%%%%%%%%%%%%%%%%%%%%%%%%%%%%%%%%%%%%%%%%%%%%%%%%%%%%%%%%%%%%%%%%
%
Obwohl die Bedeutungen von \textit{ala} und \textit{ali} sehr unterschiedlich sind, werden sie als Adjektive gleich behandelt.
Deshalb wird hier auch \textit{ali} vorgestellt.

\index{Reisen}
\begin{supertabular}{p{5,5cm}|ll}
jan \textbf{ali} li wile tawa. && Jeder m�chte reisen. \\
ma \textbf{ali} li pona. && Aller L�nder sind gut. \\
jan utala \textbf{ali} li nasa. && Alle Soldaten sind bl�d. \\
\end{supertabular} 

\index{\textit{ali}!\textit{ijo}}
\index{\textit{ijo}!\textit{ali}}
Auch \textit{ali} sollte man nicht mit \textit{ijo} verwenden.

\begin{supertabular}{p{5,5cm}|ll}
\textbf{ali} li pona. && Alles in Ordnung. \\
\end{supertabular} 
%
%%%%%%%%%%%%%%%%%%%%%%%%%%%%%%%%%%%%%%%%%%%%%%%%%%%%%%%%%%%%%%%%%%%%%%%%%%
\newpage
\index{Fragen!ja, nein}
\index{ja, nein!question}
\subsection*{Ja/Nein-Fragen}
%%%%%%%%%%%%%%%%%%%%%%%%%%%%%%%%%%%%%%%%%%%%%%%%%%%%%%%%%%%%%%%%%%%%%%%%%%

Ja/Nein-Fragen werden in \textit{toki pona} nach einem einfachen Muster gebildet.
Der Satzbau �ndert sich nicht, au�er das nach dem Verb \textit{ala} eingef�gt und das Verb danach wiederholt wird. 
Beendet wird eine Frage mit einem Fragezeichen.

\begin{supertabular}{p{5,5cm}|ll}
sina \textbf{pona ala pona}? && Reparierst du (etwas)? \\
sina \textbf{ken ala ken} lape? && Kannst du schlafen? \\
ona li \textbf{lon ala lon} tomo? && Ist er im Haus? \\
ona li \textbf{tawa ala tawa} ma ike? && Ging er in das b�se Land? \\
sina \textbf{pana ala pana} e moku tawa jan lili? && Hast du dem Kind zu Essen gegeben? \\
pipi li \textbf{moku ala moku} e kili? && Fressen K�fer Fr�chte? \\
\end{supertabular} 

Da es in Toki Pona kein Wort f�r 'sein' gibt, lassen sich manche Ja/Nein-Fragen nicht mit \textit{ala} formulieren. Beispiel: 'Ist sie eine Mutter?'

\begin{supertabular}{p{5,5cm}|ll}
ona li \textbf{mama ala mama}? && Ist sie m�tterlich? \\
% mama ala.             && Nein, sie ist nicht m�tterlich. \\ % no-dictionary
\end{supertabular} 
%
%%%%%%%%%%%%%%%%%%%%%%%%%%%%%%%%%%%%%%%%%%%%%%%%%%%%%%%%%%%%%%%%%%%%%%%%%%
\index{Fragen!ja, nein!Antwort}
\index{Antwort!Fragen!ja, nein}
\index{ja}
\index{nein}
\subsection*{Antworten}
%%%%%%%%%%%%%%%%%%%%%%%%%%%%%%%%%%%%%%%%%%%%%%%%%%%%%%%%%%%%%%%%%%%%%%%%%%

Wenn die Antwort positiv ist, wiederholt man einfach das Verb der Frage.
Ist die Antwort negativ, so f�gt man zu dem Verb \textit{ala} hinzu.

\begin{supertabular}{p{5,5cm}|ll}
sina \textbf{wile ala wile} moku? && M�chtest du essen? \\ \textbf{wile} && Ja. \\ \textbf{wile ala} && Nein. \\
 && \\ % no-dictionary
sina \textbf{lukin ala lukin} e kiwen? && Siehst du den Felsen? \\ \textbf{lukin} && Ja. \\ \textbf{lukin ala} && Nein. \\
 && \\ % no-dictionary
sina \textbf{sona ala sona} e toki mi? && Hast du verstanden was ich sagte? \\ \textbf{sona} && Ja. \\ \textbf{sona ala} && Nein. \\
\end{supertabular}  
%
%%%%%%%%%%%%%%%%%%%%%%%%%%%%%%%%%%%%%%%%%%%%%%%%%%%%%%%%%%%%%%%%%%%%%%%%%%
\subsection*{�bungen 8 (Antworten siehe Seite~\pageref{'negation_yes_no_questions'})}
%%%%%%%%%%%%%%%%%%%%%%%%%%%%%%%%%%%%%%%%%%%%%%%%%%%%%%%%%%%%%%%%%%%%%%%%%%
%
\begin{supertabular}{p{5,5cm}|ll}
Du mu�t mir sagen warum! * &&   \\ % no-dictionary
Ist ein K�fer neben mir? &&    \\ % no-dictionary
Ich kann nicht schlafen. &&    \\ % no-dictionary
Ich m�chte nicht mit dir reden. &&    \\ % no-dictionary
Er ging nicht zum See. &&   \\ % no-dictionary
 && \\ % no-dictionary
sina wile ala wile pali? wile ala.  &&    \\ % no-dictionary
jan utala li seli ala seli e tomo?   &&   \\ % no-dictionary
jan lili li ken ala moku e telo nasa.   &&   \\ % no-dictionary
sina kepeken ala kepeken ni?  &&    \\ % no-dictionary
sina ken ala ken kama?   &&   \\ % no-dictionary
\end{supertabular} 

* Du mu�t mir den Grund sagen.
%%%%%%%%%%%%%%%%%%%%%%%%%%%%%%%%%%%%%%%%%%%%%%%%%%%%%%%%%%%%%%%%%%%%%%%%%%
% eof
