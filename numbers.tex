%%%%%%%%%%%%%%%%%%%%%%%%%%%%%%%%%%%%%%%%%%%%%%%%%%%%%%%%%%%%%%%%%%%%%%%%%%
\section{Zahlen}
%
\index{Zahl}
%
%%%%%%%%%%%%%%%%%%%%%%%%%%%%%%%%%%%%%%%%%%%%%%%%%%%%%%%%%%%%%%%%%%%%%%%%%%
\subsection*{Vokabeln}
%%%%%%%%%%%%%%%%%%%%%%%%%%%%%%%%%%%%%%%%%%%%%%%%%%%%%%%%%%%%%%%%%%%%%%%%%%
%
\begin{supertabular}{p{2,5cm}|ll}
%
\index{ala}
\textbf{\dots ala} && \textit{Adjektiv, Ziffer}: 0 \\ % no-dictionary
 && \\ % no-dictionary
%
\index{wan}
\textbf{\dots wan} && \textit{Adjektiv, Ziffer}: 1, eins, ein \\ % no-dictionary
\textbf{wan} && \textit{Substantiv}: Einheit, Element, Partikel, Teil, Abschnitt, St�ck \\ % no-dictionary
\textbf{wan (e \dots)} && \textit{Verb, transitiv}: einigen, vereinigen, zusammenf�hren \\ % no-dictionary
 && \\ % no-dictionary
%
\index{tu}
\textbf{\dots tu} && \textit{Adjektiv, Ziffer}: zwei, 2 \\ % no-dictionary
\textbf{tu} && \textit{Substantiv}: Duo, Paar \\ % no-dictionary
\textbf{tu (e \dots)} && \textit{Verb, transitiv}: verdoppeln, zweiteilen, teilen, zerschneiden \\ % no-dictionary
 && \\ % no-dictionary
%
\index{luka}
\textbf{\dots luka} && \textit{Adjektiv, Ziffer}: f�nf, 5 \\ % no-dictionary
 && \\ % no-dictionary
%
\index{mute}
\textbf{\dots mute} && \textit{Adjektiv, Ziffer}: 20 (offizielles Toki Pona Buch) \\ % no-dictionary
 && \\ % no-dictionary
%
\index{ale}
\textbf{\dots ale} && \textit{Adjektiv, Ziffer}: 100 (offizielles Toki Pona Buch) \\ % no-dictionary
 && \\ % no-dictionary
%
\index{esun}
\textbf{\dots esun} && \textit{Adjektiv}: marktf�hig, verk�uflich, absatzf�hig, Kauf-, Verkauf- \\ % no-dictionary
\textbf{esun} && \textit{Substantiv}: Markt, Shop \\ % no-dictionary
\textbf{esun (e \dots)} && \textit{Verb, transitiv}: kaufen, verkaufen, tauschen \\ % no-dictionary
 && \\ % no-dictionary
%
\index{nanpa}
\textbf{nanpa \dots} && \textit{Adjektiv, Ziffer}: Zum Bilden von Ordnungszahlen. \\ % no-dictionary
\textbf{nanpa} && \textit{Substantiv}: Zahl, Ziffer, Anzahl, Ordnungszahl \\ % no-dictionary
\textbf{nanpa (e \dots)} && \textit{Verb, transitiv}: z�hlen, nummerieren, beziffern, rechnen \\ % no-dictionary
 && \\ % no-dictionary
%
\index{weka}
\textbf{\dots weka} && \textit{Adjektiv}: weg, fort, entfernt, abwesend, fehlend, vers�umend \\ % no-dictionary
\textbf{weka} && \textit{Substantiv}: Abwesenheit (von), Fehlen, Mangel, Absenz \\ % no-dictionary
\textbf{weka (e \dots)} && \textit{Verb, transitiv}: wegwerfen, entfernen, abtragen,  \\ % no-dictionary
 && \\ % no-dictionary
%
\index{\#}
 && \\ % no-dictionary
\textbf{\#} && \textit{inoffiziell}: Zur Kennzeichnung von Zahlen \\ % no-dictionary
%
\end{supertabular} \\
%
%
%
%%%%%%%%%%%%%%%%%%%%%%%%%%%%%%%%%%%%%%%%%%%%%%%%%%%%%%%%%%%%%%%%%%%%%%%%%%
\newpage
%
\subsection*{Vermeide gro�e Zahlen}
%%%%%%%%%%%%%%%%%%%%%%%%%%%%%%%%%%%%%%%%%%%%%%%%%%%%%%%%%%%%%%%%%%%%%%%%%%
%
%
%

%
%
% Zahlen k�nnen nur Adjektive und keine Adverbien sein. 
% Wie wieder zu sehen ist, sind \textit{toki pona} Adjektive komplexer als Adverbien. 
%
%
%
%
%
Gr��ere Zahlen zu bilden ist in \textit{toki pona} umst�ndlich und man sollte diese m�glichst vermeiden.
%
%%%%%%%%%%%%%%%%%%%%%%%%%%%%%%%%%%%%%%%%%%%%%%%%%%%%%%%%%%%%%%%%%%%%%%%%%%
%
\subsection*{Ziffern}
%
%%%%%%%%%%%%%%%%%%%%%%%%%%%%%%%%%%%%%%%%%%%%%%%%%%%%%%%%%%%%%%%%%%%%%%%%%%
%
Es gibt nur wenige Ziffern.
Mit \textit{wan}, \textit{tu}, \textit{luka} lassen sich aber Zahlen bilden.

\begin{supertabular}{p{5,5cm}|ll}
wan && 1 \\ 
tu  && 2 \\ 
tu wan && 2 + 1 = 3 \\
tu tu && 2 + 2 = 4 \\
luka && 5 \\
luka wan && 5 + 1 = 6 \\
luka tu && 5 + 2 = 7 \\
luka tu wan && 5 + 2 + 1 = 8 \\
luka tu tu && 5 + 2 + 2 = 9 \\
luka luka && 5 + 5 = 10 \\
luka luka wan && 5 + 5 + 1 = 11 \\
luka luka tu && 5 + 5 + 2 = 12 \\
luka luka tu wan && 5 + 5 + 2 + 1 = 13 \\
luka luka tu tu && 5 + 5 + 2 + 2 = 14 \\
luka luka luka && 5 + 5 + 5 = 15 \\
mute wan && 20 + 1 = 21 (Wird selten verwendet.) \\
ale tu && 100 + 2 = 102 (Wird selten verwendet.) \\
\end{supertabular} 

Zahlen werden wie Adjektive verwendet. 
Werden sie zusammen mit anderen Adjektive verwendet, setzt man sie an das Ende. 
Nur Prononem kann man nach Zahlen einf�gen. 
Inoffiziell kann man ein Doppelkreuz vor Zahlen einf�gen. 

\begin{supertabular}{p{5,5cm}|ll}
jan \# luka tu && sieben Leute \\
jan lili \# tu wan && drei Kinder \\
\end{supertabular} 

Gr��ere Zahlen sind umst�ndlich zu bilden.
Die Zahl 25 w�rde z.B. \textit{luka luka luka luka luka} bedeuten.
Das Lied '99 Luftballon' w�re kaum zu �bersetzen.
\textit{toki pona} ist eben eine sehr einfache Sprache.
%
%%%%%%%%%%%%%%%%%%%%%%%%%%%%%%%%%%%%%%%%%%%%%%%%%%%%%%%%%%%%%%%%%%%%%%%%%%
% \newpage
%
\subsection*{Verwende \textit{mute} f�r gro�e Zahlen}
%
\index{\textit{mute}}
%%%%%%%%%%%%%%%%%%%%%%%%%%%%%%%%%%%%%%%%%%%%%%%%%%%%%%%%%%%%%%%%%%%%%%%%%%
%
\begin{supertabular}{p{5,5cm}|ll}
jan mute li kama. && Viele Leute kamen. \\
\end{supertabular} 

Nat�rlich ist das immer sehr ungenau. 
Es k�nnten 3 oder auch 3000 Leute sein. 
Man kann es aber etwas genauer sagen.
Wir haben gelernt, dass man ein Wort nicht wiederholen sollte. 
\textit{mute} und \textit{lili} werden aber von einigen Leuten bis zu drei mal wiederholt, um gr��ere Mengen darzustellen. 
Dies ist aber schlechter Stil. 
Besser ist es \textit{mute kin} zu verwenden. 

\begin{supertabular}{p{5,5cm}|ll}
jan mute kin li kama! && Sehr, sehr viel Leute kamen. \\
jan mute lili li kama. && Einige Leuten kamen. \\
\end{supertabular} 
%
%%%%%%%%%%%%%%%%%%%%%%%%%%%%%%%%%%%%%%%%%%%%%%%%%%%%%%%%%%%%%%%%%%%%%%%%%%
%
\subsection*{Ordnungszahlen}
%
%%%%%%%%%%%%%%%%%%%%%%%%%%%%%%%%%%%%%%%%%%%%%%%%%%%%%%%%%%%%%%%%%%%%%%%%%%
%
F�r Ordnungszahlen verwendet man \textit{nanpa}.

\begin{supertabular}{p{5,5cm}|ll}
jan nanpa tu tu && die vierte Person \\
ni li jan lili ona nanpa tu. && Dies ist ihr zweites Kind. \\
meli mi nanpa wan li ' nasa. && Meine erste Freundin war verr�ckt. \\
\end{supertabular} 
%
%%%%%%%%%%%%%%%%%%%%%%%%%%%%%%%%%%%%%%%%%%%%%%%%%%%%%%%%%%%%%%%%%%%%%%%%%%
%
\subsection*{Weitere Bedeutungen von \textit{wan} und \textit{tu}}
%
\index{\textit{wan}!Verb}
\index{\textit{tu}!Verb}
%%%%%%%%%%%%%%%%%%%%%%%%%%%%%%%%%%%%%%%%%%%%%%%%%%%%%%%%%%%%%%%%%%%%%%%%%%
%
\textit{wan} kann auch als Verb 'vereinigen' bedeuten.

\begin{supertabular}{p{5,5cm}|ll}
mi en meli mi li ' wan. && Meine Freundin und ich sind verheiratet. \\
jan pali pi ma ali o wan! && Proletarier aller L�nder vereinigt euch! \\
\end{supertabular} 


\textit{tu} bedeutet 'trennen' oder 'zerteilen'.

\begin{supertabular}{p{5,5cm}|ll}
o tu e palisa ni. && Zerbreche diesen Stock.  \\
\end{supertabular} 

%%%%%%%%%%%%%%%%%%%%%%%%%%%%%%%%%%%%%%%%%%%%%%%%%%%%%%%%%%%%%%%%%%%%%%%%%%
%
\subsection*{Der Sinn des Lebens}
%
\index{42}
%%%%%%%%%%%%%%%%%%%%%%%%%%%%%%%%%%%%%%%%%%%%%%%%%%%%%%%%%%%%%%%%%%%%%%%%%%

\begin{supertabular}{p{5,5cm}|ll}
ali li ' seme? &&  Die Frage nach dem Leben, dem Universum und dem ganzen Rest. \\
ni li ' mute mute tu.  && Die Antwort ist 42. \\
\end{supertabular} 
%
%
%
%%%%%%%%%%%%%%%%%%%%%%%%%%%%%%%%%%%%%%%%%%%%%%%%%%%%%%%%%%%%%%%%%%%%%%%%%%
\newpage
%
\subsection*{Verschiedenes}
%%%%%%%%%%%%%%%%%%%%%%%%%%%%%%%%%%%%%%%%%%%%%%%%%%%%%%%%%%%%%%%%%%%%%%%%%%
%
\index{\textit{weka}!Verb}
\subsubsection*{\textit{weka}}
weka bedeutet als Verb 'sich entledigen' oder 'entfernen'.

\begin{supertabular}{p{5,5cm}|ll}
o weka e len sina. && Zieh dich aus. \\
o weka e jan lili tan, ni. && Bring das Kind weg von hier. \\ 
ona li wile ala kute e ni. && Er soll das hier nicht h�ren. \\ 
\end{supertabular} 

\index{\textit{weka}!Adjektiv}
\index{\textit{weka}!Adverb}
\textit{weka} wird auch oft als Adjektiv oder Adverb benutzt.

\begin{supertabular}{p{5,5cm}|ll}
mi ' weka. && Ich war weg. \\
mi wile tawa weka. && Ich m�chte weggehen. \\
\end{supertabular} 

\textit{weka} kann auch 'weit weg' oder 'fern' bedeuten.

\begin{supertabular}{p{5,5cm}|ll}
tomo mi li ' weka, tan ni. && Mein Haus ist weit weg von hier. \\
ma Elopa li ' weka, tan ma Mewika. && Europa ist fern von den USA. \\
\end{supertabular} 

Mit \textit{weka ala} dr�ckst du aus, das etwas nah ist.

\begin{supertabular}{p{5,5cm}|ll}
weka ala && nah \\
ma Mewika li ' weka ala tan ma Kupa. && Die USA sind nicht weit weg von Kuba. \\
\end{supertabular} 
%
%%%%%%%%%%%%%%%%%%%%%%%%%%%%%%%%%%%%%%%%%%%%%%%%%%%%%%%%%%%%%%%%%%%%%%%%%%
%
\subsubsection*{\textit{esun}}
%
\index{\textit{esun}}
%%%%%%%%%%%%%%%%%%%%%%%%%%%%%%%%%%%%%%%%%%%%%%%%%%%%%%%%%%%%%%%%%%%%%%%%%%
%

\begin{supertabular}{p{5,5cm}|ll}
mi nanpa e mani mi, lon esun suli. && Ich z�hle mein Geld in einem Supermarkt. \\
\end{supertabular}

%
%%%%%%%%%%%%%%%%%%%%%%%%%%%%%%%%%%%%%%%%%%%%%%%%%%%%%%%%%%%%%%%%%%%%%%%%%%
\subsection*{�bungen (Antworten siehe Seite ~\pageref{'numbers'})}
%%%%%%%%%%%%%%%%%%%%%%%%%%%%%%%%%%%%%%%%%%%%%%%%%%%%%%%%%%%%%%%%%%%%%%%%%%
%
Schreibe bitte die Antworten auf einen Zettel und �berpr�fe sie anschlie�end. 

Versuche diese S�tze zu �bersetzen. 
Mit dem Tool \textit{Toki Pona Parser} (\cite{www:rowa:02}) kann man Rechtschreibung und Grammatik �berpr�fen. 

\begin{supertabular}{p{5,5cm}|ll}
Ich sah drei V�gel. \\ % no-dictionary
Viele Leute kommen. \\   % no-dictionary
Der Erste ist da. \\   % no-dictionary
Ich besitze zwei Autos. \\ % no-dictionary
Einige Leute kommen. \\  % no-dictionary  
Vereinigt euch!    \\ % no-dictionary
 && \\ % no-dictionary
mi weka e ijo tu ni.   \\ % no-dictionary
o tu.   \\ % no-dictionary
mi lukin e soweli luka. \\   % no-dictionary 
mi ' weka.   \\ % no-dictionary
\end{supertabular}
%%%%%%%%%%%%%%%%%%%%%%%%%%%%%%%%%%%%%%%%%%%%%%%%%%%%%%%%%%%%%%%%%%%%%%%%%%
% eof
