%%%%%%%%%%%%%%%%%%%%%%%%%%%%%%%%%%%%%%%%%%%%%%%%%%%%%%%%%%%%%%%%%%%%%%%%%%
\section{Zahlen}
%
%%%%%%%%%%%%%%%%%%%%%%%%%%%%%%%%%%%%%%%%%%%%%%%%%%%%%%%%%%%%%%%%%%%%%%%%%%
\subsection*{Vokabeln}
%%%%%%%%%%%%%%%%%%%%%%%%%%%%%%%%%%%%%%%%%%%%%%%%%%%%%%%%%%%%%%%%%%%%%%%%%%
%
\index{\textit{ala}}
\index{\textit{wan}}
\index{\textit{tu}}
\index{\textit{luka}}
\index{\textit{nanpa}}
\index{\textit{weka}}
\index{\textit{mute}}
\index{\textit{ali}}
\index{esun}
\index{5}
\index{2}
\index{1}
\index{0}
\index{null}
\index{nichts}
\index{eins}
\index{vereinigen}
\index{zwei}
\index{entzweien}
\index{auseinanderbringen}
\index{trennen}
\index{f�nf}
\index{Hand}
\index{Arm}
\index{Nummer}
\index{Zahl}
\index{Ziffer}
\index{Anzahl}
\index{Zahlen}
\index{Ziffer}
\index{entfernen}
\index{wegwerfen}
\index{weg}
\index{Abwesenheit}
\index{Mangel}
\index{entledigen!sich}
\index{fern}
\index{Markt}
\index{Gesch�ft}
\index{20}
\index{100}
\begin{supertabular}{p{5,5cm}|ll}
ala && null, nichts \\
wan && eins, vereinigen \\
tu && zwei, entzweien, auseinanderbringen, trennen \\
luka && f�nf, Hand, Arm \\
mute && viel, 20 (Wird selten verwendet.) \\
ali && alle, komplett,  100 (Wird selten verwendet.)\\
nanpa && Nummer, Zahl, Ziffer, Anzahl, z�hlen, rechnen, nummerieren \\
weka && entfernen, wegwerfen, weg, Abwesenheit, Mangel, sich entledigen, fern \\
esun && Markt, Gesch�ft \\
\end{supertabular} 
%
%%%%%%%%%%%%%%%%%%%%%%%%%%%%%%%%%%%%%%%%%%%%%%%%%%%%%%%%%%%%%%%%%%%%%%%%%%
\subsection*{Vermeide gro�e Zahlen}
%%%%%%%%%%%%%%%%%%%%%%%%%%%%%%%%%%%%%%%%%%%%%%%%%%%%%%%%%%%%%%%%%%%%%%%%%%
%
Gr��ere Zahlen zu bilden ist in \textit{toki pona} umst�ndlich und man sollte diese m�glichst vermeiden.
%
%%%%%%%%%%%%%%%%%%%%%%%%%%%%%%%%%%%%%%%%%%%%%%%%%%%%%%%%%%%%%%%%%%%%%%%%%%
\index{Zahl!gr��er}
\subsection*{Ziffern}
%%%%%%%%%%%%%%%%%%%%%%%%%%%%%%%%%%%%%%%%%%%%%%%%%%%%%%%%%%%%%%%%%%%%%%%%%%
%
Es gibt nur wenige Ziffern.
Mit \textit{wan}, \textit{tu}, \textit{luka} lassen sich aber Zahlen bilden.

\begin{supertabular}{p{5,5cm}|ll}
wan && 1 \\ 
tu  && 2 \\ 
tu wan && 2 + 1 = 3 \\
tu tu && 2 + 2 = 4 \\
luka && 5 \\
luka wan && 5 + 1 = 6 \\
luka tu && 5 + 2 = 7 \\
luka tu wan && 5 + 2 + 1 = 8 \\
luka tu tu && 5 + 2 + 2 = 9 \\
luka luka && 5 + 5 = 10 \\
luka luka wan && 5 + 5 + 1 = 11 \\
luka luka tu && 5 + 5 + 2 = 12 \\
luka luka tu wan && 5 + 5 + 2 + 1 = 13 \\
luka luka tu tu && 5 + 5 + 2 + 2 = 14 \\
luka luka luka && 5 + 5 + 5 = 15 \\
mute wan && 20 + 1 = 21 (Wird selten verwendet.) \\
ali tu && 100 + 2 = 102 (Wird selten verwendet.) \\
\end{supertabular} 

\index{Adjektiv!Zahlen}
\index{Zahlen!Adjektiv}
Zahlen werden wie Adjektive verwendet. 
Werden sie zusammen mit anderen Adjektive verwendet, setzt man sie an das Ende. 
Nur Prononem kann man nach Zahlen einf�gen. 
Optional kann man ein Komma vor Zahlen einf�gen. 

\begin{supertabular}{p{5,5cm}|ll}
jan, \textbf{luka tu} && sieben Leute \\
jan lili, \textbf{tu wan} && drei Kinder \\
\end{supertabular} 

Gr��ere Zahlen sind umst�ndlich zu bilden.
Die Zahl 25 w�rde z.B. \textit{luka luka luka luka luka} bedeuten.
Das Lied '99 Luftballon' w�re kaum zu �bersetzen.
\textit{toki pona} ist eben eine sehr einfache Sprache.
%
%%%%%%%%%%%%%%%%%%%%%%%%%%%%%%%%%%%%%%%%%%%%%%%%%%%%%%%%%%%%%%%%%%%%%%%%%%
\newpage
\index{\textit{mute}}
\subsection*{Verwende \textit{mute} f�r gro�e Zahlen}
%%%%%%%%%%%%%%%%%%%%%%%%%%%%%%%%%%%%%%%%%%%%%%%%%%%%%%%%%%%%%%%%%%%%%%%%%%
%
\begin{supertabular}{p{5,5cm}|ll}
jan \textbf{mute} li kama. && Viele Leute kamen. \\
\end{supertabular} 

Nat�rlich ist das immer sehr ungenau. 
Es k�nnten 3 oder auch 3000 Leute sein. 
Man kann es aber etwas genauer sagen.
Wir haben gelernt, dass man ein Wort nicht wiederholen sollte. 
\textit{mute} und \textit{lili} werden aber von einigen Leuten bis zu drei mal wiederholt, um gr��ere Mengen darzustellen. 
Dies ist aber schlechter Stil. 
Besser ist es \textit{mute kin} zu verwenden. 

\index{\textit{lili}!\textit{mute}}
\index{\textit{mute}!\textit{lili}}
\index{\textit{mute}!\textit{mute}}
\index{\textit{mute}!\textit{kin}}
\index{\textit{lili}!\textit{kin}}
\begin{supertabular}{p{5,5cm}|ll}
jan \textbf{mute kin} li kama! && Sehr, sehr viel Leute kamen. \\
jan \textbf{mute lili} li kama. && Einige Leuten kamen. \\
\end{supertabular} 
%
%%%%%%%%%%%%%%%%%%%%%%%%%%%%%%%%%%%%%%%%%%%%%%%%%%%%%%%%%%%%%%%%%%%%%%%%%%
\index{Zahl!Ordnungs-}
\index{Ordnungszahl}
\subsection*{Ordnungszahlen}
%%%%%%%%%%%%%%%%%%%%%%%%%%%%%%%%%%%%%%%%%%%%%%%%%%%%%%%%%%%%%%%%%%%%%%%%%%
%
F�r Ordnungszahlen verwendet man \textit{nanpa}.

\begin{supertabular}{p{5,5cm}|ll}
jan \textbf{nanpa} tu tu && die vierte Person \\
ni li jan lili ona \textbf{nanpa} tu. && Dies ist ihr zweites Kind. \\
meli mi \textbf{nanpa} wan li ' nasa. && Meine erste Freundin war verr�ckt. \\
\end{supertabular} 
%
%%%%%%%%%%%%%%%%%%%%%%%%%%%%%%%%%%%%%%%%%%%%%%%%%%%%%%%%%%%%%%%%%%%%%%%%%%
\index{\textit{wan}!Verb}
\index{\textit{tu}!Verb}
\index{Verb!\textit{wan}}
\index{Verb!\textit{tu}}
\subsection*{Weitere Bedeutungen von \textit{wan} und \textit{tu}}
%%%%%%%%%%%%%%%%%%%%%%%%%%%%%%%%%%%%%%%%%%%%%%%%%%%%%%%%%%%%%%%%%%%%%%%%%%
%
\index{vereinigen}
\textit{wan} kann auch als Verb 'vereinigen' bedeuten.

\index{heiraten}
\index{Hochzeit}
\begin{supertabular}{p{5,5cm}|ll}
mi en meli mi li ' \textbf{wan}. && Meine Freundin und ich sind verheiratet. \\
jan pali pi ma ali o wan! && Proletarier aller L�nder vereinigt euch! \\
\end{supertabular} 

\index{trennen}
\index{zerteilen}
\index{teilen}
\textit{tu} bedeutet 'trennen' oder 'zerteilen'.

\index{zerbrechen}
\begin{supertabular}{p{5,5cm}|ll}
o \textbf{tu} e palisa ni. && Zerbreche diesen Stock.  \\
\end{supertabular} 

%%%%%%%%%%%%%%%%%%%%%%%%%%%%%%%%%%%%%%%%%%%%%%%%%%%%%%%%%%%%%%%%%%%%%%%%%%
\index{Der Sinn des Lebens}
\index{42}
\subsection*{Der Sinn des Lebens}
%%%%%%%%%%%%%%%%%%%%%%%%%%%%%%%%%%%%%%%%%%%%%%%%%%%%%%%%%%%%%%%%%%%%%%%%%%

\begin{supertabular}{p{5,5cm}|ll}
ali li ' seme? &&  Die Frage nach dem Leben, dem Universum und dem ganzen Rest. \\
ni li ' \textbf{mute mute tu}.  && Die Antwort ist 42. \\
\end{supertabular} 
%
%%%%%%%%%%%%%%%%%%%%%%%%%%%%%%%%%%%%%%%%%%%%%%%%%%%%%%%%%%%%%%%%%%%%%%%%%%
\newpage
\subsection*{Verschiedenes}
%%%%%%%%%%%%%%%%%%%%%%%%%%%%%%%%%%%%%%%%%%%%%%%%%%%%%%%%%%%%%%%%%%%%%%%%%%
%
\index{\textit{weka}!Verb}
\index{entfernen}
\index{entledigen!sich}
\index{ausziehen}
\subsubsection*{\textit{weka}}
weka bedeutet als Verb 'sich entledigen' oder 'entfernen'.

\begin{supertabular}{p{5,5cm}|ll}
o \textbf{weka} e len sina. && Zieh dich aus. \\
o \textbf{weka} e jan lili tan, ni. && Bring das Kind weg von hier. \\ 
ona li wile ala kute e ni. && Er soll das hier nicht h�ren. \\ 
\end{supertabular} 

\index{\textit{weka}!Adjektiv}
\index{\textit{weka}!Adverb}
\index{Adjektiv!\textit{weka}}
\index{Adverb!\textit{weka}}
\textit{weka} wird auch oft als Adjektiv oder Adverb benutzt.

\begin{supertabular}{p{5,5cm}|ll}
mi ' \textbf{weka}. && Ich war weg. \\
mi wile tawa \textbf{weka}. && Ich m�chte weggehen. \\
\end{supertabular} 

\index{weit!weg}
\index{weg!weit}
\index{fern}
\textit{weka} kann auch 'weit weg' oder 'fern' bedeuten.

\begin{supertabular}{p{5,5cm}|ll}
tomo mi li ' \textbf{weka}, tan ni. && Mein Haus ist weit weg von hier. \\
ma Elopa li ' \textbf{weka}, tan ma Mewika. && Europa ist fern von den USA. \\
\end{supertabular} 

\index{nah}
Mit \textit{weka ala} dr�ckst du aus, das etwas nah ist.

\begin{supertabular}{p{5,5cm}|ll}
weka ala && nah \\
ma Mewika li ' \textbf{weka ala} tan ma Kupa. && Die USA sind nicht weit weg von Kuba. \\
\end{supertabular} 
%
\index{\textit{esun}}
\index{Markt}
\subsubsection*{\textit{esun}}

\begin{supertabular}{p{5,5cm}|ll}
mi nanpa e mani mi, lon esun suli. && Ich z�hle mein Geld in einem Supermarkt. \\
\end{supertabular}

%
%%%%%%%%%%%%%%%%%%%%%%%%%%%%%%%%%%%%%%%%%%%%%%%%%%%%%%%%%%%%%%%%%%%%%%%%%%
\subsection*{�bungen 16 (Antworten siehe Seite ~\pageref{'numbers'})}
%%%%%%%%%%%%%%%%%%%%%%%%%%%%%%%%%%%%%%%%%%%%%%%%%%%%%%%%%%%%%%%%%%%%%%%%%%
%
\begin{supertabular}{p{5,5cm}|ll}
Ich sah drei V�gel. \\ % no-dictionary
Viele Leute kommen. \\   % no-dictionary
Der Erste ist da. \\   % no-dictionary
Ich besitze zwei Autos. \\ % no-dictionary
Einige Leute kommen. \\  % no-dictionary  
Vereinigt euch!    \\ % no-dictionary
 && \\ % no-dictionary
mi weka e ijo tu ni.   \\ % no-dictionary
o tu.   \\ % no-dictionary
mi lukin e soweli luka. \\   % no-dictionary 
mi ' weka.   \\ % no-dictionary
\end{supertabular}
%%%%%%%%%%%%%%%%%%%%%%%%%%%%%%%%%%%%%%%%%%%%%%%%%%%%%%%%%%%%%%%%%%%%%%%%%%
% eof
