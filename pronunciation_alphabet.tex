%%%%%%%%%%%%%%%%%%%%%%%%%%%%%%%%%%%%%%%%%%%%%%%%%%%%%%%%%%%%%%%%%%%%%%%%%%
\label{'pronunciation_alphabet'}
\section{Aussprache, Alphabet und Satzzeichen}
%%%%%%%%%%%%%%%%%%%%%%%%%%%%%%%%%%%%%%%%%%%%%%%%%%%%%%%%%%%%%%%%%%%%%%%%%%
%
toki pona hat vierzehn Buchstaben: 

%%%%%%%%%%%%%%%%%%%%%%%%%%%%%%%%%%%%%%%%%%%%%%%%%%%%%%%%%%%%%%%%%%%%%%%%%%
\index{Alphabet}
\index{Aussprache}
\index{Konsonanten}
\subsection*{Konsonanten (9)}
%%%%%%%%%%%%%%%%%%%%%%%%%%%%%%%%%%%%%%%%%%%%%%%%%%%%%%%%%%%%%%%%%%%%%%%%%%

Die Konsonanten werden wie im Deutschen ausgesprochen.

\begin{supertabular}{p{2,5cm}|ll}
\textbf{Buchstabe}   &&    \textbf{Aussprache wie in} \\ % no-dictionary
k && \textbf{K}ohle \\ % no-dictionary
l && \textbf{L}ampe \\ % no-dictionary
m && \textbf{M}ama \\ % no-dictionary
n && \textbf{N}ashorn \\ % no-dictionary
p && \textbf{P}apa \\ % no-dictionary
s && \textbf{S}egel \\ % no-dictionary
t && \textbf{T}uch \\ % no-dictionary
w && \textbf{W}asser \\ % no-dictionary
j && \textbf{J}ahr \\ % no-dictionary
\end{supertabular} 

%%%%%%%%%%%%%%%%%%%%%%%%%%%%%%%%%%%%%%%%%%%%%%%%%%%%%%%%%%%%%%%%%%%%%%%%%%
\index{Vokale}
\subsection*{Vokale (5)}
%%%%%%%%%%%%%%%%%%%%%%%%%%%%%%%%%%%%%%%%%%%%%%%%%%%%%%%%%%%%%%%%%%%%%%%%%%
%
Auch die Vokale werden wie im Deutschen ausgesprochen:

\begin{supertabular}{p{2,5cm}|ll}
\textbf{Buchstabe}   &&    \textbf{Aussprache wie in} \\ % no-dictionary
a   &&  \textbf{A}ffe   \\ % no-dictionary
e   &&  \textbf{E}he   \\ % no-dictionary
i   &&  \textbf{I}nsel   \\ % no-dictionary
o   &&  \textbf{O}hr   \\ % no-dictionary
u   &&  \textbf{U}hr   \\ % no-dictionary
\end{supertabular} \\
%
%%%%%%%%%%%%%%%%%%%%%%%%%%%%%%%%%%%%%%%%%%%%%%%%%%%%%%%%%%%%%%%%%%%%%%%%%%
\subsection*{Gro�- und Kleinschreibung}
%%%%%%%%%%%%%%%%%%%%%%%%%%%%%%%%%%%%%%%%%%%%%%%%%%%%%%%%%%%%%%%%%%%%%%%%%%
%
Alle offiziellen W�rter werden immer, selbst am Satzanfang, klein geschrieben. 
Gro�buchstaben werden nur f�r inoffizielle W�rter benutzt. 
Dies sind z.B. Eigennamen, L�nder, Sprachen und Religionen. 
%
%%%%%%%%%%%%%%%%%%%%%%%%%%%%%%%%%%%%%%%%%%%%%%%%%%%%%%%%%%%%%%%%%%%%%%%%%%
\subsection*{Satzzeichen}
%%%%%%%%%%%%%%%%%%%%%%%%%%%%%%%%%%%%%%%%%%%%%%%%%%%%%%%%%%%%%%%%%%%%%%%%%%
%
Achte bitte immer auf korrekte Satzzeichen.
Falsche beziehungsweise fehlende Satzzeichen beeintr�chtigen die Verst�ndlichkeit.

\begin{supertabular}{p{2,5cm}|ll}
\textbf{.} && \textit{Separator}: Beendet wird ein Aussagesatz mit einem Punkt. \\ % no-dictionary
\textbf{!} && \textit{Separator}: Die Befehlsform und Interjektionen enden mit einem Ausrufungszeichen. \\ % no-dictionary
\textbf{?} && \textit{Separator}: Beendet wird eine Frage mit einem Fragezeichen. \\ % no-dictionary
\textbf{:} && \textit{Separator}: Ein Doppelpunkt trennt zwei (Halb)-S�tze. \\  % no-dictionary
\textbf{,} && \textit{Separator}: Ein Komma wird nach \glqq o\grqq verwendet wenn man Leute anspricht. \\ && Optional kann es vor einer Pr�position eingef�gt werden. \\ % no-dictionary
\end{supertabular} \\
%
%%%%%%%%%%%%%%%%%%%%%%%%%%%%%%%%%%%%%%%%%%%%%%%%%%%%%%%%%%%%%%%%%%%%%%%%%%
% eof
