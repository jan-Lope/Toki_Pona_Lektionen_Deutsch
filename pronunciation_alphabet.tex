%%%%%%%%%%%%%%%%%%%%%%%%%%%%%%%%%%%%%%%%%%%%%%%%%%%%%%%%%%%%%%%%%%%%%%%%%%
\label{'pronunciation_alphabet'}
\section{Alphabet, Satzzeichen}
%%%%%%%%%%%%%%%%%%%%%%%%%%%%%%%%%%%%%%%%%%%%%%%%%%%%%%%%%%%%%%%%%%%%%%%%%%
%
toki pona hat vierzehn Buchstaben: 

%%%%%%%%%%%%%%%%%%%%%%%%%%%%%%%%%%%%%%%%%%%%%%%%%%%%%%%%%%%%%%%%%%%%%%%%%%
\subsection*{Konsonanten (9)}
\index{Konsonant}
%%%%%%%%%%%%%%%%%%%%%%%%%%%%%%%%%%%%%%%%%%%%%%%%%%%%%%%%%%%%%%%%%%%%%%%%%%

Die Konsonanten werden wie im Deutschen ausgesprochen.

\begin{supertabular}{p{2cm}|ll}
\textbf{Buchstabe}   &&    \textbf{Aussprache wie in} \\ % no-dictionary
k && \textbf{K}ohle   \\ % no-dictionary
l && \textbf{L}ampe   \\ % no-dictionary
m && \textbf{M}ama    \\ % no-dictionary
n && \textbf{N}ashorn \\ % no-dictionary
p && \textbf{P}apa    \\ % no-dictionary
s && \textbf{S}egel   \\ % no-dictionary
t && \textbf{T}uch    \\ % no-dictionary
w && \textbf{W}asser  \\ % no-dictionary
j && \textbf{J}ahr    \\ % no-dictionary
\end{supertabular} 

%%%%%%%%%%%%%%%%%%%%%%%%%%%%%%%%%%%%%%%%%%%%%%%%%%%%%%%%%%%%%%%%%%%%%%%%%%
\subsection*{Vokale (5)}
\index{Vokal}
%%%%%%%%%%%%%%%%%%%%%%%%%%%%%%%%%%%%%%%%%%%%%%%%%%%%%%%%%%%%%%%%%%%%%%%%%%
%
Auch die Vokale werden wie im Deutschen ausgesprochen:

\begin{supertabular}{p{2cm}|ll}
\textbf{Buchstabe}   &&    \textbf{Aussprache wie in} \\ % no-dictionary
a   &&  \textbf{A}ffe   \\ % no-dictionary
e   &&  \textbf{E}he    \\ % no-dictionary
i   &&  \textbf{I}nsel  \\ % no-dictionary
o   &&  \textbf{O}hr    \\ % no-dictionary
u   &&  \textbf{U}hr    \\ % no-dictionary
\end{supertabular} 

%
%%%%%%%%%%%%%%%%%%%%%%%%%%%%%%%%%%%%%%%%%%%%%%%%%%%%%%%%%%%%%%%%%%%%%%%%%%
\subsection*{Gro�- und Kleinschreibung}
\index{Gro�schreibung}
\index{Kleinschreibung}
%%%%%%%%%%%%%%%%%%%%%%%%%%%%%%%%%%%%%%%%%%%%%%%%%%%%%%%%%%%%%%%%%%%%%%%%%%
%
Alle offiziellen W�rter werden immer, selbst am Satzanfang, klein geschrieben. 
Gro�buchstaben werden nur f�r inoffizielle W�rter benutzt. 
Dies sind z.B. Eigennamen, L�nder, Sprachen und Religionen. 
%
%%%%%%%%%%%%%%%%%%%%%%%%%%%%%%%%%%%%%%%%%%%%%%%%%%%%%%%%%%%%%%%%%%%%%%%%%%
\subsection*{Sonderzeichen}
\index{Sonderzeichen}
%%%%%%%%%%%%%%%%%%%%%%%%%%%%%%%%%%%%%%%%%%%%%%%%%%%%%%%%%%%%%%%%%%%%%%%%%%
%
\begin{supertabular}{p{2cm}|ll}
\textbf{.} && \textit{Separator}: Beendet wird ein Aussagesatz mit einem Punkt. \\ % no-dictionary
\textbf{!} && \textit{Separator}: Aufforderungss�tze und Ausrufes�tze enden mit einem Ausrufungszeichen. \\ % no-dictionary
\textbf{?} && \textit{Separator}: Beendet wird eine Frage (interrogativer Satz) mit einem Fragezeichen. \\ % no-dictionary
\textbf{:} && \textit{Separator}: Ein Doppelpunkt trennt einen Hinweis-Satz von einem Satz. \\  % no-dictionary
\textbf{,} && \textit{Separator}: Ein Komma wird nach 'o' verwendet wenn, man Leute anspricht. \\ && Optional kann es vor einer Pr�position eingef�gt werden. \\ % no-dictionary
\end{supertabular} \\

%
%%%%%%%%%%%%%%%%%%%%%%%%%%%%%%%%%%%%%%%%%%%%%%%%%%%%%%%%%%%%%%%%%%%%%%%%%%
\subsection*{Separatoren}
\index{Separator}
%%%%%%%%%%%%%%%%%%%%%%%%%%%%%%%%%%%%%%%%%%%%%%%%%%%%%%%%%%%%%%%%%%%%%%%%%%
%
In diesen Lektionen werden Sonderzeichen als Separatoren bezeichnet. 
Separatoren trennen Phrasen voneinander. 
Zum Beispiel trennt ein Punkt ein Satz vom n�chsten Satz. 
In \textit{toki pona} dienen auch spezielle W�rter als Separatoren. 
In anderen Lektionen werden diese W�rter auch als 'Partikel' bezeichnet.

%
%%%%%%%%%%%%%%%%%%%%%%%%%%%%%%%%%%%%%%%%%%%%%%%%%%%%%%%%%%%%%%%%%%%%%%%%%%
\subsection*{Satzarten}
\index{Satzarten}
\index{Aussagesatz}
\index{Deklarativsatz}
\index{Fragesatz}
\index{Interrogativ}
\index{Aufforderungssatz}
\index{Imperativ}
\index{Ausrufesatz}
\index{Interjektion}
\index{Exclamatory}
\index{Gru�formel}
\index{�berschrift}
\index{Headlin}
\index{Titel}
\index{Satzzeichen}

%%%%%%%%%%%%%%%%%%%%%%%%%%%%%%%%%%%%%%%%%%%%%%%%%%%%%%%%%%%%%%%%%%%%%%%%%%
%
\textit{toki pona} kennt, wie viele Sprachen, unterschiedliche Satzarten. 

Die meisten S�tze sind Aussages�tze (Deklarativs�tze) und enden mit einem Punkt. 
Aussages�tze sind S�tze, die eine Behauptung oder eine Annahme aufstellen, also eine Aussage �ber einen Sachverhalt machen, der wahr oder falsch sein k�nnte. 

Frages�tze (Interrogativ) formulieren eine Frage. 
Sie enden mit einem Fragezeichen. 

Aufforderungss�tze (Imperativ) sind S�tze, die einen Befehl formulieren. 
Sie enden mit einem Ausrufungszeichen. 

Ausrufes�tze (Interjektionen, Exclamatory) sind S�tze, die Bewunderung oder Verwunderung zum Ausdruck bringen. 
Dazu geh�ren auch Gru�formeln. 
Sie enden mit einem Ausrufungszeichen oder einem Punkt.

�berschriften (Headlines) sind meist keine vollst�ndigen S�tze und enden nicht mit einem Satzzeichen.

Achte bitte immer auf korrekte Satzzeichen.
Falsche beziehungsweise fehlende Satzzeichen beeintr�chtigen die Verst�ndlichkeit.

%
\newpage
%%%%%%%%%%%%%%%%%%%%%%%%%%%%%%%%%%%%%%%%%%%%%%%%%%%%%%%%%%%%%%%%%%%%%%%%%%
\subsection*{�bungen (Antworten siehe Seite~\pageref{'pronunciation_alphabet'})}
%%%%%%%%%%%%%%%%%%%%%%%%%%%%%%%%%%%%%%%%%%%%%%%%%%%%%%%%%%%%%%%%%%%%%%%%%%
%

Schreibe bitte die Antworten auf einen Zettel und �berpr�fe sie anschlie�end. 

\begin{supertabular}{p{5cm}|ll}
Was sind Separatoren? &&   \\ % no-dictionary
 && \\ % no-dictionary
Welche Phrase hat kein Satzzeichen am Ende? &&  \\ % no-dictionary
 && \\ % no-dictionary
Welcher Separator steht am Ende eines Aussagesatzes? && \\ % no-dictionary
 && \\ % no-dictionary
Wann werden offizielle \textit{toki pona} W�rter gro� geschrieben? && \\ % no-dictionary
 && \\ % no-dictionary
Was darf vor oder nach einem Separator meist nicht stehen? &&  \\ % no-dictionary
\end{supertabular} \\% no-dictionary

%
%%%%%%%%%%%%%%%%%%%%%%%%%%%%%%%%%%%%%%%%%%%%%%%%%%%%%%%%%%%%%%%%%%%%%%%%%%
% eof
