%%%%%%%%%%%%%%%%%%%%%%%%%%%%%%%%%%%%%%%%%%%%%%%%%%%%%%%%%%%%%%%%%%%%%%%%%%
\section{Einleitung}
%%%%%%%%%%%%%%%%%%%%%%%%%%%%%%%%%%%%%%%%%%%%%%%%%%%%%%%%%%%%%%%%%%%%%%%%%%
%
\textit{\textbf{toki!} }

Das Ziel der von Sonja Lang (2001) geschaffenen Sprache \textit{toki pona} ist der Minimalismus. 
\textit{toki pona} besteht aus nur etwa 120 W�rtern, die in ihrer Form nicht ver�ndert werden. 
Entsprechend ihrer Position im Satz k�nnen die W�rter ihre Wortart und damit auch in ihrer Bedeutung variieren. 
Um etwas genauer zu beschreiben, kombiniert man die W�rter. 

Es ist nicht das Ziel von \textit{toki pona}, komplexe Sachverhalte zu beschreiben. 
Dissertationen und wissenschaftliche Arbeiten werden nie in \textit{toki pona} verfasst. 
Juristen, B�rokraten, Theologen und Politiker seien vor den Nebenwirken dieser Sprache gewarnt. 

\textit{toki pona} ist also nicht angetreten, um das Verst�ndigungsproblem auf der Welt zu l�sen. 
Man kann diese Sprache aber in einem Monat lernen und f�r einen Kaffeeklatsch reicht es allemal. 
\textit{toki pona} ist auf intelligente Art einfach und Joga f�r das Gehirn. 
Wer sich f�r Sprachen interessiert, aber verschachtelte Nebens�tze und Kommas, hasst wird sicherlich Spa� an \textit{toki pona} haben. 

Vielleicht kann nur eine nat�rliche Sprache mit \textit{toki pona} verglichen werden.
Es ist die Sprache der Pirah\'{a} (\cite{www:piraha:01}).
Zum Beispiel hat diese Sprache keine Rekursion.

%
%% 
%
\textit{toki pona} hat sich seit 2001 weiterentwickelt. 
Diese Lektionen basieren daher auf den aktualisierten Lektionen von B. J. Knight (jan Pije) \cite{www:Pije:01} (2003, 2015) und dem offiziellen \textit{toki pona} Buch \cite{www:tokipona.org} von Sonja Lang (2014).  
Bei diesen Lektionen wird gro�er Wert auf die Darlegung grammatikalischer Regeln gelegt. 
So lassen sich Mi�verst�ndnisse durch fehlerhafte Grammatik vermeiden.

Also viel Spa� bei den Lektionen und beim Lernen von \textit{toki pona}! 
Beim Vokabelnlernen hilft Memrise \cite{www:memrise:01}. 
Ein W�rterbuch Deutsch-\textit{toki pona} findet man hier: \cite{www:rowa:01}. 
Weitere Links zu \textit{toki pona} findet man auf der Website \cite{www:rowa:01}.

Mit dem Tool \textit{Toki Pona Parser} (\cite{www:rowa:02}) kann man Rechtschreibung und Grammatik von \textit{toki pona}-S�tzen �berpr�fen. 

\textit{toki pona li pona, tawa sina.}

%
%%%%%%%%%%%%%%%%%%%%%%%%%%%%%%%%%%%%%%%%%%%%%%%%%%%%%%%%%%%%%%%%%%%%%%%%%%
% eof
