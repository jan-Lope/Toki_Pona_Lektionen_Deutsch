%%%%%%%%%%%%%%%%%%%%%%%%%%%%%%%%%%%%%%%%%%%%%%%%%%%%%%%%%%%%%%%%%%%%%%%%%%
\section{Farben}
%%%%%%%%%%%%%%%%%%%%%%%%%%%%%%%%%%%%%%%%%%%%%%%%%%%%%%%%%%%%%%%%%%%%%%%%%%
%
%%%%%%%%%%%%%%%%%%%%%%%%%%%%%%%%%%%%%%%%%%%%%%%%%%%%%%%%%%%%%%%%%%%%%%%%%%
\subsection*{Vokabeln}
%%%%%%%%%%%%%%%%%%%%%%%%%%%%%%%%%%%%%%%%%%%%%%%%%%%%%%%%%%%%%%%%%%%%%%%%%%
%
\index{\textit{jelo}}
\index{\textit{kule}}
\index{\textit{laso}}
\index{\textit{loje}}
\index{\textit{pimeja}}
\index{\textit{sitelen}}
\index{\textit{walo}}
\index{anstreichen}
\index{f�rben}
\index{Farbe}
\index{farbenfreudig}
\index{bunt}
\index{zeichnen}
\index{schreiben}
\index{malen}
\index{Bild}
\index{Gem�lde}
\index{Foto}
\index{gelb}
\index{hellgr�n}
\index{blau}
\index{cyan}
\index{rot}
\index{schwarz}
\index{dunkel}
\index{Dunkelheit}
\index{Finsternis}
\index{schw�rzen}
\index{wei�}
\index{hell}
\index{Helligkeit}
\begin{supertabular}{p{5,5cm}|ll}
kule && anstreichen, f�rben, Farbe, farbenfreudig, bunt \\ 
sitelen && zeichnen, schreiben, malen, Bild, Gem�lde, Foto \\
jelo && gelb, hellgr�n \\
laso && blau, cyan \\
loje && rot \\
pimeja && schwarz, dunkel, Dunkelheit, Finsternis, schw�rzen \\
walo && wei�, hell, Helligkeit \\
\end{supertabular} 
%
%%%%%%%%%%%%%%%%%%%%%%%%%%%%%%%%%%%%%%%%%%%%%%%%%%%%%%%%%%%%%%%%%%%%%%%%%%
\index{Farbe}
\index{Farbe!Kombinationen}
\index{lila}
\index{gr�n}
\index{grau}
\subsection*{Farbkombinationen}
\subsubsection*{\textit{jelo}, \textit{laso}, \textit{loje}, \textit{pimeja} und \textit{walo} als Adjektive}

%%%%%%%%%%%%%%%%%%%%%%%%%%%%%%%%%%%%%%%%%%%%%%%%%%%%%%%%%%%%%%%%%%%%%%%%%%
%
In \textit{toki pona} gibt es keine W�rter f�r die Farben Lila, Gr�n, Grau usw.
Man kann aber Farben aus mehreren W�rtern bilden.

\begin{supertabular}{p{5,5cm}|ll}
laso loje && lila \\
laso jelo && gr�n \\
loje jelo && orange \\
loje walo && rosa  \\
walo pimeja && grau \\
\end{supertabular} 

Man kann auch Farben aus mehr als zwei W�rtern mischen.
Das Ziel von \textit{toki pona} ist aber die Einfachheit.
Vermeide daher komplexe Wortzusammensetzungen.
Die Reihenfolge der Farben ist �brigens egal.

\begin{supertabular}{p{5,5cm}|ll}
laso loje / loje laso && lila \\
laso jelo / jelo laso && gr�n \\
\end{supertabular} 
%
%% \includegraphics[scale=0.5]{sikekule3.png}
%
%%%%%%%%%%%%%%%%%%%%%%%%%%%%%%%%%%%%%%%%%%%%%%%%%%%%%%%%%%%%%%%%%%%%%%%%%%
\index{Farbe!\textit{pi}}
\index{Farbe!\textit{en}}
\index{\textit{pi}!Farbe}
\index{\textit{en}!Farbe}
\index{Farbe!Muster}
\index{Muster!Farben}
\subsubsection*{Farbenmuster mit \textit{pi} und \textit{en}}
%%%%%%%%%%%%%%%%%%%%%%%%%%%%%%%%%%%%%%%%%%%%%%%%%%%%%%%%%%%%%%%%%%%%%%%%%%
%
Stelle dir ein rotes T-Shirt mit blauen Streifen in der Mitte vor. 
Man kann es nicht \textit{len loje laso} nennen, denn das w�rde ja ein lila T-Shirt beschreiben. 
Um die Farben zu trennen verwenden wir \textit{en}.
Um den bunten Gegenstand von den Farben zu trennen dient \textit{pi}.
\textit{loje} und \textit{laso} sind hier Substantive. 

\begin{supertabular}{p{5,5cm}|ll}
len \textbf{pi} loje \textbf{en} laso && ein T-Shirt mit Rot und Blau \\
\end{supertabular} 
%
%%%%%%%%%%%%%%%%%%%%%%%%%%%%%%%%%%%%%%%%%%%%%%%%%%%%%%%%%%%%%%%%%%%%%%%%%%
\index{\textit{kule}!\textit{seme}}
\index{was!f�r eine Farbe}
\index{Farbe!was f�r eine}
\subsection*{\textit{kule}}
%%%%%%%%%%%%%%%%%%%%%%%%%%%%%%%%%%%%%%%%%%%%%%%%%%%%%%%%%%%%%%%%%%%%%%%%%%
%
\index{\textit{kule}!Substantiv}
\index{Substantiv!\textit{kule}}
\textbf{\textit{kule}  als Substantiv} \\
\begin{supertabular}{p{5,5cm}|ll}
ni li \textbf{kule seme?} && Was ist dies f�r eine Farbe? \\
\end{supertabular} 

\index{\textit{kule}!Verb}
\index{Verb!\textit{kule}}
\index{malen}
\index{streichen!Farbe}
\textbf{\textit{kule} als Verb} \\
\begin{supertabular}{p{5,5cm}|ll}
mi \textbf{kule} e lipu &&Ich bemale das Blatt. \\
\end{supertabular} 
%
%%%%%%%%%%%%%%%%%%%%%%%%%%%%%%%%%%%%%%%%%%%%%%%%%%%%%%%%%%%%%%%%%%%%%%%%%%
\newpage
\subsection*{Verschiedenes}
\subsubsection*{sitelen}
%%%%%%%%%%%%%%%%%%%%%%%%%%%%%%%%%%%%%%%%%%%%%%%%%%%%%%%%%%%%%%%%%%%%%%%%%%
%
\index{\textit{sitelen}}
\index{Bild}
\index{Gem�lde}
\index{malen}
\index{schreiben}

\index{Bild!bewegt}
\index{bewegte Bilder}
\index{Film}
\index{Fernsehen!Show}
\index{Show!Fernsehen}
\index{Landkarte}
\index{Karte!Land-}
\index{\textit{sitelen}!\textit{ma}}
\index{\textit{ma}!\textit{sitelen}}
Als Substantiv bedeutet \textit{sitelen} 'Bild' oder 'Gem�lde' und als Verb 'malen' oder 'schreiben'.
Nat�rlich kann man auch mit \textit{sitelen} zusammengesetzte Substantiv bilden.

\begin{supertabular}{p{5,5cm}|ll}
\textbf{sitelen tawa}  && Film (bewegte Bilder) \\
\textbf{sitelen tawa} 'Fahrenheit 9/11' li pona, tawa mi. && Ich mag den Film 'Fahrenheit 9/11'. \\
\textbf{sitelen tawa} 'Bowling for Columbine' li pona kin. && Der Film 'Bowling for Columbine' ist auch gut. \\
\textbf{sitelen ma} && Landkarte \\
o pana e \textbf{sitelen ma}, tawa mi. && Gib mir die Landkarte. \\
\end{supertabular} 
%
%%%%%%%%%%%%%%%%%%%%%%%%%%%%%%%%%%%%%%%%%%%%%%%%%%%%%%%%%%%%%%%%%%%%%%%%%%
\subsection*{�bungen 13 (Antworten siehe ~\pageref{'colors'})}
%%%%%%%%%%%%%%%%%%%%%%%%%%%%%%%%%%%%%%%%%%%%%%%%%%%%%%%%%%%%%%%%%%%%%%%%%%
%
\begin{supertabular}{p{5,5cm}|ll}
Ich sehe die blaue Tasche nicht. &&   \\ % no-dictionary
Kleine gr�ne Menschen kamen vom Himmel. &&   \\ % no-dictionary
Ich mag die Farbe Lila.  &&  \\ % no-dictionary
Der Himmel ist blau. &&   \\ % no-dictionary
Siehe diesen roten K�fer.  &&  \\ % no-dictionary
Ich brauche die Landkarte.  &&  \\ % no-dictionary
Schaust du dir 'Akte-X' an? &&  \\  % no-dictionary
Welche Farbe magst du? &&  \\  % no-dictionary
 && \\ % no-dictionary
suno li jelo. &&   \\ % no-dictionary
telo suli li laso.  &&  \\ % no-dictionary
mi wile moku e kili loje.  &&  \\ % no-dictionary
ona li kule e tomo tawa. &&   \\ % no-dictionary
\end{supertabular} 

Ein kleines Gedicht.

\begin{supertabular}{p{5,5cm}|ll}
ma mi li pimeja. && \\ % no-dictionary
kalama ala li lon && \\ % no-dictionary
mi lape. mi sona. && \\ % no-dictionary
\end{supertabular} 
%%%%%%%%%%%%%%%%%%%%%%%%%%%%%%%%%%%%%%%%%%%%%%%%%%%%%%%%%%%%%%%%%%%%%%%%%%
% eof
