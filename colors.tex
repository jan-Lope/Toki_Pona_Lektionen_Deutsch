%%%%%%%%%%%%%%%%%%%%%%%%%%%%%%%%%%%%%%%%%%%%%%%%%%%%%%%%%%%%%%%%%%%%%%%%%%
\section{Farben}
%%%%%%%%%%%%%%%%%%%%%%%%%%%%%%%%%%%%%%%%%%%%%%%%%%%%%%%%%%%%%%%%%%%%%%%%%%
%
%%%%%%%%%%%%%%%%%%%%%%%%%%%%%%%%%%%%%%%%%%%%%%%%%%%%%%%%%%%%%%%%%%%%%%%%%%
\subsection*{Vokabeln}
%%%%%%%%%%%%%%%%%%%%%%%%%%%%%%%%%%%%%%%%%%%%%%%%%%%%%%%%%%%%%%%%%%%%%%%%%%
%
\begin{supertabular}{p{2,5cm}|ll}
%
\index{jelo}
\textbf{\dots jelo} && \textit{Adjektiv}: gelb, hellgr�n \\ % no-dictionary
\textbf{jelo} && \textit{Substantiv}: Gelb, Hellgr�n \\ % no-dictionary
 && \\ % no-dictionary
%
\index{kule}
\textbf{\dots kule} && \textit{Adjektiv}: farbenfreudig \\ % no-dictionary
\textbf{kule} && \textit{Substantiv}: Farbe, Lack \\ % no-dictionary
\textbf{kule (e \dots)} && \textit{Verb, transitiv}: f�rben, anstreichen, streichen \\ % no-dictionary
 && \\ % no-dictionary
%
\index{laso}
\textbf{\dots laso} && \textit{Adjektiv}: blau, cyan \\ % no-dictionary
\textbf{laso} && \textit{Substantiv}: Blau, Cyan \\ % no-dictionary
 && \\ % no-dictionary
%
\index{loje}
\textbf{\dots loje} && \textit{Adjektiv}: rot, r�tlich \\ % no-dictionary
\textbf{loje} && \textit{Substantiv}: Rot \\ % no-dictionary
 && \\ % no-dictionary
%
\index{pimeja}
\textbf{\dots pimeja} && \textit{Adjektiv}: schwarz, dunkel \\ % no-dictionary
\textbf{pimeja} && \textit{Substantiv}: Dunkelheit, Finsternis, Schw�rze, Schatten \\ % no-dictionary
\textbf{pimeja (e \dots)} && \textit{Verb, transitiv}: verdunkeln, schw�rzen \\ % no-dictionary
 && \\ % no-dictionary
%
\index{sitelen}
\textbf{\dots sitelen} && \textit{Adjektiv}: bildlich, figurative, bildhafte, metaphorisch \\ % no-dictionary
\textbf{\dots sitelen} && \textit{Adverb}: bildlich \\ % no-dictionary
\textbf{sitelen} && \textit{Substantiv}: Bild, Abbildung, Foto, Film, Darstellung, Gem�lde \\ % no-dictionary
\textbf{sitelen (e \dots)} && \textit{Verb, transitiv}: zeichnen, malen, schreiben \\ % no-dictionary
 && \\ % no-dictionary
%
\index{walo}
\textbf{\dots walo} && \textit{Adjektiv}: wei�, hell, bleich \\ % no-dictionary
\textbf{walo} && \textit{Substantiv}: Helligkeit, Wei�e \\ % no-dictionary
\textbf{walo (e \dots)} && \textit{Verb, transitiv}: bleichen \\ % no-dictionary
%
\end{supertabular} \\
%
%%%%%%%%%%%%%%%%%%%%%%%%%%%%%%%%%%%%%%%%%%%%%%%%%%%%%%%%%%%%%%%%%%%%%%%%%%
\newpage
%
\subsection*{Farbkombinationen}
\index{Farbe}
%
\subsubsection*{\textit{jelo}, \textit{laso}, \textit{loje}, \textit{pimeja} und \textit{walo} als Adjektive}

%%%%%%%%%%%%%%%%%%%%%%%%%%%%%%%%%%%%%%%%%%%%%%%%%%%%%%%%%%%%%%%%%%%%%%%%%%
%
In \textit{toki pona} gibt es keine W�rter f�r die Farben Lila, Gr�n, Grau usw.
Man kann aber Farben aus mehreren W�rtern bilden.

\begin{supertabular}{p{5,5cm}|ll}
laso loje && lila \\
laso jelo && gr�n \\
loje jelo && orange \\
loje walo && rosa  \\
walo pimeja && grau \\
\end{supertabular} 

Man kann auch Farben aus mehr als zwei W�rtern mischen.
Das Ziel von \textit{toki pona} ist aber die Einfachheit.
Vermeide daher komplexe Wortzusammensetzungen.
Die Reihenfolge der Farben ist �brigens egal.

\begin{supertabular}{p{5,5cm}|ll}
laso loje / loje laso && lila \\
laso jelo / jelo laso && gr�n \\
\end{supertabular} 
%
%% \includegraphics[scale=0.5]{sikekule3.png}
%
%%%%%%%%%%%%%%%%%%%%%%%%%%%%%%%%%%%%%%%%%%%%%%%%%%%%%%%%%%%%%%%%%%%%%%%%%%
\subsubsection*{Farbenmuster mit \textit{pi} und \textit{en}}
%
%%%%%%%%%%%%%%%%%%%%%%%%%%%%%%%%%%%%%%%%%%%%%%%%%%%%%%%%%%%%%%%%%%%%%%%%%%
%
Stelle dir ein rotes T-Shirt mit blauen Streifen in der Mitte vor. 
Man kann es nicht \textit{len loje laso} nennen, denn das w�rde ja ein lila T-Shirt beschreiben. 
Um die Farben zu trennen verwenden wir \textit{en}.
Um den bunten Gegenstand von den Farben zu trennen dient \textit{pi}.
\textit{loje} und \textit{laso} sind hier Substantive. 

\begin{supertabular}{p{5,5cm}|ll}
len pi loje en laso && ein T-Shirt mit Rot und Blau \\
\end{supertabular} 
%
%
%%%%%%%%%%%%%%%%%%%%%%%%%%%%%%%%%%%%%%%%%%%%%%%%%%%%%%%%%%%%%%%%%%%%%%%%%%
\subsection*{\textit{kule}}
%
\index{\textit{kule}!Substantiv}
\index{\textit{kule}!Verb}
%%%%%%%%%%%%%%%%%%%%%%%%%%%%%%%%%%%%%%%%%%%%%%%%%%%%%%%%%%%%%%%%%%%%%%%%%%
%
\textbf{\textit{kule}  als Substantiv} \\
\begin{supertabular}{p{5,5cm}|ll}
ni li ' kule seme? && Was ist dies f�r eine Farbe? \\
\end{supertabular} 

\textbf{\textit{kule} als Verb} \\
\begin{supertabular}{p{5,5cm}|ll}
mi kule e lipu && Ich bemale das Blatt. \\
\end{supertabular} 
%
%%%%%%%%%%%%%%%%%%%%%%%%%%%%%%%%%%%%%%%%%%%%%%%%%%%%%%%%%%%%%%%%%%%%%%%%%%
% \newpage
\subsection*{Verschiedenes}
\subsubsection*{sitelen}
%
\index{\textit{sitelen}}
%%%%%%%%%%%%%%%%%%%%%%%%%%%%%%%%%%%%%%%%%%%%%%%%%%%%%%%%%%%%%%%%%%%%%%%%%%
%
Als Substantiv bedeutet \textit{sitelen} 'Bild' oder 'Gem�lde' und als Verb 'malen' oder 'schreiben'.
Nat�rlich kann man auch mit \textit{sitelen} zusammengesetzte Substantiv bilden.

\begin{supertabular}{p{5,5cm}|ll}
sitelen tawa  && Film (bewegte Bilder) \\
sitelen tawa 'Fahrenheit 9/11' li pona, tawa mi. && Ich mag den Film 'Fahrenheit 9/11'. \\
sitelen tawa 'Bowling for Columbine' li pona kin. && Der Film 'Bowling for Columbine' ist auch gut. \\
sitelen ma && Landkarte \\
o pana e sitelen ma, tawa mi. && Gib mir die Landkarte. \\
\end{supertabular} 
%
\newpage
%%%%%%%%%%%%%%%%%%%%%%%%%%%%%%%%%%%%%%%%%%%%%%%%%%%%%%%%%%%%%%%%%%%%%%%%%%
\subsection*{�bungen (Antworten siehe ~\pageref{'colors'})}
%%%%%%%%%%%%%%%%%%%%%%%%%%%%%%%%%%%%%%%%%%%%%%%%%%%%%%%%%%%%%%%%%%%%%%%%%%
%
Schreibe bitte die Antworten auf einen Zettel und �berpr�fe sie anschlie�end. 

Versuche diese S�tze zu �bersetzen. 
Mit dem Tool \textit{Toki Pona Parser} (\cite{www:rowa:02}) kann man Rechtschreibung und Grammatik �berpr�fen. 

\begin{supertabular}{p{5,5cm}|ll}
Ich sehe die blaue Tasche nicht. &&   \\ % no-dictionary
Kleine gr�ne Menschen kamen vom Himmel. &&   \\ % no-dictionary
Ich mag die Farbe Lila.  &&  \\ % no-dictionary
Der Himmel ist blau. &&   \\ % no-dictionary
Siehe diesen roten K�fer.  &&  \\ % no-dictionary
Ich brauche die Landkarte.  &&  \\ % no-dictionary
Schaust du dir 'Akte-X' an? &&  \\  % no-dictionary
Welche Farbe magst du? &&  \\  % no-dictionary
 && \\ % no-dictionary
suno li ' jelo. &&   \\ % no-dictionary
telo suli li ' laso.  &&  \\ % no-dictionary
mi wile moku e kili loje.  &&  \\ % no-dictionary
ona li kule e tomo tawa. &&   \\ % no-dictionary
\end{supertabular} 

Ein kleines Gedicht.

\begin{supertabular}{p{5,5cm}|ll}
ma mi li ' pimeja. && \\ % no-dictionary
kalama ala li lon && \\ % no-dictionary
mi lape. mi sona. && \\ % no-dictionary
\end{supertabular} 
%%%%%%%%%%%%%%%%%%%%%%%%%%%%%%%%%%%%%%%%%%%%%%%%%%%%%%%%%%%%%%%%%%%%%%%%%%
% eof
