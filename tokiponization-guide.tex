%%%%%%%%%%%%%%%%%%%%%%%%%%%%%%%%%%%%%%%%%%%%%%%%%%%%%%%%%%%%%%%%%%%%%%%%%%
\section{Phonetische Transkription von Namen}
%
\label{'phonet_trans'}
%%%%%%%%%%%%%%%%%%%%%%%%%%%%%%%%%%%%%%%%%%%%%%%%%%%%%%%%%%%%%%%%%%%%%%%%%%
%
Um Eigennamen an die toki pona-Aussprache anzupasen, sind die folgenden Regeln zu beachten.
Siehe dazu auch das Alphabet und die Aussprache (Seite~\pageref{'pronunciation_alphabet'}).
Ein Hilfsmittel zur Transkription von Namen findet man unter tokipona.net \cite{www:tokipona.net:01}.

\begin{itemize}
 \item 
Es ist besser die 'Idee' des Wortes zu �bersetzen als den Namen in ein unverst�ndliches toki pona-Korsett zu zw�ngen. 
Beispiel: Jean Chr�tien, Ministerpr�sident von Kanada = 'jan lawa pi ma Kanata' ist besser als 'jan Kesijen'.
 \item 
Verwende die Original-Aussprache und nicht die Schreibweise als Grundlage.
 \item 
Wird der Name in mehr als in einer Sprache verwendet, nehme die dominanteste Form.
 \item 
Geh�rt der Name zu keiner Sprache, verwende die internationale Form. 
Beispiel: 'Atlantik' = 'Alansi'.
 \item 
Bevorzuge die umgangssprachliche Form gegen�ber der offiziellen Aussprache.
Beispiel: 'Toronto' = 'Towano', nicht 'Tolonto'.
 \item 
Wenn jemand seinen eigenen Namen an toki pona anpassen will, so braucht er sich dazu nicht exakt an diese Regeln halten.
 \item 
Namen von Nationen, Sprachen und Religionen sind bereits umgewandelt.
Fehlt ein Name, dann mache einen Vorschlag in der toki pona-Diskussions-Liste.
 \item 
Wenn m�glich finde die allgemeine Wurzel des Namens f�r die Nation, die Sprache und den Menschen des Landes. 
Beispiel: England, englisch, Engl�nder = 'Inli'.
 \item 
St�dte und Orte k�nnen einen 'offiziellen' toki pona-Namen erhalten, wenn diese international bekannt sind. 
 \item 
Niemand zwingt dich einem tokiponisierte Namen zu benutzen. 
Das ist reiner Spa�. 
\end{itemize} 

\subsection*{Silben Inoffizielle W�rter}

\begin{itemize}
 \item
Jede Silbe besteht aus einem Konsonanten und einem Vokal plus einem optionalen \textit{n}.
 \item
Der n�chste Silbe nach einer optionalen \textit{n} kann nicht mit einem \textit{n} beginnen.
 \item
Die erste Silbe eines Wortes muss nicht mit einem Konsonanten beginnen.
 \item
Die Silben \textit{ti} und \textit{tin} werden zu \textit{si} und \textit{sin}.
 \item
Der Konsonant \textit{w} kann nicht vor \textit{o} oder \textit {u} verwendet werden.
 \item
Der Konsonant \textit {j} kann nicht vor \textit {i} verwendet werden.
\end{itemize} 

%
%%%%%%%%%%%%%%%%%%%%%%%%%%%%%%%%%%%%%%%%%%%%%%%%%%%%%%%%%%%%%%%%%%%%%%%%%%
\newpage
\subsection*{Phonetische Regeln}
%%%%%%%%%%%%%%%%%%%%%%%%%%%%%%%%%%%%%%%%%%%%%%%%%%%%%%%%%%%%%%%%%%%%%%%%%%
%
\begin{itemize}
 \item 
Stimmhafte Konsonanten werden stimmlos. 
Beispiele: 'b' = 'p', 'd' = 't', 'g' = 'k'.
 \item 
'v' wird zu 'w'.
 \item 
'f' wird zu 'p'.
 \item 
Das 'r' der meisten Sprachen wird zu 'l'.
 \item 
Das 'r' in Sprachen, �hnlich der englischen Sprache, wird zu 'w'.
 \item 
Jeder Uvular- oder Velar-Konsonant wird zu 'k', einschlie�lich dem franz�sischen und deutschen 'r'.
 \item 
Ein 'sh' oder 'sch' am Ende eines Wortes sollte zu 'si' werden.
Beispiel: Lush = Lusi.
% \item 
%The schwa can become any vowel in Toki Pona and is often influenced by neighbouring vowels for cute %reduplication.
 \item 
Es ist besser die Anzahl der Silben zu erhalten und ein Konsonant zu entfernen, als ein neuen Vokal einzuf�gen.
Beispiel: 'Chuck' = 'Sa', nicht 'Saku'.
 \item
Bei mehreren, hintereinander folgenden Konsonanten, sollte der dominante Konsonant erhalten bleiben.
Entferne zuerst Frikativlaute, wie 's' und 'l'.
Beispiel: 'Esperanto' = 'Epelanto' 
Erhalte auch einen Konsonant am Anfang einer neuen Silbe.
Beispiel: 'Atling' = 'Alin'.
 \item 
Approximants wie 'j' und 'w' in Konsonant-Cluster k�nnern entweder in eine eigene Silbe konvertiert werden  (Beispiel: 'Swe' = 'Suwe'; 'Pju' = 'Piju') oder entfernt werden ('Swe' = 'Se'; 'Pju' = 'Pu').
 \item 
In einigen F�llen ist es besser den Buchstaben leicht zu �ndern, statt ihn zu entfernen.
Beispiel: 'Lubnan' = 'Lunpan', nicht 'Lupan' oder 'Lunan'.
 \item 
Das englische 'th' kann in 't' oder 's' konvertiert werden.
 \item 
Die illegalen Silben 'ti', 'wo' und 'wu' werden zu 'si', 'o' und 'u' konvertiert.
Beispiel: 'Antarktika' = 'Antasika'.
 \item 
Affricates werden generell in Ficatives konvertiert.
Beispiel: 'John' = 'San', nicht 'Tan')
 \item 
Jeder nasale Konsonant am Ende einer Silbe wird zu 'n' konvertiert.
Beispiel: 'Fam' = 'Pan'.
 \item 
Nasale Vokale (franz�sisch, portugiesisch) werden auch zu 'n' konvertiert.
 \item 
Wenn es notwendig ist die Silbenstruktur zu erhalten, dann k�nnen die Konsonanten 'w' oder 'j' als euphonische �berg�nge eingef�gt werden.
Beispiel: 'Tai' = 'Tawi'; 'Nihon' = 'Nijon'; 'Eom' = 'Ejon'.
Es ist auch m�glich den Konsonant zu verschieben, den man sonst entfernen m��te.
Beispiel: 'Monkeal' = 'Monkela', nicht 'Monkeja', 'Euska' = 'Esuka'.
 \item 
Stimmlose Lateral-Konsonanten werden zu 's' konvertiert.
 \item 
Wenn notwendig, muss ein Wort ver�ndert werden, um Missverst�ndnisse zu vermeiden.
Beispiel: 'Allah' = 'jan sewi Ila', nicht 'jan sewi Ala' = kein Gott.
Wenn m�glich verwende ein verwandtes Wort der Ursprungssprache.
\end{itemize} 
% 
%%%%%%%%%%%%%%%%%%%%%%%%%%%%%%%%%%%%%%%%%%%%%%%%%%%%%%%%%%%%%%%%%%%%%%%%%%
% eof
